% changed order of word better 210506
\chapter{Reasons for Hope}
\addcontentsline{toc}{chapterdescription}{We are entering the next great shift. The meaning you make of the shift shapes the reality you experience, and the emergent strategies you can use. The lenses Picasso and Einstein used to create great shifts in art and physics can guide us today in developing adaptive capacity in yourself, your organisation, and your economy.}
\addcontentsline{toc}{chapterdescription}{\pagebreak}
\label{chapter:reasons-for-hope}


\begin{chapterquotation}
Every valuable human being must be a radical and a rebel, for what they must aim at is to make things better than they are.\\
\raggedleft\textemdash Niels Bohr\cite{bohr-radical}\index{Bohr, Neils}


\centering
You may never know what results come from your action. But if you do nothing, there will be no result\\
\raggedleft\textemdash Mahatma Gandhi\index{Gandhi, Mahatma}
\end{chapterquotation}


\section{Shift happens}
Are you experiencing climate anxiety\index{climate!anxiety}? Are you experiencing climate grief?\index{climate!grief} Are you tempted to give up hope\index{hope}, and join the ranks of the doomists preparing for complete system collapse? Are you actively doing whatever you can to prevent even more harm, and to regenerate our environment and society?


We are at the beginning of the biggest shift \index{shift} \index{climate!emergency} that humanity has experienced in at least 75,000 years, since the eruption of the Toba volcano caused so much pollution that it triggered a rapid change in the global climate, and humankind faced near extinction. This time it may even be a more severe shift.


I am doing this chapter a few months after 2019 broke a number of records for the hottest ever, and for the biggest fires across swathes of Australia and Western USA. A year of ever more ice loss from glaciers, the Arctic\index{Arctic} and Antarctic,\index{Antarctic} microparticle pollution, ocean-wide plastic pollution, modern-day slavery, in fact so much that we could fill the entire book just listing everything. And now, as we do the final proofs, the virus SARS-CoV-2 is causing Covid-19 and ravaging most countries; another warning shot (after SARS-CoV-1, MERS etc.) across our bows, warning of the unbelievably bigger threats to come.


This is a Do It Yourself (DIY) book, filled with grounded reasons for hope, even if you seldom feel hopeful. By the end you will know why now is the best time ever, and exactly what you can do yourself to build a society, an economy, and businesses that are fundamentally regenerative for you, the environment, and profit. Why we all can rise to the greatest challenge humanity has faced in tens of thousands of years, and why our choices now can make this decade humanity's finest. 


Or not. The choice is ours.


It starts with you, and it starts with us. It starts with how we work together, how we incorporate business, and it starts with our economy. It starts everywhere, because we are talking about a circle, not a straight line. So start where you are; put one chapter of this book into practice, and then continue round the circle. 


Wherever you start, rising to the challenge of the shift happening around you demands that you shift who you are, how you think, and how you make meaning. You could start here, where you have the most control.


We are all in emergent times, which means none of us can predict the future from the present. In emergent times, at the start of a planet-wide shift, doing better what worked before is almost certain to make things worse, faster.


Doing differently at this scale requires each of us to become someone different, especially to give up hope of being certain. We need to shift our being to rise to emergent challenges. Emergence means we cannot know what will be, only what is, and what has been. It means that uncertainty reigns, and that the reality we each experience is inherently and irreducibly nebulous, (excellently described by David Chapman\cite{chapman-meaningness-nebulosity}\index{Chapman, David}). 


Shift\index{shift} happens, and shift is happening. Some people have given up; but most likely you have not. You are reading this because you are concerned, and actively searching for something practical to do. You may be feeling frustration, anxiety, perhaps even depression, but you are also motivated to act and make a difference.


You might believe that a back-to-basics\index{back-to-basics} approach will work. Replicating what worked in the past but trying harder. However, in our emergent reality, where shift happens that you cannot control nor predict, everything is up for grabs. Solutions that once worked, work no longer. Roads that used to look smooth and clearly marked have turned into unmarked paths petering out on impassable rocky slopes.


I have written this book so that you, your organisations, and our economy can thrive in these emergent times. 


I have scanned the past for similar emergent times where great shift\index{shift}  happened to see what worked and what did not. I have also explored a number of disciplines today that will enable you and our organisations to develop the emergent capacities needed to thrive in today's emergent times.


I start by looking at what each of us can learn from art and physics. Picasso and Einstein were key players in the transformation of art and physics a century ago. The lessons they learnt in driving that transformation are of great relevance to today. 


They found that dialogue\index{dialogue}, pluralism\index{pluralism}, and multiple perspectives are essential. They also recognised that the interwoven tapestry of the meaning\hyp{}making stories\index{meaning-making stories}\textemdash those which each individual uses to attribute meaning, and to thereby create their experience of reality\textemdash shaped what we could see and understand. They recognised that understanding differently, in order to act differently, required an inner shift in their identity or being, and hence in their meaning\hyp{}making stories.


To understand yourself, and to understand how to develop your emergent capacities, I've added the latest in neuroscience\index{neuroscience}, psychology, \index{psychology} and the other social sciences. I've also included the latest thinking from management, leadership, and economics so that you can develop new ways of constructing organisations to thrive in these emergent times of great shift. 


You are reading a book that is designed to span multiple disciplines, and spans the scales from the microscopic quantum level, through to your size, and onto the size of our global organisations and ecosystems.


In the past, most people experienced most of the time a predictable world. The world that each of us is experiencing today has fundamentally changed; it is now inherently unpredictable and emergent. This means that there is always more than one correct view, more than one correct answer. It means that you can never fully capture any view or answer by looking at the present and the past. You can only fully determine it by looking backwards from the future after the future has emerged.


When shift is happening in nebulous emergent times, the shadows the emergent future throws on the present are as powerful a force shaping the present as the past. For example, how are your hopes or fears of what may happen already determining what you do today?


So you can no longer thrive by trying to first gather as much relevant information as possible (from your present and past) to then predict what course of action is right. The outcomes, however much data you have, and however accurately you try to predict, are always filled with inherent, irreducible uncertainty. 


If you only learn more skills to predict better you will still fail, because you will still be acting from within the box of your current meaning\hyp{}making stories. You need to build a bigger box of stories, by developing your adaptive capacity, and the adaptive capacity of your organisation, through getting increasingly better at recognising and embracing contradiction and ambiguity. 


It has now become clear that neoclassical economics is flawed. As you will see reading this book, and as Kate Raworth\index{Raworth, Kate}\cite{raworth-doughnut} describes in her Guardian article\cite{raworth-guardian}, part of what has gone wrong in classical physics is the belief that it can serve as a basis for something as nebulous and subjective as the economy of a human society. In this book, we explore how the lenses of quantum physics and relativity can help us ask different questions that can lead us to better answers.


\section{You make meaning}
A central theme of this book is that you, like all other human beings, are extremely good at seeing patterns and making meaning. Regardless of whether the pattern and meaning is actually there, purely your fantasy, or more often somewhere in between.


Being human centres on your capacity to make meaning. You and a friend may look at exactly the same sunset (right now Jack and I are looking at the sun setting behind the foothills of the Drakensberg in South Africa\index{South Africa}). You make meaning of beauty and inspiration looking at the sunset, while your friend makes meaning of air pollution. If I am there, I will also tell you that the sunset means blue light is scattered far more than red light by the micro fluctuations of density in the air.


All of these meanings have truth, if only for you in your reality.


Even though the sun setting is a complete illusion you create from the earth turning on its axis. 


You make meaning by looking at your library of meaning\hyp{}making templates, and choosing one of them to make meaning in that moment. Throughout this book we will constantly be pointing at how we all actively make meaning using our own personal and unique library of possible meaning\hyp{}making templates. 


You are the only person experiencing exactly the reality\cite{sci-am-reality} and meaning you experience as you read this book. You truly are unique, and living in your own unique reality (but, of course, broad swathes of your experienced reality are very close to the unique experienced realities of many others).


Few of these stories are purely your fantasy, almost all of them have some grounding. Equally, none of these stories can ever be complete, containing all meanings that anyone alive today, in the past, or in the future, could make. 


Your meaning making shapes the reality you experience out of the raw material that is actually present. Raw material like the sunset behind the mountains.


\begin{longstoryblock}
I (Graham) remember growing up as a white teenager in apartheid South Africa\index{South Africa} during the 70s. A country divided into regions according to skin colour and tribe. The inhumane apartheid clearly could not continue much longer, but how would it end? Unable to grasp the emergent nature of such a shift, everything I predicted involved revolution and collapse. I saw no reason to hope for anything else. 


Lying in bed at night, outside sounds would trigger fear, keeping me awake, afraid. Fear and hopelessness were pervasive for myself and many others who wanted and needed a better South Africa, but doubted a smooth and peaceful route to it could exist.  


On 11 February 1990, everything changed for good. Nelson Mandela\index{Mandela, Nelson} was released from prison without nearly the violence I had feared. Unique in world history: when had the founder of the armed wing of an anti-government struggle movement been released peacefully? Even better, Mandela was already leading dialogue between groups previously unwilling to talk, laying the foundation for a truth and reconciliation process, for restorative rather than punitive justice. 


Mandela, de Klerk\index{de Klerk, F.W.}, Tutu\index{Tutu, Desmond}, and many South Africans rejected worn-out maps and lenses, and invented a new emergent strategy. Suddenly there was hope: hope that a new South Africa, that all South Africans could participate in and prosper, might emerge.


I learned three important lessons growing up in South Africa: much of the world is emergent, not predictable; fear is not the only emotional response to an impending great shift: hope is equally plausible and attainable; and, we can equip ourselves to thrive in an emergent world. When we do, radically new options emerge. 
\end{longstoryblock}


I hope this shows why it is an unhelpful, self-fulfilling prophecy to choose the doomist ‘we are doomed, total systems collapse is the only possible outcome’ stance: because the meaning\hyp{}making you use co-creates the reality you experience. There is always potential in an emergence-filled world.
\section{Emergent strategies}
\label{section:emergent-strategies}
\index{strategies!emergent}
If you are focusing on generating outcomes that you cannot see, predict, or control from where you stand now, and are maximising the chance of seeing the value of the outcomes as they emerge\textemdash and making use of them just in time\textemdash you have an emergent strategy. 


Choose wisely how you make meaning of what is happening now, so that you can adopt an emergent strategy that gives you the best chance of a good outcome. The late Barbara Marx Hubbard\index{Hubbard, Barbara Marx} was a pioneer of making meaning of the crises and chaos of today as the birthing of a new era of humanity\cite{hubbard-birth}, rather than the end of civilisation. Since we will only know after we have done our best which of the two it is; and the choice of meaning we make determines what we do; choose the meaning making most likely to enable the outcome you desire. 


The Theory U approach, popularised by Otto Scharmer\index{Scharmer, Otto}\cite{scharmer-U} and derived from the scientific approach that led to quantum physics, is a superb way of choosing and then finding wise action. The essence is to let go of what you know, poke at things, see what happens, and then learn. 


So chip away at the meaning\hyp{}making stories\index{meaning-making stories} shaping your reality so that you can better see what actually is, not what you want to see. As you've just read above, your meaning\hyp{}making stories play a central role in shaping the reality that you experience\textemdash your internally shaped experience of reality is always a simplification and distortion of what actually is. You will read a lot more about this in Chapters~\ref{chapter:who-am-i-base} to~\ref{chapter:who-am-i-one}, and especially in Section~\ref{section:biases} on cognitive biases. 


Nothing else works if you are in an emergent context, because the events streaming your way are inherently and unpredictably nebulous and complex, unfolding in new and unfamiliar ways. If your context is unpredictable and new, your strategy must be capable of generating new outcomes that you cannot predict before you start acting. Your actions, your tactics, and often even your strategy itself, can only be articulated just in time. Your choices and actions generate unpredictable and unfamiliar outcomes, which in turn generate unfamiliar and unpredictable changes in the subsequent strategy and actions. 


An emergent strategy creates a prosperous future through innovation and taking ownership of creating something new. It depends on you developing your adaptive capacity, not just your skills or competencies. Taking a back-to-basics\index{back-to-basics} approach, insisting on doing harder or better what worked well for you in the past, will not work. This is blind faith, not grounded hope\index{hope}.


\begin{longstoryblock}
Unfortunately, during the final year of writing this book, I (Jack) was going through a very emotional divorce\textemdash emotional because we still loved each other, but after a marriage of 26 years we could no longer live together. 


Initially, we tried to rekindle the vestiges of old sparks by recreating what had worked so well, so long ago: revisiting our honeymoon, and even wearing a favorite suit of mine that my wife had bought for our rehearsal dinner. 


Sadly, we realized that back- to-basics didn’t work. Circumstances had changed and we had changed, so rather than going back to the past and replicating what had magically worked then, we needed to move to the present. Were we deluded by blind faith? No, not really; just an honest attempt to recreate what had once worked for us, but, alas, that magical past only existed in memory.


\end{longstoryblock}


You can find no better example than Kodak\index{Kodak}, whose scientists developed the digital camera in 1975\cite{winger-innovation}. They built the first commercial digital camera (the Apple QuickTake) in 1994, but didn't dare let anyone know. In 2005 Kodak was the largest manufacturer of digital cameras in the world. By 2012 it was bankrupt. 


What went wrong? The emergent digital photography stories were incompatible with Kodak's meaning\hyp{}making story. Too many of Kodak’s management and shareholders assumed that the company would prosper by sticking with the highly profitable past (emulsion-based photographic film) and were confident that they could control the pace of change. They failed to see the emergent change, and failed to follow an emergent strategy.


When you have a successful emergent strategy, you neither abandon nor emulate the past. You give up on the hope that establishing a successful path to the future means doing better what you already know. You give up hope that you can identify what is right by finding the right lens to look at your world through. 


Emergent strategies require you to use multiple lenses to view reality, and to work with multiple disciplines in an integrated way. You will need to work simultaneously with contradictory beliefs, recognising that they are all needed in order to make sense of your emergent reality and so that you can have generative dialogues with others who have widely different views from yours. 


An emergent strategy is 90\% about the stories, only 10\% about the facts. We write the stories, and so we choose the future; we can choose the future we want, as Christiana Figueres \index{Figueres, Christiana}and Tom Rivett-Carnac \index{Rivett-Carnac, Tom} describe in their book \emph{The Future We Choose: Surviving the Climate Crisis}\cite{figueres-future}\index{The Future We Choose: Surviving the Climate Crisis}.


Developing the capacity to use multiple lenses was at the core of the transformation of art and physics a century ago. Whether you are more artistic, or more scientific, you can learn from the parallels in the crises you are facing today with those faced by artists and physicists back then.


\section{Applying Picasso and Einstein}
\label{section:applying-picasso-einstein}
\subsection{Classical}
The story here begins with the Italian artist Giotto di Bondone\index{di Bondone, Giotto} (1276-1337), who convinced everyone (well, almost everyone) that art should have only one perspective: a human perspective. Before Giotto, European artists were expected to portray reality from God's perspective, without depth.


But for Giotto, a painting should begin and end with what you would see as a human being. To do this on a two-dimensional canvas he portrayed the depth of a three-dimensional landscape by making distant objects smaller than near objects, and parallel lines move closer together the further away they were in the landscape. 


The outcome was that all lines of sight converge on your eye as a viewer of the painting, in exactly the same way they would do if you were standing at the spot where the artist had stood in the physical landscape\cite{shlain-art-and-physics}.


So art became technical, based on geometry\index{geometry}  and measurable laws. Giotto shifted from God-centric to human-centric. But his goal was still to paint one, privileged, geometric-centred perspective, not the nuanced perspective of all of you and your diverse meaning\hyp{}making stories.


Similarly, Isaac Newton\index{Newton, Isaac} revolutionised physics\index{physics}  by using a new lens \index{lens} to see everything that happened on earth and in space. He and his peers developed the deterministic world of classical physics\index{physics!classical}. 


Newton convincingly demonstrated that nature’s deterministic laws could be discovered by combining human reason and mathematics with the hard unambiguous data coming from observation and measurement.


His \emph{Principia}, published in 1687, provided the foundation of the new worldview, gave hope and understanding to everyone, and led to the Industrial Revolution.


And so a new reality was born, with a can-do spirit of optimism.


Giotto \index{di Bondone, Giotto}  and Newton \index{Newton, Isaac} used similar thinking and triggered a great shift in art and physics. The worlds of art and physics complemented each other: both were based on Euclid's geometry, and both co-evolved during the age of Enlightenment. 


Soon, this new worldview dominated all aspects of life.


Towards the end of the 19th century, however, the hurried pace of technological change and a growing amount of unambiguous hard data began to challenge these concepts in physics. For example, in classical physics, describing the world as either particles or waves, the data from experiments using X-rays showed that something wasn’t working\index{physics!classical}\cite{antliff-leighten-cubism-culture}.


\subsection{Picasso and Braque}
Pablo Picasso\index{Picasso, Pablo} (1881-1985) and Georges Braque\index{Braque, Georges} (1882-1963) were also dissatisfied with art’s capacity to portray the emergent reality that they were seeing. So they created Cubism\index{Cubism}\footnote{Cubism was penned by the conventional (and well-established) French art critic Louis Vauxcelles\index{Vauxcelles, Louis}. Attending a 1907 exhibition in Paris, he labeled Picasso and Braque’s initial exhibit a “bunch of little cubes”. Incidentally, Vauxcelles had earlier attended a 1905 exhibit of painters wildly experimenting with splashy colors (directly influencing Cubism) and derisively quipped that the paintings look like “wild beasts”. Fauve, the French name for a wild beast, stuck as an appellation for this important group of artists.}, which was seen by many as a revolution of the same size and importance as Giotto's revolution five centuries earlier\cite{shlain-art-and-physics}. 


Giotto expected that privileged viewers could only see what they could see from where they stood, and that an artist was expected to simply reproduce this privileged view of reality. If you wanted to see more, you needed to walk around the object, and portray this in a fixed sequence.


But this assumes that reality is exactly what you see, and no more than you see. Picasso and Braque realised that reality is far more than that. Reality is not just what you can see, it is not just what it seems, and it's filled with subjectivity, paradox, contradiction, and is inherently nebulous. So why represent this reality with a monolithic, linear, geometric perspective?


Cubism\index{Cubism} sliced and diced the representation of an object, and presented simultaneously all sides from all perspectives. No single perspective is privileged in Cubism. Picasso and Braque created a new road where no one had ever travelled, a new road that was far better at getting to where they wanted to get to: understanding the rapidly changing reality they were experiencing. 


By ending the privileged one-eye point of view, Cubism paved the way for Surrealism and abstract art. And most importantly, they have given you a set of pluralistic, non-linear tools to understand yourself, your organisation, and your economy.


You can only succeed in the emergent reality you are experiencing by using multiple lenses. Throughout this book you'll read more about how a lens is a tool that you use to see more clearly something relevant by hiding the irrelevant background, and how to use multiple ones. New lenses have long driven great shifts in civilisation. For example, the invention of eyeglasses doubled the skilled craft workforce and their productivity, and led directly to the invention of the microscope and telescope\cite{landes-wealth-and-poverty}.


For example, governments, to neutralise any perceived threat to their stability and security, start with raw data: chatter, tip-offs, sightings, wiretaps, etc. But once the data is collected, the real work of sorting it and making sense begins\cite{townsend-desperate}. If only one lens was used for this, a lot would be missed. Multiple lenses\index{lens!multiple} are required, across all scales from global down to the smallest local scale, spanning police, government, psychology, economic, sociological, historical, and many more, in order to see clearly; and each lens may cause help to be seen as a threat.


Your choice of lens\index{lens} determines the reality you experience, as does your choice to use just one lens or multiple lenses. Your choice of lenses determines whether you can use an emergent strategy or not. You need multiple lenses, giving you multiple perspectives, none of which is privileged, to build a strategy fit for today's emergent reality.


The same emergent reality that led Picasso\index{Picasso, Pablo} and Braque \index{Braque, Georges} to use new and multiple lenses, even though they were contradictory, led Einstein\index{Einstein, Albert} to do the same in physics.


\subsection{Einstein}
\index{Einstein, Albert|(}
Einstein wrestled with a number of inconsistencies in Newtonian physics\index{physics!Newtonian}, which could not explain new data.


Despite the stellar achievements of Newtonian physics in accurately describing everything from the movement of a grain of sand through the movement of apples to the movement of planets and stars, it was getting harder to make accurate calculations of anything moving very fast or anything very small. These problems led to quantum physics and relativity, which accurately describe how the world works when things are extremely small and extremely big.


Newton\index{Newton, Isaac} assumed that mass, energy, space, and time were separate and distinct; and that each could be measured without influencing the others\cite{newton-principia}. Objects had mass, and moved against a neutral and irrelevant backdrop of space and time. An object's motion through inert space, and the time it took to move, could be predicted and depicted to any precision. Two viewers, anywhere in the universe, would see exactly the same events and sequence of events.


However, the closer physicists looked at the world, the more clear it became that it wasn’t behaving this way. Einstein saw that if two of you observe the same events from widely different places, and / or you are moving at different speeds, you may well end up seeing a different sequence of events; and possibly even different events. It was clear that events were relative and subjective to the observer, rather than objective. None of your views is absolutely correct, none of them privileged. 


Einstein's general theory of relativity\index{general relativity} was even more profound. Although Newton could explain how objects were attracted to each other, he could not explain why they were attracted to each other. He could not explain gravity.


Newton was troubled by his inability. Einstein \index{Einstein, Albert} was equally troubled, but fortunate to be living in a time where solid experimental data was emerging that he then used to develop his theory of general relativity. General relativity describes how massive objects (think of the Earth) distort the surrounding space; and because they distort the surrounding space\index{space}, they change how objects move.


Think of putting a heavy cannonball onto the middle of a trampoline. The cannonball distorts what had been a flat surface, changing the way a light marble will then move if you put it onto the trampoline. It will automatically roll towards the heavy cannonball because the cannonball has distorted space. This is how gravity\index{gravity} is created, where a large object such as a planet or even a black hole distorts space by its very presence. It creates the reality that it and any other object near it experiences by changing space.


Einstein realised that gravity is unlike the three other fundamental forces (the electromagnetic, weak, and strong forces); it is, in fact, not a force. Rather, it is geometry\index{geometry}. At large scales, matter shapes spacetime (creates the geometry of spacetime), and then spacetime\index{spacetime} (geometry) tells matter how to move.


Space\index{space} is even more present and active at a subatomic scale. What Newton\index{Newton, Isaac} had thought was a neutral vacuum\index{vacuum}, inert and playing no role in the movement of small particles, turns out in quantum mechanics to have such a powerful effect, e.g. it’s better to think of the vacuum\index{vacuum} as the primary source of an electron's properties, rather than the electron itself having these properties in isolation. 


Throughout the book, especially in Chapter~\ref{chapter:who-am-i-one}, we will go deeper into how your meaning\hyp{}making stories\index{meaning-making stories}, and those you are embedded in, are an active, not a neutral background; colouring, shaping, or even creating the reality you experience.


You may find that Einstein's theories sound counterintuitive, or not even understandable. You, as everyone else does, experience the three spatial dimensions and time as very different. But to really understand how our world actually works, we need to abandon this simple view, and shift to thinking in terms of four dimensions; time and space are one.


When you use the lens of four dimensional spacetime, you can accurately and reliably predict what will happen, and you find that outcomes which seemed wrong using Newton's physics are perfectly normal and understandable. You can only get here if you give up on using Newton's single, paradox-free lens\index{lens}, and instead adopt Einstein's multiple, paradoxical lens\index{lens}.
\index{Einstein, Albert|)}


\subsection{Picasso and Einstein: a complementary pair}
\label{section:reality-actuality}
General relativity\index{general relativity} and Cubism\index{Cubism} both changed the game in the same way. Using them, you can now understand reality by making sense of the apparent contradiction between what you see and experience (reality) and what actually is (actuality). You now have ways of understanding the intrinsic ambiguity inherent in actuality.\index{actuality} General relativity and Cubism recognise that actuality is inherently what it is and all that it is, so any lens will distort what you experience, and give you a false and misleading picture. Yet you cannot not use your lenses.


In a nutshell, you need an appropriate combination of accurate and up-to-date lenses in order to match the reality you experience as closely as possible to actuality\index{actuality}.


Newton\index{Newton, Isaac} simply assumed that an independent observer could measure objectively, without any distortion or influencing what was observed. Einstein\index{Einstein, Albert} realised that this fundamental assumption was not valid, that this assumption was why classical theory gave certain useless, and sometimes even harmful, predictions. 


If you say that an electron is inside the box, and someone else claims that the electron is outside the box, both of you can be right. 


Two people with mutually exclusive realities can both be right. These paradoxes are ubiquitous throughout life. 


You have far more chance of success if you give up on only seeing opposites as mutually exclusive. You have undoubtedly learnt early in your life that something is either true or false, right or wrong. In art, quantum physics, and in life itself, sometimes two perspectives are mutually exclusive pairs, and at other times they are complementary pairs. Both are true, rather than either / or. 


I believe that, to deal with the emergent reality and crises of today, complementarity must be the starting point across all the disciplines needed. Complementary pairs\index{complementary pairs} are used throughout this book, and described in Chapter~\ref{chapter:emergence-einstein-picasso}.


Quantum physics\index{physics!quantum} is weird, but this is how our world works. Richard Feynman\index{Feynman, Richard}, widely regarded as Einstein's equal in his generation, said that if you do not find quantum physics confusing, then you do not understand it. Humans cannot understand it because we are one to two metres tall, and can only understand the world through the basic common sense and observation possible as a being that size. You cannot directly experience quantum reality. 


But your mobile phone, and many of the other conveniences of modern life, all rely on relativity and quantum physics.


Nobel laureate Walter Heisenberg\index{Heisenberg, Walter} developed his uncertainty principle\index{uncertainty!principle} in 1927. The uncertainty principle reflects an inherent unknowability in nature that has nothing to do with our human ability to know. For example, an electron’s momentum and its position are a complementary pair. The more you know about an electron's momentum, the less you can know about its position, and vice versa. This finally laid to rest the myth that a neutral observer can just look at an electron, and see all actuality without creating the reality that the observer sees. Uncertainty and subjectivity\index{subjectivity} exist at the most fundamental level of nature.


Some people interpret this as saying that everything that you experience, you create; and that you have total control over what you experience. This is absolutely not what quantum mechanics or the uncertainty principle is saying. Rather, it says that when you observe an electron, your action of observing creates the final reality that you see, out of a limited range of possible outcomes. 


A similar uncertainty principle is at work in art. For example, Cubism \index{Cubism}criticises much of the earlier art: the more precisely the painter has painted depth and perception from a single privileged perspective, the less freedom you as viewer have to engage with the painting and experience what you might experience looking at the actual landscape. Picasso\index{Picasso, Pablo} deeply understood that the reality you experience of a work of art is co-created by the artist capturing a limited range of possibilities, and by you, the viewer, looking at the art and shaping the final reality you experience.


If you apply these uncertainty principles to the emergent world you are living in today, you can probably already see that the more detail you have in defining and executing your strategy the less you can recognise emergent possibilities for success. If you are about to start up a business, this points at (like Heisenberg’s uncertainty principle)\index{Heisenberg, Walter} getting the right balance between putting time into detailed strategy; and time into improvising, testing minimum viable prototypes, and pivoting when new learnings come your way.


In the emergent reality you live in, at best on the border between complex and chaos, there are fundamental limits on what you can know. You seldom have any more than just enough time to react and alter course and strategy. Reacting quickly to imperfect, incomplete data that you've got because you tried something counts far more than waiting until you have the perfect analysis of complete data before acting.


Picasso and Einstein\index{Einstein, Albert} took huge risks abandoning the well-established perspectives used by their contemporaries.


They took these risks because they knew that if they didn’t they would not be able to live with themselves. They knew that the reality they were experiencing required them to take them. In that sense, they may not really have seen this as risk-taking. Later in this book you will develop more clarity on how your meaning\hyp{}making stories, your nature, and your capacity to use different thought forms, creates your reality, including your perception of risk and reward.


To thrive in the emergent world you live in, you need to see these meaning\hyp{}making stories clearly, so that you can reframe the reality that they shape to one that is more helpful to you.


Once you can transform your meaning\hyp{}making stories\index{meaning-making stories}, you can become antifragile\index{antifragile}. You can grow your capacity to thrive, from fragile, through resilient and robust, to antifragile. Resilient simply means the capacity to survive unchanged in the face of attack, while antifragile is the capacity to transform in the face of attack, becoming more able to thrive because of the attack. (See \emph{Antifragility}\cite{taleb-antifragile} for more.)


You will see in this book how to reapply the essence of the transformation a century ago in art and physics to build new types of organisations that are also antifragile: developmental, self-organising, and incorporated as FairShares Commons\index{FairShares Commons}, or free companies\index{company!free}. 


You will also learn how you can use these to transform your identity, self-devel\-op\-ment, and ability to develop your emergent capacity. This will give you antifragility. I believe all this is essential for you, your organisation, and our economy to thrive.


\section{Developing your adaptive capacity}
\label{section:intro-your-ad-cap}
\index{adaptive capacity|(}


Part~\ref{part:you} on page~\pageref{part:you}, Chapters~\ref{chapter:who-am-i-base} through to~\ref{chapter:who-am-i-one}, is all about how you can develop your adaptive capacity. 


You all have the potential to do this, and it is imperative if we are all going to rise to the adaptive challenges facing us. Developing your adaptive capacity is a real challenge, though, because it means accepting the inherent ambiguity, uncertainty and unpredictability in your life.


The first step is to know who you are and where you've been. Just as Einstein\index{Einstein, Albert} taught us that the properties of a particle (say an electron) are not independently owned by the particle itself, but instead depend on the entire context of the particle, including the path it has travelled, the same is true for you. Who you uniquely are depends on the path you've travelled and the meaning that you have created and attached to that path through your interactions with others.


The Ubuntu\index{Ubuntu} philosophy of Southern Africa captures the inherent interconnectedness of everyone and everything very well. “I am because we are”, or “I am because you are”, are the two common translations into English of the essence of Ubuntu. This has much in common with Picasso's\index{Picasso, Pablo} art, relativity,\index{relativity} and quantum physics.


Ubuntu recognises that none of us can become human, let alone our full selves, without the influence of everyone else around us. It reflects the full interconnectedness of everyone and everything. The meaning\hyp{}making stories that define who you are today have been shaped by everybody that you and they have interacted with; and onwards down the links of interaction. This interconnectedness is at the very heart of what it is to be human.


Path dependency\index{path dependency} and interconnectedness\index{interconnectedness} are central to  how the world works. This new perspective a century ago in physics and art now needs to be repeated in business and economics, and relearnt at an individual level by many in the West. 


So who you are, and the decisions you take, depends on your entire history, your beliefs about your future, your context and the meaning that you have made of your context. The internal reality that you've constructed, and the maps you use to navigate through life. 


Who you are is emergent, meaning that your identity and your path through life is created by each step you take. In Part~\ref{part:you} you can learn how to tilt the emergent probabilities in your favour, and Part~\ref{part:organisations} covers how to do the same for your business.


Think back to when you were 10 years old, and who you wanted to become when you grew up. For some of you reading this, you will likely say \emph{Yikes, I'm glad that that didn't happen.} You cannot control who you are, nor who you will become, from who you are today. 


Some of you will find this lack of control a bit scary. But it has a powerful upside. Who you are, and the reality that you will experience in the future, is fluid. Even just a few minutes in the future, who you are and the reality you will experience can change within the range of options open to you.


You are emergent, and your future is emergent. Both are created by the meaning\hyp{}making stories you are using to shape your reality out of the countless events and possibilities in actuality, and that you have no control over.


There is a fundamental limit to how well you can plan and steer. If you try to predict who you should become and then steer yourself there in a predictive, controlled way, you are far more likely to lose opportunities that you might later prefer. 


Some of you reading this may already have experienced your midlife crisis, or the quarter-life crisis that some experience in their mid-20s to mid-30s. Any mid-something crisis has, at its root, the fact that the lenses and meaning\hyp{}making stories that have worked for you before are failing you now. You have two choices: either you embrace the lesson that the world is giving you, and change the lenses and meaning\hyp{}making stories\index{meaning-making stories} that you use as maps to navigate your life, or you try to go back to your past. And, as we’ve said before, trying to go back to the past, trying to shrink your world to one that fits the smaller maps that you built when younger and less wise, usually fails.


I find Otto Laske's \index{Laske, Otto}map of maps, the Constructive Developmental Framework\index{Constructive Developmental Framework} (CDF)\cite{laske-vol1,laske-vol2} especially useful. The CDF is the central foundation of our book. In Part~\ref{part:you} you will learn how to use this map to connect what is and what can be, who you are and who you can become, and to develop your adaptive capacity.


Central to this map\index{maps} is the developmental sequence of self-identity\index{self-identity}, i.e., the categories of meaning\hyp{}making stories that adults progress through as they age, along with the forms of thought that you develop over time. The self-identity development is derived from Kegan’s\index{Kegan, Robert} work on stages of adult development, and the thought forms from the Frankfurt School of philosophy.


To transform yourself, your organisation, and your economy, the better you understand where you are in this CDF map of maps, the better you can adapt yourself to who your reality is calling for you to become. Keep in mind, this is not a deterministic map. It is a map of probabilities and emergence in the face of adaptive challenges. It does not tell you exactly who you are, nor can it predict exactly who you will become. Far more, as in quantum physics, it lays out categories and descriptions of territory.


I cannot emphasise enough that there is nothing in the CDF map of maps saying that you are better or worse than somebody who is somewhere else in this map of maps. Any given stage is not better or worse than any other stage. As you will uncover later in the book, any judgement of better or worse is so dependent on context and perspective that any attempt to judge anything or anyone as inherently better or worse without stating clearly the context and your unique internal frame of reference is fundamentally flawed.


As well as Kegan's\index{Kegan, Robert} stages of adult development, the CDF framework enables you to recognise how we take in information from the outside world and put it together in your thoughts to make sense of what is happening. The more subtly you can work with multiple forms of thought, the less likely you are to bias or distort the picture you build up as your internal reality from the puzzle pieces you take in. No matter how fluid you are using all the different forms of thought, the internal reality that you construct is always an incomplete and distorted representation of what actually is.


The bigger your adaptive capacity\index{adaptive capacity} in different forms of thought (also called sense making), the more able you are to go beyond linear logic forms of thought, and deeper into the 28 different patterns often called meta-thinking, transformational thinking, or dialectic thinking. You can read about this in detail in Chapter~\ref{chapter:who-am-i-sense}.


The third element of the CDF map of maps (Chapter~\ref{chapter:who-am-i-nature}) helps you work with your innate and unchangeable nature in a more subtle way. The aspects of your nature, like your psychological traits and needs, that you either cannot change, or at best only very slowly. Neuroscience\index{neuroscience} suggests that many of these are hardwired into us, either in our DNA,\index{DNA} or at a very early stage of our lives. Recognising these and learning how to work with them in a subtle way, harnessing the value that they bring to help you succeed when your emergent reality throws surprises, should be your focus. 


For example, some of you reading this might experience periods of depression or anxiety, and you may well have wished that you could edit out everything in your genes that predisposes you to them. But by doing so you might at the same time edit out traits and strengths that you need now or will need soon. Your depression might be one side of a superb strength that you have, of feeling intuitively the gap between what could be and what is. When you turn the strength outwards, you invent a breakthrough product that solves problems people didn't even know they had, and a start-up that perhaps transforms the climate emergency. No one who is lacking this strength would ever see that product. When you turn the exact same strength inwards, and see the gap between who you are now and could be, then your feeling of the gap turns into depression.


\begin{longstoryblock}
While I (Jack) have experienced much joy in my life, I’ve also endured severe and debilitating depression\index{depression}. Without getting into the details of why, or how I have dealt with it, depression is an integral part of who I am. And, while I wouldn’t wish this on anyone, it has given me a creative edge, a humility, empathy, sense of humor\textemdash yes!\textemdash and an understanding that I couldn’t have obtained otherwise. It has given me the creative tools to write this book (along with three other works of non-fiction) and the lively imagination to write two novels. 


Of course, I do not offer my past as a formula for creative success, only the well-worked advice that rather than delete the ostensibly harmful/negative aspects of your personality (like depression), re-edit them into positive meaning\hyp{}making stories (like creative edge) to enable you to be who you can be.   
\end{longstoryblock}


Those of you who have read about Picasso\index{Picasso, Pablo} and Einstein\index{Einstein, Albert} as people will know that they were not universally acclaimed as nice, comfortable people to spend time with. Words like arrogant, inconsiderate, disparaging, and many more have been used to describe some of the interpersonal behaviours of one or both of them. Both had a restlessness, a drive to ask questions and constantly explore new areas, and to know everything. This enabled them to create their radical breakthroughs in their fields, yet made the relationship others had with them fraught with issues. 


Their single-minded focus, a dogged determination, and the belief that they were alive with a specific mission and purpose, that failing to fulfil that mission and purpose was the most relevant measure that their lives would be measured against, is the characteristic that enabled them to ignore the opinions of others, to keep going for decades, despite setback after setback. And today we judge that as their strength. This is the same characteristic that, in their interpersonal context, was and is judged as arrogant.


I believe this is another example of the complementarity of strengths and weaknesses. Of how the same root characteristic shows up in one situation as a strength, and in another situation as a weakness. As Peter Drucker \index{Drucker, Peter}says\cite{drucker-eff-exec}, when you hire someone with a superb strength in one area, you will always get with it an equally big weakness, so the organisation’s purpose is to make the strength productive and the weakness irrelevant through shaping the context.


As you read this book, you will likely also see the value of simply accepting some of your weaknesses in order to make their complementary strengths even stronger, and partnering with somebody very different to you, so that together your different strengths are made productive and none of your different weaknesses gets in the way. 


Tying it all together, Part~\ref{part:you} centres on how you construct your self-identity and the reality you experience through your meaning\hyp{}making stories, which act both as lenses to see through and maps to navigate by. You, as an individual, will read about cutting-edge approaches to growing and transforming your own meaning\hyp{}making capacity, so that you can rise to the challenges you are facing, and build the new kinds of organisations we need to construct a regenerative economy capable of addressing our global challenges\index{adaptive capacity|)}.


\section{Developing Adaptive Organisations}
\index{Adaptive Organisation|(}
Any organisation lacking the full adaptive capacity we describe in this book will fall short of its potential. Developing Adaptive Organisations, and business ecosystems built of Adaptive Organisations, is the focus of Part~\ref{part:organisations}. 


An Adaptive Organisation has the capacity to: 


\begin{itemize}
\item see clearly what is happening;
\item see clearly the difference between the relevant and irrelevant;
\item attach appropriate meaning to what is happening; 
\item act appropriately. 
\end{itemize}


An Adaptive Organisation must be antifragile\index{antifragile}, i.e., not only withstanding damaging events, but transforming through them to become even more capable\cite{taleb-antifragile}. 


This can mean that characteristics your meaning\hyp{}making stories\index{meaning-making stories} have always seen as a weakness to remove can end up a strength that you require. An Adaptive Organisation has the capacity to thrive, because it uses a multiplicity of perspectives, in situations where you cannot decide until long afterwards whether a given characteristic is a strength or a weakness, or even both.


An Adaptive Organisation can evaluate how much complexity and adaptive capacity is required by each role within, while evaluating the adaptive capacity of each individual, so that the organisation’s adaptive demands and supplies are matched. We need this to build a regenerative economy capable of addressing our global crises.


I will take you through the specific characteristics of Adaptive Organisations in sufficient detail in Part~\ref{part:organisations}, Chapters~\ref{chapter:who-is-your-organisation-base} to~\ref{chapter:growing-regenerative-organisations}, for you to begin building your own. 


Part~\ref{part:organisations} begins by looking at the organisation as a living being with its own meaning\hyp{}making stories, composed of individuals with their own meaning\hyp{}making stories interacting with each other. 


Approaches like Sociocracy\index{Sociocracy}, Holacracy\index{Holacracy}, and requisite organisation design can be ideal complements. Add developmental approaches like the Evolutesix Adaptive Way, and the organisation can harness conflict\index{conflict} to grow each individual, team, and the organisation as a whole. 


However, this is not enough to become antifragile and adaptive. The new, crucial component is the FairShares Commons\index{FairShares Commons} legal structure. This structure enables all stakeholders, from each individual consumer through to the planet's natural environment, to engage in the company reaching its full regenerative potential across all capitals. This incorporation is vital if multiple businesses that together form a circular economy are to trust each other over the decades needed for a large circular economy to function, and to go beyond a circular economy into a fully regenerative economy across all capitals.


I will also discuss successful examples of Adaptive Organisations, especially those already using the full FairShares Commons or parts of that incorporation, such as Evolutesix,\index{Evolutesix} founded by one of the authors (Graham), which I believe is a successful prototype\index{Adaptive Organisation|)}.


\section{Society and the economy}
This is the focus of Part~\ref{part:economy} on Page~\pageref{part:economy}, and culminates in Chapter~\ref{chapter:economy-of-the-free}, where we paint what we believe the general essence of a viable regenerative economy ought to look like. An economy we call The Economy of the Free\index{Economy of the Free}, quite different to what we have today, which is a free economy. 


Can you define an economy\index{economy}? You spend your life in your economy; you have an intuitive idea of what it is and how it works.


I define an economy as a tool invented by society to do the job of provisioning for us all. We need an economy in order to buy needed goods and services; and to sell, share, and even offer for free, what we can produce. 


Because the economy has been created by society, it can never be independent, separate from society. And society, in turn, is dependent on our natural environment's capacity to support human life.  


To understand, know, and change our economy we must choose multiple lenses and maps, i.e., meaning\hyp{}making stories. You'll read more about the meaning\hyp{}making story\index{meaning-making stories} stages in Part~\ref{part:you}, and will recognise that if you grow to a later stage, and a higher meta-thinking fluidity, you can see more clearly emergent complexities and ambiguities than you once did, when you could only use the lenses available at lower stages and fluidity of thought. 


Unfortunately, too many influential experts, including economists\index{economists} and managers, have a meaning\hyp{}making centred prior to Stage~4. Many are at a stage where self-identity and expertise are synonymous, so any challenge to their discipline is interpreted as a challenge to self-identity, making it difficult or even impossible for them to deploy multiple lenses \index{lens!multiple}and maps\index{maps}. This is at the heart of much of the friction between different disciplines, and between society and the economists. 


Steve Keen\cite{keen-debunking}\index{Keen, Steve} wrote that if you could invite a late 19th century physicist, biologist, mathematician, chemist, and an economist to the 21st century, each would be bewildered, not recognising their discipline’s progress\textemdash save the economists, who would pick up right where they left off. 


What you will see, in the language of this book, is that Keen is pointing at the relevance  to the reality that they shape of developmental stages of individuals and professional groups. Many professional groups have too many people who still derive their sense of self-identity and self-esteem from their expertise and their belonging to a community of similar experts. This is a barrier to addressing our global challenges when the expertise is well past its sell-by date. In this book, for obvious reasons, I will write about economics and business professionals, but it is equally applicable to all professions. 


Many of you are reading this book precisely because you know intuitively that something is as fundamentally wrong with our economy\index{economy} and economics as once was wrong with flat-earth navigation theory. The fact that we have a climate emergency, and all the other injustices is, I believe, overwhelmingly conclusive evidence. 


Just as physicists and artists realised a century ago that classical art and physics could never do the job, but rather a radical transcend-and-include revolution was needed, so too is it time to make a radical shift to what comes next, after neoclassical economics and current business practices.


In economics and business, practitioners continue to use single lenses\index{lens}, often anchored in the classical reality of the 19th century. Often without hard data to evidence that their lens is valid today, just assuming that since it once worked in one context, it will always work in today's contexts. 


In Chapter~\ref{chapter:economy-of-the-free}, I paint a picture of the Economy of the Free\index{Economy of the Free}, an economy that I believe will work for you, and all life on our planet. Then you can read how we will build one with the building blocks of the Adaptive Organisation covered in Part~\ref{part:organisations}, including the Adaptive Way covered in Part~\ref{part:you}. 
 
Using the philosophy of Ubuntu\index{Ubuntu} as one of a number of different lenses for economics and business, practitioners can help us find our way to get better at describing our economy. 


Because it is an element of human life, business and our economy is emergent and inherently nebulous. It emerges from the highly interconnected, path-dependent, intangible essence of human life. Using this lens\index{lens} to look at our economy you see the rationality in those decisions that neoclassical economics considers irrational. One reason why much in economics\index{economics} is now failing to do the job we need lies in its failing to build off the very interconnected, path-dependent, meaning\hyp{}making, story-dependent essence of what a human being is. 


I believe that a crucial part of the regenerative economy we need is the inclusion of all stakeholders directly in the governance of, and sharing the wealth generated by, our business. This means that companies themselves can no longer be deemed property of the investor, but, like all human legal persons, are fully free. Which makes the crucial difference between the current free economy, and the Economy of the Free.


Inspired by the Ubuntu\index{Ubuntu} “I am because you are”, and everything else we cover in this book, we close Part~\ref{part:economy} with a concept that we hope will provoke the development of an approach to \emph{what we can say about economies} that integrates everything valid across the full breadth of today's complementary approaches. Taking general relativity\index{general relativity} as a metaphor, I propose a geometric approach to economics, which I suggest could be named a general theory of economies\index{general theory of economies}.


By the end of the book, you will have read everything that should help you to make a difference to the crises we are facing. Everything that is useful for you to develop the adaptive capacity in yourself, your organisation, and your economy to rise to the adaptive challenges at your doorstep. 


Keep in mind what we wrote in the preface: read the parts in whichever order best fits you and your needs.


\section{Hope}
\label{section:hope}\index{hope|(}


A good example of working with emergence, that delivered far more success than I (Graham) or anyone else had any reason to hope for, was the process of transforming South Africa into an all-stakeholder democracy. Yes, much still needs to be done, and there is much to be criticised; South Africa\index{South Africa} has fallen short of the vision. Yet looking at the emergent context in the late 1980s and 1990s, the process succeeded spectacularly compared to all probable outcomes. 


The Mont Fleur\index{South Africa!Mont Fleur} meetings\cite{kahane-solving}, near Cape Town\index{South Africa!Cape Town}, were critical in developing the adaptive capacity (multiple lenses, accurate maps, dialogue, pluralism, and Stage~4 and~5 meaning\hyp{}making in individuals and organisations) needed to develop a viable route to a post-apartheid South Africa. Key players from all groups, each initially following their own strategy to build a viable future, came together. Through emergent dialogue they fully grasped the consequences of each other’s meaning\hyp{}making stories, and the reality these stories were creating. Each group clearly understood that each strategy offered elements:


\begin{itemize}
\item needed in the desired future South Africa; and
\item potentially sabotaging the journey to this desired new South Africa.
\end{itemize}


The dialogue\index{dialogue} shifted from ‘us vs. them’ to integrating the unique value of each, recognising the inherent oneness of all. Everyone actively listened with empathy, learning to use each other's lenses and meaning\hyp{}making stories\index{meaning-making stories}. the dialogue process then spread across the country, engaging people from every walk of life with much-needed empathy, based on multiple maps\index{maps} and lenses\index{lens}. 


Compared to the lack of such dialogue in many of today’s leaders, the maturity of the South African leaders then is obvious. Today you often see people splitting into opposing camps, unable (or unwilling) to understand the other, let alone dialogue, or even appreciate what is valuable in each other's uniqueness. If South Africa\index{South Africa} had insisted then on simplistic right versus wrong according to whichever lens the most powerful person used, a viable transition would have never occurred. The only viable transition is one that works for almost everyone, and that hears every perspective.


The South African experience suggests five key ingredients for any successful adaptive strategy fit for our emergent reality filled with adaptive challenges.


\begin{itemize}
\item Recognise that the inherent, irreducible contradictions in your reality are real, not an artifact of right or wrong perspectives; and so all lenses must be used simultaneously to see all contradictions.
\item Realise the inherent impossibility of knowing all qualitative and quantitative aspects in advance.
\item Develop emergent capacities in individuals and organisations, along with the requisite structures and processes; only then can we have emergent dialogue.
\item Accept that your meaning\hyp{}making stories shape or even create the reality that you experience. Learn how to change your stories to change the reality you experience.
\item It's up to you to act, especially to build something new that works better than the old.
\end{itemize}


The emergent Mont Fleur\index{South Africa!Mont Fleur} dialogue succeeded in South Africa, 
because everyone realised that pure competition between the parties would lead to a South Africa none of them wanted; but the right mix of collaboration and competition could. Section~\ref{section:non-ergodic} reproduces this in the context of business to show how our economy can have at least 1,000 times more power to reverse our headlong rush into a climate catastrophe\index{climate!catastrophe}, maybe even 1,000,000 times more, if we incorporate in a multi-stakeholder collaborative way, such as the FairShares Commons\index{FairShares Commons} described in this book.


This is why I (Jack and Graham) wrote this book.\index{hope|)}  There are still emergent ways of turning planet Earth around, despite the growing fatalistic despair. We just cannot see them from where we are now, using the lenses we currently have.