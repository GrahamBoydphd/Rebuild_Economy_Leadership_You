\noindent A great book for everyone who thinks that this cannot be the end of the story. Economy, organizational management and personal self-development brought to a whole new level!


{\raggedleft \textemdash \textbf{Magdalena Sch{\"a}fer, Senior HR Manager, Organisational Psychologist}}
\vspace{1.5ex}


\noindent \emph{Rebuild} sets new standards and takes the reader into a whole new path for a better future and a paradigm shift on how the current and future investors and leaders must conduct business and behave towards their followers, collaborators and the wider stakeholder community. This book revolutionises the reader's views, assumptions and approaches for a more responsible, equitable and sustainable life as a whole. A must read, if you are sincerely interested to be the change you want to see!


{\raggedleft \textemdash \textbf{Antonio Potenza, CEO Proodos Impact Capital}}
\vspace{1.5ex}


\noindent \emph{Rebuild} delivers exactly what it says on the cover: a toolkit for you to build a better world. It contains the most useful business and personal toolkit I’ve seen, and has radically transformed my worldview. Organisations, jobs and practically everything\textemdash we're trying to reinvent them, but are we succeeding? This book makes reinvention work; it's written as part guidebook, part theoretical overview with scientific underpinnings, and a completely relatable narrative with fascinating anecdotes. It is for anyone who wants to build solutions to our global challenges. 
 
 {\raggedleft \textemdash \textbf{Marie-Nicole Schuster. Leadership development consultant}}
\vspace{1.5ex}




\noindent The \emph{Adaptive Way} chapter in \emph{Rebuild} made me realise that seeking validation from others to build my business (Solve Earth) is a self-defeating trap, because no single person has the capability to understand fully what I, or any other person, is capable of. I can continue on my journey into the unknown with advice and guidance from others, but ultimately, I must rely on my own decision making and self-validation. This is an incredibly liberating and empowering insight that gives me the courage to take on the biggest challenges in sustainability and building a better future society.


 {\raggedleft \textemdash \textbf{Dr Gareth J Thompson, Founder, Solve Earth Limited}}
\vspace{1.5ex}




\noindent We know now that organisations who embody natural and evolutionary design principles, not only thrive but build critical resilience and sustainability. 
This architecture also needs to be re-evolved and re-generated to meet the needs of today's global market dynamics and for our collective humanity so you hold in your hands a book that gives you a road map to a journey how this is best achieved and executed. 


{\raggedleft \textemdash \textbf{Robert Dellner, PHD
Author, Integral Impact Investments (i3): Building and navigating a full-spectrum systems approach to investing}}
\vspace{1.5ex}


\noindent \emph{Rebuild} presents an integrated vision\textemdash one that transcends polarities and makes the old divisions obsolete. This is one of many qualities shared by great visions. It is also a systemic vision that not only aims to transform our consciousness, but also the systems that forcefully perpetuate and enforce the old consciousness. It is a pragmatic vision based on available experiments, examples, and research, and doesn't require rebuilding society from scratch. And it is a radical vision, that doesn't focus on putting patches on the problem, but emerges from asking powerful ambitious questions like \emph{how can we build an economy that harnesses all of the human capacities, like cooperation, compassion, and self-interest, to provision for all while regenerating all our capitals: financial, social, and  natural?}


{\raggedleft \textemdash \textbf{Chen Zvi. Climate activist in Extinction Rebellion;
Researcher in systemic societal transformation}}
\vspace{1.5ex}




\noindent I have seen this book develop over many years, and participated in online events organised by Graham after the crowdfunding campaign went live to support its development. The result, now published as 'Rebuild', is a thoughtful contribution to individual, organisational and societal development. It brings the fields of philosophy and practice together productively in the service of building a commons economy, arguing that we need stewards to manage assets for the benefit of current and future generations. 
Firmly rooted in a pragmatic application of social constructionist philosophy, Rebuild is a toolkit for individual and organisational development. It draws extensively on ideas developed by Otto Laske, Elinor Ostrom and members of the FairShares Association as well as the educational artefacts and papers published by the FairShares Institute at Sheffield Hallam University. At the heart of the argument is the case for FairShares Commons Companies that steward six capitals for future generations, and which enfranchise the founders, producers, workers, customers, users and investors who contribute and develop each form of capital. It is a bold vision, and a major step forward in realising an economy that supports sustainable development.


{\raggedleft \textemdash \textbf{Professor Rory Ridley-Duff,  
FairShares Institute for Co-operative Social Entrepreneurship, 
Sheffield Hallam University }}
\vspace{1.5ex}