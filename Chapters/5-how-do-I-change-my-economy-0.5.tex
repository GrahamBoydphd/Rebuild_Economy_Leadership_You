\chapter{The Economy of the Free}
\addcontentsline{toc}{chapterdescription}{To address our global challenges, we need an Economy of the Free, rather than our current Free Economy. Here’s how we should construct it, and how it leads to regenerative capitalism built on regenerative businesses in regenerative ecosystems that are inherently circular. The stock market, boom and bust, trade, finance, sustainability, work and your life will all look much better in an Economy of the Free.} 
%\addcontentsline{toc}{chapterdescription}{\pagebreak}
\label{chapter:economy-of-the-free}


\begin{chapterquotation}
It is good to have an end to journey toward, but it is the journey that matters in the end.\\
\raggedleft\textemdash Ursula K Le Guin\index{Le Guin, Ursula K.}


\centering
Let your vision be world-embracing rather than confined to your own self.\\
\raggedleft\textemdash Baha’u’llah\index{Baha’u’llah}
\end{chapterquotation}




You bring a new reality into your world by first writing a new story. You begin writing a new story by imagining what might be. Leaping into an unknown potential, painting broad swathes of imaginary colour. Then growing your reality into this aspiration. 


This chapter tells a story of what can be, and in part already exists. A story to replace the stories today’s business ‘truths’ concretised. This new story already partly exists, in companies like Bridgewater\cite{dalio-principles}, one of the world's largest hedge funds with USD160 billion assets under management in 2017, which has consistently used some of the elements of this new story. The founder and CEO, Ray Dalio, is quite clear that this is the reason for their phenomenal success.


Highly successful businesses like CCA, Zappos, Robert Bosch, Carl Zeiss, Mondragon\index{Mondragon}, John Lewis, Interface,  IKEA, Patagonia, Whole Foods\index{Whole Foods Markets}, Visa up to 2007, and many more, have proven over decades each of this story’s individual building blocks. There is no question that each building block has given them a competitive edge, especially in a crisis.


The important question is: how much more powerful do these building blocks become when you use them all together, and enough companies using all the building blocks form an ecosystem? Look at everything that Paul Polman\index{Polman, Paul} accomplished during his decade as CEO of Unilever. The huge strides that the company took during those 10 years are testimony to just what can be done with the largest multinationals. And yet, the gap between how far Unilever did change and both the demands of today's crises and Paul Polman's beliefs is still much too big.


How much more could he have done if even one prototype ecosystem of the Economy of the Free existed, providing hard compelling data of the benefits to each stakeholder within their set of needs and dominant stories?


As you read this chapter, use it to light up your imagination. Imagine what an Economy of the Free can bring to you. Write your story of your Economy of the Free\index{Economy of the Free}, and then do everything you can to grow your reality into that aspiration. I believe that together we can build the reality painted in this chapter. We just need to harness your uniquenesses, all of you reading this book, together.


You can also take encouragement from many countries; e.g., Germany, where individuals across the spectrum from politics, bureaucrats, business, and civil society are all taking initial steps to transform the entire foundations in the direction that I am pointing out in this book, for example via the Purpose Foundation's\index{Purpose Foundation} lobbying, consulting, and investment activities. 


If something doesn't make sense to you yet, you may need more fluidity in one or more of the forms of thought described in Chapter~\ref{chapter:who-am-i-sense}. Without sufficient fluidity in them, I (Jack and Graham) could not have broken through the limits of the classical lenses of business and economics\index{economics}, nor seen the necessity of multiple perspectives to make sense of what is happening, nor could we have begun the transformational journey of bringing multiple systems into a new balance. 


\section{Overcoming obstacles}
There are obstacles we need to overcome to build the Economy of the Free, a regenerative economy quite different to the free economy we have today (where precious little is free). Some key obstacles are listed below, and this book shows some actions you can take to overcome them.


Every system generates certain kinds of benefits for some or all. Any change in the system will either change those benefits, or at least trigger fears that they will change. And so the very powerful cognitive bias of loss aversion will fight back. 


Whilst you may not believe these obstacles can ever be overcome, I believe they can, because I have been here before. Recall the story of South Africa in Section~\ref{section:hope}, describing how the Mont Fleur\index{South Africa!Mont Fleur} process overcame obstacles I (Graham) had believed insurmountable as a child. It worked because it used everything described in the rest of this book.


By applying everything in this book to yourself and your work, especially if you are a leader of any nature, you will be able to take the first steps needed to overcome the obstacles and build an Economy of the Free.\index{Economy of the Free} And the best way to overcome any obstacle is to transform it into a supportive ally.


\begin{description}
\item[Power]\index{power} Those in positions of power benefit from today's economy, and some will be loath to lose their benefits. Benefits like power, control, money etc. They have the power to slow down transforming our economy to an Economy of the Free. In South Africa\index{South Africa} those in power (military, police, business and political leaders) were transformed into allies by creating a path to a new South Africa that was better for them than the counter-coup they could have staged to retain Apartheid\index{South Africa!apartheid}.


\item[Meaning\hyp{}making stories]\index{meaning-making stories} Making the Economy of the Free the new reality requires enough of us to use the appropriate meaning\hyp{}making stories. Historically transforming the meaning\hyp{}making stories of large numbers of individuals has happened over at least a generation. But we do not have the time left to wait for death's role here. Fortunately, we now know how to change meaning\hyp{}making stories, and describe how in this book, in Chapters~\ref{chapter:who-am-i-base} to~\ref{chapter:who-am-i-sense}. 


\item[Economists\index{economists}, lawyers, etc.] People in key positions of power  concretising meaning\hyp{}making stories into institutions are powerful allies in change, provided they are willing to, and supported in, changing their own meaning\hyp{}making stories. Because this changes their self-identity, it is akin to a little death. Who I am now, that I am likely proud of, needs to die so that the next me can live. We know how to do this, and this book gives the recipe: raise awareness of who you can be, and provide support to ease self-transformation using powerful transformation tools.


\item[Economics discipline]\index{economics} The authors of \emph{Econocracy}\index{The Econocracy} clearly describe the disconnection between the contents of most of the economics discipline (including all the institutions\index{institutions}), and what we need society and its economy to actually be for us to have a thriving future deep into future generations. This is not due to ignorance, because we have a good enough idea of where to go, and which steps to take first. Such as those described in this book, which is intended to catalyse action across the board.


\item[Money]\index{money} Our current mono-crop currency system (i.e., money as the only currency) is an obstacle, because money cannot represent all capitals in flow in the undistorted, bias-free way needed. Couple this with the number of powerful people and institutions in the financial space and this is a barrier. 


\item[Government] Local and national governments will no longer need to do much of what they currently do, in terms of regulation, taxation, etc., and so some incumbents will want to maintain the status quo. The good news is that government will become a direct stakeholder in the Economy of the Free\index{Economy of the Free}. So there is a more desirable future, and a path to it, for government.


\item[Company owners] Loss aversion is such a powerful driver of human behaviour that we even avoid large gains if we fear the small loss needed to achieve the gain. Losing the illusion of control to gain a wealthier life overall for yourself, future generations, and the rest of society will be a big shift for many owners of today's companies. Again, this book gives powerful tools to do so, and South Africa’s Mont Fleur\index{South Africa!Mont Fleur} process shows how it can work.


\item[System inertia] Overcoming the system inertia will take an Aikido-like feint. Following the advice of Buckminster Fuller\index{Fuller, Buckminister}, we need to build a new system rapidly, and once that is far enough, the old system will rapidly transfer resources into the new. Do not try to overcome the inertia of the old system; it will do that job by itself once a big enough prototype of the new one is visibly working better. This will also address all the above barriers. 


\item[Human capacity to imagine] Our capacity to imagine what can be, how to get there, and who we can become, is far bigger than most of us realise. We just need to use it fully by developing ourselves, as described in this book, and especially our capacity for uncertainty. The work of R3.0, mentioned below, and their templates for a transition to a regenerative economy\index{economy!regenerative}, are a superb resource of what works.


\item[Human capacity for uncertainty] \index{uncertainty} This is the crux. The journey to a new economy and society that works for all people and for a better world is filled with uncertainty. It is deeply Cynefin complex, or even chaotic (see the list on P\pageref{list:cynefin}). The bigger you develop your capacity for uncertainty, the better you can see and work with your cognitive biases, and the better you can put everything into practice to build the new system we need now.
\end{description}


Our current global system cannot continue unchanged; there is an inexorable current moving us towards a new system. What that will be depends on the path we take. 


Ohm's law\index{Ohm’s law} in physics states:


\begin{equation}
        I = \frac{V}{R}.
\end{equation}


The current flowing ($I$) equals the force pushing ($V$) divided by the resistance ($R$). Relevant, at least as a partial metaphor, to where we are now. Partial, because Ohm's law\index{Ohm’s law} is linear, but we are dealing with a highly complex nonlinear open system. In the worst case, pushing for change may even maintain our dysfunctional system, not change it.


The speed of change we need, i.e., the current, will continue to increase as we head deeper into all our crises. I see two main scenarios. 


Either we manage to reduce the resistance at least as fast as the current increases. Then the same, or less force is needed to drive the change. This way we are most likely to preserve all the wealth we need to build the new economy. (By wealth\index{wealth} I mean all different kinds of capital and access to those capitals\index{capitals} by all stakeholders\index{stakeholders}.)


Or the resistance stays as it is, perhaps even increases. Then the amount of force will continue to grow. The more force is needed to drive the current we need, the more we will destroy the wealth needed to build the new economy. 


\section{Your economy, your job to change it}
One story that creates your experience of the economy is that we lack the power to change it. And so, to change how the economy works, \emph{we} somehow need to convince \emph{them}, whoever they may be, to change themselves and the economy.


None of you needs to buy into that story. It's your economy, you can change it, it is your job to change it. Now let's dive into how you, as one individual, can take action to change it. And we must change it; even though many are enjoying a life that would have seemed like an unachievable, fantastic utopia to anyone alive a few centuries ago.


The first step is simply to recognise the extent to which the stories that we buy into take away or direct our power to act. Begin with the big shift Einstein\index{Einstein, Albert} and Picasso\index{Picasso, Pablo} brought in, rather there is no single privileged perspective and there are multiple perspectives that have value. Of course, they did not say that every perspective has equal value\textemdash both clearly recognised that some perspectives give you more value than others and that some opinions are simply nonsense.


Darwin\index{Darwin, Charles}  recognised that evolution\footnote{Another myth is that evolution is about competition between; survival of the strongest. Actually Darwin recognised that evolution is about the best fit to the conditions, which often is collaboration within groups and species. Evolution\index{evolution}  often works more on groups, less on individuals~\cite{wilson-this-view-of-life}. } proceeds incrementally. There are no revolutions driven from within. The only point where nature experiences the reality of a revolution is when a major crisis catapults an ecosystem into revolution, say a rapid change in temperature triggered by some natural calamity blanketing the atmosphere and either trapping heat in or preventing heat getting in. Such as massive volcanic eruptions, or planet-wide burning of carbon in engines.


You are a participant in our economy because you are reading this book. You may have bought the book, you may have funded the book in advance through our crowdfunding campaign, or you may have invested in the FairShares Commons\index{FairShares Commons}  publishing company that published this book. However you've done it, you're participating in the economy.


As you saw in the section on Ubuntu~\ref{section:ubuntu}, \index{Ubuntu} simply participating in the economy shapes the reality that you experience, and the different realities that others experience. Your stories, your capacity for transformational thinking, and your biases shape how you participate in the economy\index{economy}, and thereby your experience. Begin rewriting those stories today and you will begin experiencing a different reality tomorrow.


More than anything else today, we need to grow our wisdom to choose which stories shape our reality and which stories we refuse to allow to shape it. The wiser the choices you make today, the more likely your reality tomorrow will be a reality you thrive in, and a reality you feel hope in.


But your capacity for wisdom disappears quickly in the face of fear and many other strong emotions. We are hard-wired to do this, as you will see in the sections on biases~\ref{section:biases}. You will see how your unique combination of biases and stories shrinks the wisdom of your choices when you are in the grip of certain strong emotions, and amplifies the wisdom of your choices when in the grip of other emotions.


\subsection{Disruptive innovation for our economy}
Creating the Economy of the Free\index{Economy of the Free}  described in this chapter is a large scale example of disruptive innovation. Read the book of Clayton Christensen, \emph{The Innovator's DNA: Mastering the Five Skills of Disruptive Innovators}\cite{christensen-dna} to get more adept at applying the five skills of disruptive innovation to all aspects of your economy\index{DNA}. \index{Christensen, Clayton}


The essence of disruptive innovation\index{innovation}  begins with owning and rewriting the stories that shape the current reality. What could be more disruptive than taking the final logical step and freeing from ownership all legal persons, not just human legal persons like yourself? What could be more disruptive than reframing capitalism to include all capitals and demanding that our businesses regenerate all capitals\index{capitals}, not just financial capital? What could be more disruptive than reframing markets to facilitate the flow of all capitals from where they are abundant to where they are needed, fully valuing each capital for what it is, not for how much money the current exchange rate between capitals allocates to it, an exchange rate\index{exchange rate} that we humans have invented? 


Every economy\index{economy}  goes through periods of disruption driven by changes in context. And one of the biggest changes in context we need to respond to, at the extremes of the adaptive capacity we have in an Economy of the Free\index{Economy of the Free}, is climate emergency\index{climate!emergency}  and the 17 SDGs.\index{Sustainable Development Goals, UN 17} 


Another change in context that we will harness in an Economy of the Free to grow our adaptive capacity and respond to climate change is artificial intelligence. This is the next big great shift in technology, after the shifts that we have had over the past few centuries beginning with the Industrial Revolution.


At the moment, the dominant story around artificial intelligence\index{artificial intelligence}  is one of fear. In the Economy of the Free, the dominant story around artificial intelligence is one of hope. Because the foundations of artificial intelligence are all commons, rather than owned by a very small number of human and non-human legal persons. So the wealth that artificial intelligence generates across all capitals is shared by all, not just the few rent-seeking owners.
\subsection{Nested living ecosystems}
\label{section:living-ecosystems}


The nested living ecosystems\cite{wahl-regenerative} in an Economy of the Free all have boundaries, but the boundaries keep shifting and moving in different contexts for different people every time you look at them. You don't fall uniquely inside or outside an ecosystem. You may be deep in, strongly bridge between many, or be mostly outside; and this keeps changing.


You also can't pin down exactly what your role is and how it contributes to any of the nested living ecosystems\index{ecosystem}. The big difference between any living ecosystem and the same number of elements disconnected from each other but next to each other is intuitively clear to you. You appreciate and feel comfortable with the fact that you cannot reduce any living ecosystem to the sum of its parts. The very essence of life is an emergent phenomenon that happens through the interrelatedness and interactivity of all the parts. 


To do this you need Thought Form P5\index{thought forms (28)}  from Chapter~\ref{chapter:who-am-i-sense} to remind yourself that the ecosystem is constantly under construction, as is your knowledge of the ecosystem. This is an example of why the self-development section of this book is an essential part. Until someone has fluidity in all the thought forms needed to see something clearly, ecosystems remain too fuzzy or even hidden to work with them. 


You experience daily how power is active throughout ecosystems, at all scales, and all types of power. Remember how power was once more limited to political power\index{power} and money power, and in the hands of the few? An Economy of the Free is the effective, desirable end of the growing trend today, where every single capital exercises power, every single stakeholder group exercises power, in each nested layer of this ecosystem of ecosystems. This diverse spread of all powers exercised by many leads to the natural checks and balances that keep an Economy of the Free\index{Economy of the Free} at maximum adaptive capacity and regenerative capacity. Unleashing power \emph{to} is far more satisfying for you than the old economy’s power \emph{over}\cite{kahane-power-and-love}.


And this wide diversity of power\index{power} to, in the hands of all, is enabled by the strong bonding within each ecosystem\index{ecosystem}, from the local to the global. Everyone knows that they are contributing their uniqueness as an accepted and valid member of one small local micro ecosystem, one medium-sized regional ecosystem, and one global ecosystem of businesses. The full value of power in uniqueness and the full value of oneness as part
of all, is enabled by the strong bonding within each ecosystem of your daily experience.


Adam Kahane's\index{Kahane, Adam}  book, \emph{Power and Love}\cite{kahane-power-and-love} describes well the balance needed in yourself, your organisation, and your ecosystems between power to do; and love, which refers to everything that bonds us together in our oneness. Power enables your individual uniqueness to flourish at each scale, and love enables all to stand together in the strength of your common oneness at each scale.


These ecosystems have much in common with the beehives. The beehive is the primary living entity, not the individual bee. Each bee is an autonomous self-governing member of the hive; it decides for itself what to do, when, and how; with input from what all the other bees are communicating. And each bee is doing that within predefined functions that support the hive as a whole.


The hive has far greater adaptive capacity than any individual bee. The hive as a whole reacts more and faster to evolutionary drivers. If the temperature goes outside the range of an individual bee, the air conditioning bees become more active keeping the hive cool or warm. If some kind of predator comes close to the hive, the bees will attack even though each bee that stings dies in the process. Very much like you do not hesitate to sacrifice a few skin cells to pick up something rough.


Occasionally a FairShares Commons\index{FairShares Commons}  business in one of your ecosystems recognises that it is time to die. Maybe it merges with another, or because it has run its course. This is an enormous relief because zombie companies\index{company!zombie}  are no longer being kept alive artificially, draining financial, human, and environmental capital. Instead, companies voluntarily die gracefully when the conditions and need (a.k.a. the driver) that called that company into existence disappear, or change enough that the company no longer fits a niche. Then all the capitals\index{capitals} and resources in it are immediately put to use elsewhere in the ecosystem\index{ecosystem}, rather than wasted. This is why a FairShares Commons\index{FairShares Commons} ecosystem is inherently better for all stakeholders. (Apart from the few only interested in taking at the cost of others, because now they cannot easily take.) 


Occasionally, the smaller-scale ecosystems die in some way, because they no longer fill a niche, or perhaps they have grown too old to hold together and hold up all the power flowing through them. Again, this is a welcome and natural part of a thriving, living set of nested ecosystems. 


Autopoiesis\index{autopoiesis} is a recognised and maximised capacity of all ecosystems at all scales. As in nature, every entity at each scale has its own natural lifespan and life cycle, from birth through growth and maturity onto old age and eventual death. This is welcomed as a fundamental source of adaptive capacity\index{adaptive capacity}  in the system.


Sex is the other essential adaptive capacity in ecosystems of living entities and between ecosystems of living entities. The necessity of exchanging, mixing and matching DNA in multiple combinations, maximises the adaptive capacity. Whilst every single company and ecosystem has a clear boundary, it also has a thriving exchange with other similar companies and ecosystems. This thriving exchange of DNA leads to offspring companies that have something new, and every time there is some big change or shift, one of these new companies and new ecosystems turns out to have exactly the new, unpredictable, and unplayable combination of DNA\index{DNA} needed to thrive.


So FairShares companies and ecosystems die a natural death, with all their accumulated capitals shifting to those new companies and ecosystems being born because they  now best fit the new challenges. This mix of random exchanges of DNA, and the high interactivity within and between ecosystems of all sizes and up and down the nested hierarchy of ecosystems, becomes a core part of living a satisfying, thriving life in businesses that are good for you to work in; in an economy that does the job of provisioning; and on a planet whose living ecosystem supports life better each year. The common ground between ecology, economy, and ecosystems is visibly harnessed, rather than relegated to dictionary entries on the origin of words.


There are no more environmental externalities\index{externalities} because it's clear that the global business ecosystem fits into and is supported by the global natural ecosystem.


In many situations in your old economy, you really need to trust another human being to be able to work effectively with them, especially if they were part of another organisation. Often the best you could do was to set up some kind of contract based on what you do to each other if something went wrong. In an Economy of the Free\index{Economy of the Free}, trust is now first and foremost trust in the structures, processes, and norms of how to interact with each other in an ecosystem and between ecosystems. Trust is externalised.


The biggest reason to trust lies in knowing that every time there is some breakdown or conflict, you have exactly the right tools\index{tools} to use that conflict \index{conflict} to adapt and improve yourself and your ecosystem.


In physics\index{physics}, free energy\index{energy!free} is the fraction of the total energy\index{energy}  that is available to nature to use. In the Economy of the Free\index{Economy of the Free}, your free energy will be the highest it can be. You will have more energy to put to work doing the things that you enjoy, for yourself, for your friends and family, and for your colleagues. The free energy of each ecosystem at each scale is as high as possible. Very little is wasted in unproductive friction, what we call Job~2 in Part~\ref{part:you}.


Imagine how much more your life is now.
\section{Regenerative capitalism}
\label{section:regenerative-capitalism}
I firmly believe that capitalism\index{capitalism} is an essential, powerful tool that we cannot do without in rising to the adaptive challenges of our climate emergency\index{climate!emergency} and the 17 SDGs.\index{Sustainable Development Goals, UN 17} 
But, to paraphrase Star Trek, it's not capitalism as we know it today, Jim.


We need regenerative capitalism, \index{capitalism!regenerative}  a big capitalism that multiplies all capitals; unlike our current small capitalism. 


\begin{itemize}
\item Regenerative capitalism treats all capitals equitably. 


\item Regenerative capitalism has at least one currency \index{currency} for each capital. A naturally right currency, where the currency reflects the value attributed to the capital in unbiased flow. (As you will read below, in my usage of the word, what you may think of as different national currencies are all one currency: positive interest bank debt.)


\item Regenerative capitalism delivers appreciation, growth of each capital in its own natural units. Regenerative capitalism harnesses the power of business, now to multiply all capitals, including natural capitals, not just money.


\item Regenerative capitalism gives governance power to each capital, proportional to the overall meaning of that capital to the ecosystem\index{ecosystem}. Regenerative capitalism recognises the roles of a multiplicity of stakeholders representing all the capitals as members of corporations. Each capital\index{capitals} relevant to the corporation is part of the governance steering it into the future, and shares in the wealth\index{wealth}  generated by the corporation\textemdash the wealth generated in the capital that you represent, and an appropriate share of the wealth generated in the other capitals because of the role played by the capital you represent.


\item Regenerative capitalism treats ecosystems\index{ecosystems} as primary and is built up of a nested hierarchy of ecosystems stretching from your global ecosystems all the way down to you and your family.
\end{itemize}




Taken together, these facets of regenerative capitalism\index{capitalism!regenerative}  give you the foundation you need to build the antifragile reality we need for all to thrive\cite{russell-thrivability}, one going beyond a circular economy, beyond sustainability, to a regenerative economy. 


This is a missing ingredient for green (or blue) economy\footnote{The blue economy concept is both used to refer purely to marine life\cite{wiki-blue-economy}, and inclusively by Gunter Pauli for the whole planet\cite{pauli-blue-economy}.}, an economy for the common good\cite{ecg-website} (an innovative business scorecard looking at a much broader range of indicators than typical business scorecards), and the triple bottom line (people, profit, and planet as equally important measures of business success, not just profit); integral impact investing\cite{dellner-integral} (integrating multiple different elements according to Wilber’s\cite{wilber-integral} or Lessem’s approach), and much more. 


I'll expand here on what I wrote about property\index{property} and freedom\index{freedom}  as complementary pairs\index{complementary pairs}. Clearly the concept of property is something that we need, in order to construct a viable economy, as is the concept of freedom. 


At some point between you as a free-living being where it's clearly harmful to apply any concepts of property, through to the equipment you use, where the concept of property is beneficial, there's some transition zone. Maybe similar to quantum physics, a transition zone where both concepts are simultaneously valid for everything in that zone. (Even if something is owned by a group of people, as in the sharing economy\index{economy!sharing}, the concept of property is still being applied. For a commons to function, the concept of property is essential.)


So to build a regenerative economy\index{economy!regenerative}, we need to figure out what best belongs in the meaning\hyp{}making story of freedom only, what best belongs into the meaning\hyp{}making story of property only, and what we may need to invent that integrates both as a complementary pair.


For example, take a forest. In economics\index{economics}  today almost all forests are part of the concept of property. They may be owned by an individual, a corporation, the state, or they are not owned and therefore open to being claimed as property by anyone ready and willing to do work on the forest. Equally, the atmosphere, or the Arctic and Antarctic can fall into the concept of property, even if no one has yet claimed ownership; or into that of freedom; or into something new reflecting them as complementary pairs.


I believe that the regenerative economy\index{economy!regenerative} we need to address our climate crisis is based on applying the meaning\hyp{}making story of freedom to all large scale ecosystems\index{ecosystems}, certainly all those at a planetary scale, at least down to corporations as non-human beings, and yourself as a human being. If we use anything of property at all, it will be to enable freedom (i.e., the full complementary pairing),\index{complementary pairs} certainly not in opposition to freedom. 


In a regenerative economy, where the atmosphere is as free as you are, and where your role becomes one analogous to a parent\textemdash speaking on behalf of something that is free but unable to speak itself.


The Economy of the Free\index{Economy of the Free} is a fully regenerative economy\index{economy!regenerative} across all capitals\index{capitals}; the ones that we have invented that the consumer society values, and the ones that have always been part of nature because life itself values them. All the companies manufacturing any product are part of small and large circles in a circular economy. Every single product at the end of its life cycle is the raw material for something else.


The small and large circles function because the companies are all FairShares Commons\index{FairShares Commons} companies. So all have an appropriate benefit from the wealth generated by each company in the circles as the product moves around the circular economy. Now, every company in the circle has the incentive to do what is right and regenerative for the whole across the long-term, and that is in the interests of society as a whole across seven generations.


The triple bottom line\index{triple bottom line} is a no-brainer because all categories of stakeholders are engaged in the decisions taken at the general meeting level. The company stewards that are legally required to veto any decision that might lead to a loss of freedom\index{freedom}, and any decision that is benefiting the present at the expense of future generations, are naturally maintaining a triple bottom line. In fact, this triple bottom line is in the deepest spirit of John Elkington's\index{Elkington, John} 1994 proposal\cite{elkington-triple}.


There is almost no need for regulation or legislation because the interests of all stakeholders including future generations are fully represented within each FairShares Commons company and within the ecosystems of companies at all scales. Instead of externalising the power to meet these needs to the institution of government, the power to ensure these needs are met is internalised within each company and their ecosystems. The distinction between you as a citizen, you as a consumer, you as a voter, and you as a worker and manager still exists, but you now wear all those hats within the business ecosystems you are part of.


I (Graham) am an advocate for R3.0\footnote{https://www.r3-0.org/}. It is developing everything needed to redesign our economy to build resilience in our social and ecological systems and regeneration\index{R3.0}  beyond the baseline of social and ecological sustainability thresholds to thrivability. It has an excellent set of templates, complementing everything in this book, and our Evolutesix\index{Evolutesix} materials, on how to create a regenerative economy and build a better world. 


I strongly recommend you join R3.0. It has nine blueprints available:


\begin{enumerate}
\item   Reporting
\item   Accounting
\item   Data
\item   New business models
\item   Transformation journey
\item   Sustainable finance
\item   Value cycles
\item   Governments, multilaterals and foundations
\item   Educational transformation.
\end{enumerate}


For any of these approaches to work, we must build them on the freedom\index{freedom}  needed for accountability. Only the free can take accountability for the consequences of their actions. Only a free company, a commons of joint capacity inclusive of all stakeholders\index{stakeholders}, can ever be the building blocks of a regenerative and circular economy. \index{economy!regenerative} 


\emph{A free economy only works well when it is an Economy of the Free}.\index{Economy of the Free} 
\section{Escaping gravity}
The stories that create the reality you experience are like gravity. They are always pulling you in one direction. In the Economy of the Free, many stories have been rewritten, and you have escaped the reality of the old ones. More importantly, you are now using gravity in support of your freedom.


If you have ever tried juggling, you know that you need gravity\index{gravity}  in order to juggle. If you are in deep outer space, with almost no gravity, and you throw a ball up, it is not going to fall down. Here are a few examples of how you are escaping the old gravity and juggling with the new gravity.


\subsection{The government's new role}
A government’s reality is as much shaped by the stories of what government means as anything else. In the new Economy of the Free, the role of government has changed fundamentally, because the story is now one of maintaining the integrity of freedoms fundamental to the Economy of the Free\index{Economy of the Free}. From your individual freedom, through the freedom of your organisations, to the freedom of humanity as a whole to thrive. 


The government is now there to maintain the overall functioning of society and our business ecosystems of the largest scale, much as you see city municipalities doing today at the city level. Almost all regulation, almost all roles of government\index{government}, is now happening within the ecosystems\index{ecosystems}.


For example, the needs of towns, cities, regions, and nations to have enough money to build and maintain services are now primarily met by a share of the surplus wealth\index{wealth} generated and realising their share of capital gain.


Instead of elected government officials, and civil servants, engaging in battle on your behalf with multinationals headquartered in a different country, if you are one of the stakeholders you are directly part of these companies’ general meetings and ecosystems. As a customer, as a supplier, or as a member of some other stakeholder group, you attend the general meetings and have the power\index{power} directly yourself to protect your interests in a way that balances out with the needs of other stakeholder\index{stakeholders} groups to protect their interests. With the seven-generation requirement, there is very little need for regulation to protect the environment long-term.


The proponents of both light and heavy government intervention end up getting what they want.


The climate emergency\index{climate!emergency} is now being dealt with using the full power that we have. Decades of failure by national governments to come to a global agreement on climate change and what to do about it have turned into direct agreement at all scales up to the global ecosystem because each scale representatives of that scale are coming into agreement with their neighbours.


In the Economy of the Free, governments of nations, as well as regional governments and representatives of blocs of multiple nations all have a role to play, the roles that all the other stakeholder groups and ecosystems play. Governments represent the uniqueness of one country, in balance with the different agents across all the scales of the Economy of the Free, representing the oneness and the uniqueness of each layer within the ecosystem of ecosystems.


Lobbyists\index{lobbyists} still exist, but the role they play no longer has the disproportionate power that money currently buys. There's the full diversity of engagement in lobbying, dialogue and decisions at all scales of the hierarchy of ecosystems.
\subsection{The economy's new stories}
\index{stories} 
The central story now in the Economy of the Free\index{Economy of the Free} is that it is illegal, ineffective, and morally wrong to apply any part of property law to any individual, any incorporated business, or any ecosystem\index{ecosystem} of businesses at any scale from the smallest through to the global business ecosystem.\index{ecosystem} 


Different capitals may or may not be owned within that, but only to the extent that owning the capital does not prevent the freedom of any legal person, whether human or nonhuman, nor ecosystems of legal persons.


You know that your economy is inherently complex, sometimes on the edge of chaos, in the sense of Dave Snowden's\index{Snowden, Dave}  Cynefin\cite{snowden-cynefin, cynefin} diagram\index{Cynefin}. So you no longer try to gather lots of data in order to analyse it and identify good practice. Instead, you know that the economy in detail is emergent, which means acting, monitoring, and reacting. 


The story of your economy is now one of moving, changing, and adapting faster than the drivers of the economy are changing, rather than attempting to predict or control the economy.




\subsection{Economists' new role}
Economists\index{economists}  now are more like sports psychologists. They develop the capacity of ecosystems at all scales to react quickly to what's happening, and grow their adaptive capacity, but it's the ecosystem itself that does the reaction and balancing.


The work of economists no longer has an anchor in the few owning property or capital and renting it to the many. The new story of economists, just as the new story of government,  revolves around freedom, and multiple capitals\index{capitals!multiple}, and the access to those capitals for all. The central role of economists is understanding freedom and the flow of capital, from where it is abundant to where it is needed or being unnecessarily constrained.


The central story of economists is now about provisioning at all scales of the ecosystem, across all capitals and currencies. 


In some ways, the economist is now much closer to the story and work of a gardener or farmer. Now that businesses are legally full living beings free from any application of property law\index{property!law}, and stewarded as any other voiceless living being would be, the role of economists is to create the conditions for these living beings to thrive. And to study the consequences of different types, numbers, and densities of businesses, in one ecosystem; the flows of capitals within ecosystems and between ecosystems, and what enables those capitals to flow. 


Most importantly, the economist’s role is constantly experimenting, as evolution does in nature, with small variations in how the Economy of the Free\index{Economy of the Free}  is structured and its processes. And then keeping what works better, and leaving what works less well. Providing the feedback loop that nature uses in evolution to maximise the adaptive capacity of its living ecosystems.


The central story of economists\index{economists} is maximising the adaptive capacity\index{adaptive capacity}  of living business ecosystems, keeping them on the edge between complexity and chaos\index{chaos} in the Cynefin diagram\index{Cynefin}, at the sweet spot for maximum speed and adaptive capacity in response to the changes of context that drive evolution\index{evolution}.


\subsection{Explore, experiment, improv}
You learn how your meaning\hyp{}making stories\index{meaning-making stories}  shape the reality you experience, and how to use exploration, experimentation, and improv\index{improv} to generate new experiences that your stories can use to rewrite themselves in chapter~\ref{chapter:who-am-i-meaning}. Everything that you read about there, along with your capacity for transformational thinking from chapter~\ref{chapter:who-am-i-sense} is what you and all the players across all stakeholder groups and ecosystems have used to dismantle the old stories and craft the stories that underpinned the Economy of the Free’s reality.


Since we're talking about a Cynefin complex economy and journey towards your Economy of the Free, improv\index{improv} is essential. You got to the Economy of the Free by going out there and using improv to do things that differed from your old story. By just going out and doing, and seeing what happened, and then reacting, step-by-step you gained the experiences you and everyone else needed to craft the new stories that shaped the new reality.


Experts, including economists, are now highly skilled in improv and experimentation, and it is a core component of their training. 


\section{The stock market looks like}
\index{stock market|(} 
If every company was a FairShares Commons\index{FairShares Commons}, and part of an ecosystem\index{ecosystem} of them, which itself was embedded in successively larger ecosystems, our stock market would function in an inclusive way, doing the job we need it to do. It would ensure that the economy as a whole has the capacity to (re)generate wealth\index{wealth} in all capitals. 


The stock market today, as you read in Section~\ref{section:stories-define-economy}, is created by stories, e.g., that it is better to own 100\% of a small pie than a slice of a large pie. Even if your slice of a large pie has more pie in total than your owning all of a small pie. So the reality we experience on the stock market is a competition between investors\index{investor} for the largest percentage of the company, regardless of the impact on the company, its stakeholders, and on society at large. Protect the long-term values integrity of your impact investment by incorporating as a FairShares Commons.


The unbundling of governance, information, and wealth-sharing rights and obligations in Free and FairShares Commons\index{FairShares Commons} companies shifts the stock market from pure competition to an optimum mix of competition and collaboration.


Steering the company into the future through your governance rights and obligations is distinct from any share of the wealth generated through past activities of the company. (Past and future are no longer hard-wired together in the bundling of rights in the share.) This means that the interaction between the people engaged in buying and selling the new kinds of shares that represent all kinds of capitals in the stock market automatically leads to an alignment of each person's selfish needs, the company’s needs as a whole, and the needs of society as a whole. You can read more in the article by Jordan Barry et. al\cite{barry-game-theory}.


In one fell swoop, this meets the needs of everyone calling for market freedom and a minimum of government regulation; and everyone in favour of the high levels of regulation found in the highly socialist countries because the stock market delivers the objectives of both extremes in one.


By now, you might have recognised how this is the application of exactly the thinking that transformed classical physics\index{physics!classical} into quantum physics\index{physics!quantum}. The recognition that there actually was no contradiction between particles and waves; rather, what really worked was their integration into complementary pairs\index{complementary pairs}. By seeing competition\index{competition}  and collaboration through the lenses\index{lens} of the transformational thinking patterns of Chapter~\ref{chapter:who-am-i-sense} we get to a libertarian capitalist stock-market that also meets the objectives of Marx!\index{Marx, Karl} 


So, unbundling the rights of shares\index{shares} brings a representative breadth of all stakeholders\index{stakeholders} into governance and the company’s welfare; and extending the legal personhood of a non-human to the full extent of the legal personhood of a human includes the right to freedom from ownership and the right to self-determination. And these both shift the entire paradigms of the stock market from our current classical paradigm to a new one. 


Then what happens when we add in an ecosystem of markets spanning the full breadth of all capitals and all currencies?


We end up with the central pillar that we need in a regenerative economy\index{economy!regenerative}; a mechanism for all capitals\index{capitals}  to grow. And this, because we all behave as ourselves, with our human spread of needs, interests, and biases\index{biases}.  Not because some higher power (whether that is regulation and policing, religion and social coercion, or inner values) has forced us to act contrary to our stories or nature.


And this means that every capital grows in its own currency\index{currency}, i.e., units of value, not necessarily as measured in money.


Of course, because this is a stock market that spans all capitals, and because the stories and dynamics are based on extending freedom to all on the market, both the humans and companies as living beings, the stock market will grow everything in a way that balances from system to system. Each pie will grow to the point where all the pies are at their biggest, and not beyond that. Nothing will grow at the long-term cost of something else, just as in nature if the apex predator grows too large, it will very soon be cut back to a sustainable level. 


As you have read so far in this book, it makes a lot of sense to regard a company as a living being. The natural consequence of this perspective closes the last gap between human and non-human legal persons, namely the removal of any element of property law applied to any legal person. As you have read, that opens up exactly what we need for companies to engage with full responsibility and accountability in society, including on the stock market.\index{stock market|)} 


\section{Boom and bust looks like}
The cycles of boom and bust that are part of the reality we experience are created by the stories that run us. Physics\index{physics} is full of examples similar to boom and bust that we can learn from. For example, start with a bunch of magnets all pointing in the same way. Then start up a magnetic field pointing in the opposite way and slowly increase it. At some point all the magnets will flip to the opposite direction.


In physics, these flips from one state to a very different state are very well understood. One characteristic of systems that flip from one state to another is that each element (e.g., a single magnet) only has a very small number of choices (point north or south), and there is very little diversity in the system (we're all the same kind of magnet). As soon as you increase the number of options that each individual has, and especially if you increase the diversity, you no longer get extreme flips.


The booms and busts that we have are a consequence of an economic system that emerges from our dominant stories of ownership, separation of investors\index{investor} from all other stakeholders, separation of the next quarter from the next seven generations, and everything that you have read in this book up to now.


The cycles in an economy of free people and companies will have quite different dynamics driving them. Boom and bust will still happen, but the antifragile nature of an Economy of the Free makes them flatter and further apart. 


What you will have is what you find in nature. Within a certain sub-ecosystem, for example between the fish and the algae of a pond, if something happens that drastically reduces the number of fish eating the algae, (a heron arrives and eats them all, except the 10 babies that have just hatched) the pond may get covered by algae. Sooner or later those 10 baby fish will grow up, feed on the plentiful supply of algae, have many babies, eat up all the algae, and then begin dying off because they don't have enough food. 


Boom and bust, but only that pond. All the other ponds across the world will be fine. And even that pond would have been fine if it had multiple different types of fish, multiple different types of food for the fish; in other words, a sufficiently complete ecosystem. Learning from nature as a whole, biomimicry, is now a clear part of the expertise the new generation of economics students know they need. 


So you will have in the smallest ecosystems boom and bust cycles, but these will usually be harmless for the global economy. 


You can see how this plays out looking at the Spanish cooperative Mondragon\index{Mondragon}. It began in 1943 as a technical college founded in the Basque region of Spain by a priest and was incorporated in 1956 when he selected for graduates to build a cooperative fully in line with his humanitarian Catholic teachings. 257 companies, 74,117 employees in 2014, and whilst individual companies shut down or needed to reduce their workforce to survive during the 2008 global financial meltdown, all staff were retained and supported within the ecosystem. 


Not one person was added to Spain's unemployment statistics. Compared to the average company in Spain, which collapsed, or if the company survived, shed employees and failed to hire recent graduates at such a rate that the unemployment level rose from 8\% in 2007 to 20\% by 2010 and peaked at 27\% in 2013.


An Economy of the Free\index{Economy of the Free} globally will be like Mondragon, scaled up and on steroids. Individual companies will be able to start, grow, and scale as rapidly as fruit flies or as slowly as a sequoia. But the ecosystem as a whole will stay well-balanced because the huge diversity and interactivity across all stakeholders and engagement of all stakeholders\index{stakeholders} generates antifragility.\index{antifragile} 


Even more powerfully, all scales of the economy\index{economy}  are naturally regenerative ecosystems  of companies. From small ecosystems to large, you'll have the appropriate elements from banking, manufacturing, IT, artificial intelligence\index{artificial intelligence}, upcycling, etc. This diversity of businesses with mutual independence and interdependence is again exactly how nature creates antifragility.


You cannot ask for a better moat protecting your retirement from the consequences of our global boom and bust cycles.


You probably know people who lost their job because their company went bankrupt, or was close to bankruptcy during Covid-19,\index{Covid-19}  or further back in 2008. You may know people who lost their house. These people suffer the consequences of the boom and bust cycles that exist because of the dominant stories that shape the reality we experience. If you are the kind of person that solves these problems at their root cause, think of starting up a FairShares Commons\index{FairShares Commons} company today, linked together in ecosystems with other FairShares Commons companies, and your children at least, and possibly you, will never experience that again.
\section{Trade}
\index{trade|(} 
Trade today is partially free. Currently 80\% of our trade is between multinationals\index{multinationals}  within their supply chains. This trade freedom is primarily in the interests of the multinationals and the global finance industry based on money.


In an Economy of the Free\index{Economy of the Free},  trade will have the maximum freedom at all scales of the economy, from trade between individuals, through trade within and between local and regional business ecosystems, and onto global trade within the global business ecosystem.


The maximum freedom in trade is another application of the complementary pairs found in quantum mechanics. The freedom of each individual to be fully their unique selves and have all their unique needs met, and the freedom of all of us in our oneness as humanity to be fully ourselves together, form a complementary pair. The maximum freedom in trade will be at the sweet spot integrating those complementary pairs. 


Increasing your freedom\index{freedom} at some point begins decreasing our freedom. Vice versa, at some point increasing a community’s freedom takes away freedom from individuals. You and the Economy of the Free will be at the sweet spot of this complementary pair of freedoms, but this sweet spot keeps moving, and has fuzzy edges, because it depends on everyone’s inner realities.


Today, the dominant trading units are multinationals. In an Economy of the Free there will not be any dominant trading unit. Trade will happen at all scales. 


In this sense, trade will be local, within the smallest of business ecosystems, and conducted through the currencies within that ecosystem. Trade will be between ecosystems of the same scale, from neighbouring ecosystems through to those half-way around the world. Trade will be up and down the hierarchy of ecosystems of ecosystems. All using the complementary currencies they find most useful to represent the full spectrum of capitals being traded.


Trade today is largely managed for you in an opaque way, by the interests and power structures of your politicians, finance industry, and multinationals. You don’t have much potential to engage in governing it yourself.


In an Economy of the Free\index{Economy of the Free}, trade is not managed for you by any higher power; it is fully emergent. Fully emergent trade, unlike managed trade, is self-organising, self-governing, and self-managing. It is autopoietic, i.e., having full capacity to create, reproduce, and shut down any aspect of itself. Creative destruction\cite{schumpeter-capitalism}, upgraded to the Economy of the Free. This will result in trade automatically leading to antifragility of the global economy, because it will be the scaled-up version of each of you and your businesses’ adaptive capacity.


In an Economy of the Free, \index{Economy of the Free} there will be rules of trade. Unlike today, where the rules of trade are defined by a few individuals in positions of power, the rules of trade will emerge from the self-governing processes within and between ecosystems. In full adaptive capacity, any tensions or conflicts will be harnessed for what they are: information on what is not working, and how to adjust the rules, structures, and process of trade so that it works in the maximum interests of all and freedom of all at all scales.


There will still be power hierarchies, from you and your power through to global organisations that use their power to meet their needs. 


The big difference versus today, though, is that there will be power hierarchies at all scales, of all types, that lead to an emergent dynamic equilibrium\index{equilibrium}  because of the wide diversity of different powerful stakeholders with different types of power.\index{power}  Just as in nature\index{nature}, the lion has the power to meet its needs, and the virus has the power to meet its needs. The virus \index{virus} can bring down the lion, just as much as the lion can bring down the springbok. 


Nature as a whole thrives because power balance gives it the adaptive capacity that it needs.


In an Economy of the Free, trade will be local, where local best serves the interests of all; it will be regional and multiregional, where regional trade best serves the interests of all; and it will be the global business ecosystem, where global trade best serves the interests of all. Deciding on those interests and how to best serve them will be done within and between ecosystems by the stakeholders representing the interests.


You will be an intrinsic part of that decision process, representing your interests.


This will neither be self-reliant at a local or national level, nor will it be dependent on other regions or nations in the kind of polarity you see today. Rather, it will be inter-reliant and interdependent in the same way that our natural ecosystems are. Just the same as life on the planet is one interdependent and inter-reliant hierarchy of interacting scales of ecosystems, all the way down to every single living organism.\index{trade|)} 




\section{Finance looks like}
\index{finance|(} 
In this section we are expanding the underlying story of what finance and the finance sector means. In the Economy of the Free\index{Economy of the Free}, the finance sector includes all the capitals\index{capitals} and their associated currencies that play a role in the economy doing its job of provisioning\index{provisioning}.


Today, large amounts of your pension fund\index{pension!fund}  are invested by your asset managers in businesses and assets that have no value, or are even harmful, in regenerating a planet that you will want to retire in. In an Economy of the Free, all the capitals that you put aside to make use of when you retire, or are ill, are put to work towards regeneration. 


Instead of your pension being invested in mining coal for fuel, or oil for fuel (despite every \euro{} invested in extraction from the ground requiring multiple euros to extract the carbon back out of the atmosphere), your pension will only be invested in businesses that are regenerative. After all, what good does it do you to put lots of money into a pension fund, only for the underlying asset base to have no value in the world that you are going to retire into? 


Finance operates at all scales and in all currencies (not just in positive interest bank debt monies) in an Economy of the Free. 


So if you need any kind of capital to start up a business that is at a small scale within your local ecosystem, you will have sources of those capitals available. This will not be free, without any strings attached; in fact, your peers in the ecosystem may well demand even more scrutiny and engagement with your plans than today before you get the resources you need.


But these will be free people and free businesses that are part of your ecosystem. They will get to know you, and your track record, and develop a deep understanding of what you plan to do with the capitals that they invest with you. They will deeply understand how you will regenerate the capitals many times over if you are successful in your endeavours.


The finance \index{finance} sector is broad across all capitals, regenerative across all capitals, and in service of that regeneration. Not as we see it today, where capital is of service to the finance sector. Leaders have a full understanding of the nature of money, and the currencies of other capitals, along with great capacity for the post-logical thinking and systemic complexity\cite{henderson-hourglass} of Part~\ref{part:you}.


Money\index{money}  is no longer the primary basis of full power\index{power}  in the economy. 


There are still some individuals who, through luck or their natural talents, have large quantities of money. But this does not turn them into a plutocracy with power over those in need of that capital, because all capitals are on an equal footing. So you have a large diversity of capitals and a large number of people who are wealthy in at least one of those capitals.
\section{Sustainability looks like}
\index{sustainability|(} 
In an economy where you are a free living being, and your company is too; where  companies are stewarded with the next seven generations in mind, all stakeholders in the ecosystem are involved in governance and share in the wealth generation, and all capitals are valued for what they contribute to life; sustainability, regeneration and circularity will inevitably emerge fast without extra effort or cost. So whatever happens in one generation, it increases the viability of future generations. 


In an Economy of the Free, there are few regulations, e.g., telling you to use the train instead of flying. Instead, you have the freedom to use your wealth across all capitals in whatever way seems best in your reality, including being part of one whole. Just as in nature, acting in what you judge best in your reality will mostly lead to bottom up, emergent outcomes best for the ecosystem as a whole, because the inherent nature of a FairShares Commons\index{FairShares Commons}  aligns your best interests with the best interests of your organisation and society. 


This Economy of the Free satisfies the description of sustainability created by the Brundtland Commission:


\begin{quote}
development that meets the needs of the present without compromising the ability of future generations to meet their own needs. [\ldots] meeting essential needs requires not only a new era of economic growth for nations in which the majority are poor, but an assurance that those poor get their fair share of the resources required to sustain that growth. Such equity would be aided by political systems that secure effective citizen participation in decision-making, and by greater democracy, in international decision-making\cite{wced-sustainability}.
\end{quote}


Clearly an Economy of the Free\index{economy}  composed of ecosystems of FairShares Commons companies satisfies all of this. We are all poor in one or more of life’s capitals, whether it's time, energy, or money. In the FairShares company, all invest the capitals they are rich in, and benefit from the wealth\index{wealth}  generated in capitals\index{capitals}  they are poor in. Because all stakeholders\index{stakeholders}  engage in the company’s long-term direction via the general meetings, everyone is involved in decision-making.


Many distinctions between people disappear in individual companies and the economy as a whole. All show up wearing their respective stakeholder hats and engage in stewarding the company wisely into the future. Everyone involved in a FairShares Commons\index{FairShares Commons}  company is also engaged in global governance, because FairShares Commons companies naturally form into ecosystems, and ecosystems of ecosystems, that will eventually touch everyone.\index{sustainability|)} 
\section{Work and reward}
\index{work and reward|(} 
\label{section:ubi}
An Economy of the Free\index{Economy of the Free}  is not purely joyful nor fully fair. It will be unfair, but the unfairness is shared by all. There will be change, and disruptions to work\index{work}.  There will continue to be great shifts in the types of work needed and available, and times when you must reskill yourself.


But in an Economy of the Free, the ecosystems of free FairShares Commons\index{FairShares Commons}  companies support that reskilling. Companies will neither keep themselves nor specific jobs alive purely to avoid the evolutionary pressures of changing work context and disruptive innovation. Instead, you have an unconditional basic income (UBI)\index{unconditional basic income}  and the same access to developing yourself that you may have enjoyed through free schooling. In a Economy of the Free this is now lifelong, recognising the lifelong need for development.


While writing this book the Swiss held a referendum on universal basic income (UBI), which sadly failed to pass, with only 23\% of voters backing it. The proposal was to give every adult legally resident in Switzerland an income of 2,500 Swiss francs (USD 2,554) per month. But UBI will come.


An objection you may have to UBI is that it will remove incentives for people to work. There is some truth to this, but it hides other truths. The past century and a half in the industrial era, and going even further back than that, has shown the value of the right incentives and rewards for work. There is equally strong evidence that reward \emph{dis}-incentivises work.


How can that be possible, a lot of people will be thinking now? How can paying somebody to do work, be both an incentive and a disincentive? It depends crucially on aligning why somebody is motivated to work (the meaning they make of work and themselves), and the work itself.


In today's economy\index{economy}, the dominant story\index{stories} is that work means suffering in order to earn money to consume what brings you joy. 


In the Economy of the Free, work\index{work} brings you the financial income that you can use to meet your needs. And work also brings you many other capitals, enabling you to meet many other needs. In particular, it becomes part of your self-development. Instead of your work being restricted to what will earn you enough money to survive, it is an inherent part of your freedom to develop into who you can become. And for your businesses to develop into who they can become.


Bringing in a basic income for all citizens is, for society and the economy, changing the background that you are embedded in. Following the principle of Ubuntu\index{Ubuntu}, it will lead to a fundamental change of self-identity, of our fears and hopes.


Everyone has at least some entrepreneurial, creative aspects, and some level of commitment to working on something that you find truly valuable. 


Your motivation is not just money. What motivates you also includes things like the sheer joy of mastery, self-respect, the respect of your peers, the motivation of seeing the end of something that you have created being used by others, the autonomy of being able to decide for yourself, and own what you do with your own time and energy. Maybe for you, having a very clear purpose to make a difference in the world.


\begin{longstoryblock}
When I (Jack) was younger I worked in construction. Years later I passed a building that I had helped to build and took pride in contributing to creating something of quality that was still standing, that still fulfilled its purpose. 
\end{longstoryblock}


All these will be equally possible in an Economy of the Free\index{Economy of the Free}  because people and businesses are free to make the most of themselves. Adapting the quote from Abraham Maslow\index{Maslow, Abraham}  (Chapter~\ref{chapter:know-our-economy}), in an Economy of the Free, if you are a poet, you are free to be your best poetic self; if you are gifted at multiplying capitals, you are free to multiply capitals; if you are creative, you are maximally free to create. Regardless of whether you are an individual or an organisation. 


Your work and purpose are aligned. Your work will centre around everything that is of value to you, from developing yourself, through developing and caring for others, to anything else. The story of your work is now the story of your meaning in life, not the story of consumption, nor the story of survival. Your work is now one of the places where you are your best self, and the best possible playground to develop yourself into who you can become.


This is not a life of simple bliss, free of stress. Instead, you actively seek out stress because that is just what you need at work to give you the direction and energy you need to grow yourself.\index{work and reward|)} 
\section{Your life looks like}
In an Economy of the Free\index{Economy of the Free}, your life will be very different to what it can be in today's economy. Given that none of us have ever lived in a global Economy of the Free, it's impossible to describe precisely all the highs and lows you’ll experience in your life; but some are clear! 


The joy you have collaborating in a team that simply clicks, that delivers meaningful results fast and efficiently, will be there more frequently and reliably than today. Equally, the extreme tension that you feel when you are in conflict with people who see the world differently to you will remain, perhaps even more strongly than today. The big difference is that you know this is an essential part of the self-governance that gives you the freedom you enjoy.


You will revel in this freedom\index{freedom}  to be your best self, and the freedom of the FairShares Commons\index{FairShares Commons}  companies you are part of to be their best selves. You will spend time enjoying the benefits of all the wealth you and your FairShares Commons companies generate across all the capitals, enabling a better life through less economic growth\cite{hickel-less}.\index{Hickel, Jason} 


The shift that appeals the most to me is that in an Economy of the Free I will read headlines that talk about the regeneration of our climate. I will read headlines about the polar ice caps  regaining their ice cover, ocean currents regaining their power to balance temperatures between equator and poles. I will no longer read headlines about how the high summer temperatures have yet again broken the ice melt records shown in Figure~\ref{fig:arctic-death-spiral}.


In Section~\ref{section:western-economy-failing} I described the precariat\index{precariat}.  In the Economy of the Free the precariat cannot exist, because the economy does the job of provisioning for all. It is the Collabonomics that Klaus-Michael Christensen\index{Christensen!Klaus-Michael} introduced me to. You may still be working in the gig economy; you may not know from one day to the next whether you will be in paid work or not. You will not have any doubts, though, about your capacity to feed, clothe and house yourself, because the UBI\cite{standing-basic-income} takes care of your basic needs.


An Economy of the Free naturally includes the elements of the economics of happiness\cite{anielski-economics-happiness}. The success in applying the happiness economy’s principles across multiple scales, from business through to nations like Bhutan,\index{Bhutan} demonstrates that it has value, and will emerge as part of an Economy of the Free.


If you are any kind of investor\index{investor}, but especially if your focus is impact, circularity, regeneration, or sustainability, and expressing yourself, your legacy and values through your investments, the FairShares Commons and Economy of the Free gives you the most powerful tool I know of to do just that, to make your money matter. 


This is the kind of integration of a paradox that generates all disruptive innovation; here, by reducing your direct immediate control you get better protection over the long term of your intent and values. Because, as you will read in Chapter~\ref{chapter:create-your-FSC}, later investors cannot easily capture the company, suppress your intent, to extract money for themselves.