%Changed the Zhou Enlai wording on 9 Janaury 2021
% Added index to Granny Dot’s and Weatherwax


%\addtocontents{toc}{\protect\newpage}
%\addtocontents{toc}{\protect\goodbreak}
\chapter{Stories, lenses and maps}
\addcontentsline{toc}{chapterdescription}{The stories, lenses and maps we use generate the unique reality each of us experiences. The types of questions we ask, and how we ask them, determines what we can see through the lens we use, and interpret with our meaning-making story templates. Cargo cults and scientism put us at risk of fooling ourselves. We need to harness conflict and manage boundaries to get out of the mess that we’ve got ourselves into. The six layers of our economy, from each individual through to the global economy as an ecosystem of ecosystems, and how to turn conflict into antifragility in each layer.}
%\addcontentsline{toc}{chapterdescription}{\pagebreak}
\label{chapter:emergence-einstein-picasso}


\begin{chapterquotation}
Reality is merely an illusion, albeit a very persistent one.\\
\raggedleft\textemdash Albert Einstein\index{Einstein, Albert}


\centering
You do not see things as they are, you see things as you are.\\
\raggedleft\textemdash Anonymous\footnote{This has been attributed to so many different people the origin is best referenced as anonymous}.
\end{chapterquotation}




\section{Asking questions}
\begin{longstoryblock}
When my (Jack) kids were younger I told them that there was no such thing as a stupid question; that the only wrong question was the one not asked. But now I realise that there can be wrong questions, or at least harmful questions, because questions are fateful: they determine where you start, where you look, and where you end, if you end at all. In a way each question already includes its answer. 
\end{longstoryblock}


Far too many in today's world dogmatically believe that they know the right answer. That the only issue is not implementing that one answer well enough. 


However, effectively addressing the challenges we face today can only be done by creating generative dialogue\index{dialogue!generative} including multiple answers across all relevant disciplines, perspectives, and groups of people. Even more valuable than answers and knowledge of historical answers to historical questions\footnote{I realise that this statement itself, is me claiming dogmatically that I know the right answer. As you read the rest of this section, maybe you will be the person who can point out to me why I am wrong, and convince me that questioning and generative dialogue are irrelevant. Which I guess then means that we are in a generative dialogue, and I'm right! We are back to complementary pairs\index{complementary pairs}, and two opposite statements being simultaneously true.} is the capacity to ask generative questions.






So you can also see that, rather than being about right or wrong, questions should be more about generating new understanding, ones that neither entrench your current understanding, nor create misunderstanding.


The word ‘mu’ (Japanese, Korean) or ‘wu’ (Chinese), is often used in zen to say: \emph{please un-ask the question because it takes you away from understanding}. You can only know a viable question when you realise that the answers point in the wrong direction. 


Truly powerful questions are generative: they lead you to more answers and more questions in unexpected ways.


For example, each time Ernest Rutherford\index{Rutherford, Ernest} asked \emph{where is the electron?} he got a different answer. This told him clearly that something was fundamentally wrong with classical mechanics. Eventually he began questioning the question, realising that asking \emph{where is the electron} prevented him from seeing and understanding the true nature of an electron: it has no place. 


If you have been asking questions, and none of the answers have worked, maybe you are in the same place as Rutherford. These could be questions like: how do we regulate business, or fix the banks. I believe that questions like these are taking us further away from understanding what the root causes actually are, and how to build a system that works for all of us and for our planet.


Every question assumes that some class of answers exists. For example, Rutherford's question assumed that the concept of position had meaning and that an answer existed. It's a very rare question that is both useful and assumes nothing. Even the very broad question, \emph{how can we build an economy that works}, often assumes human society as it is now.


In the language of Chapter~\ref{chapter:who-am-i-sense}, specifically thought form C6, every question is based on some frame of reference. If the frame of reference is valid, you might get useful answers to the question, but if the entire frame of reference is invalid, then the question as a whole is invalid, and no answers to the question can have any connection with actuality\textemdash except by coincidence. Which is why I emphasize in this book the importance of uncovering the frames of reference you are using. The more fluid you are in all 28 thought forms\index{thought forms (28)} of Chapter~\ref{chapter:who-am-i-sense}, the better and faster you can see what is missing in how you are making sense of actuality\index{actuality}.


Of course, you might also be using the wrong technique or tool to answer the question. Perhaps you are measuring the height of a wall, but the tape measure you are using is itself 10cm longer at midday at 35\textdegree{}C compared to early in the morning when the temperature is 15\textdegree{}C, giving you useless answers without your realising it. 


However, this is easier to deal with. The underlying question itself is a valid question to be asking, and calibrating your tools and techniques will quickly lead you to realise that you need to use a tape measure that stays the same length at all temperatures.


Harder for all of us, and for both of us writing this book: we knew something was wrong with the economy and business design, but we cannot be sure if even the questions are helpful. For Jack and Graham, given the probably outdated and myopic, disciplined-based lenses we were each using, we both needed the other’s lenses\index{lens}. 


The questions you ask, and how / what answer you hear, are created by the lenses you use. Which means that there are always inherent limits to what you know, and can know. Just as in quantum physics.\index{physics!quantum}




\section{Lenses}
\index{lens|(}


Because the first seven years of my (Graham’s) working career was in theoretical high-temperature particle physics, I have looked at everything I've touched since then through the lenses of a particle physicist. For example, I am always asking myself how the final performance of an entire business can be improved by working on the smallest particles that make up that business and their interactions. 


If you want to dive deeper into why the economy is failing, what it should be, and what’s needed in our toolkit to get there, you have to know which lenses you are using and how they shape what you see. You also need to know which lenses others are using, and how they shape what they see, and you then need to make a judgement call about which lenses you could use. 


By shape I mean that lenses always distort actuality, just as the Mercator projection distorts our spherical planet as it creates a flat map. 


You also need to actively look for what you don’t know you don't know. Which lenses don’t you know exist, and so you don't know what you would see through them? Most of us only use the standard lenses of our discipline. The result is that we can fail to see clearly trends that are driving disruptive change. 


Our current education paradigm focuses too much on simply transmitting the current canonical thinking in each discipline, rather than how to think for yourself, how to ask questions, how to validate your answers against actuality, and how to defend against your own cognitive biases. This leads to those whom Nassim Taleb\cite{taleb-skin}\index{Taleb, Nassim} labels \emph{Intellectual Yet Idiot} or \emph{IYI}. 


One reason why you imagine that a single lens is enough is because you trust your memories and your knowledge. But your mind changes the past you remember, and constantly reshapes what you know, because it fails to remember what actually was\cite{loftus-false-memory-wiki, loftus-false-memory-ted}. You only remember the last time you remembered. So your knowledge, and your memory of the past, constantly changes, being reconstructed by your minds. Like space interacting with an electron, your mind is far from an inert canvas remembering accurately what was; far more it creates your memory\textemdash a reshaped version of what was.


Any lens distorts when it does its job. It needs to hide some things so that you can see others more clearly. The distortion is even greater when you attempt to communicate what you see through your lens to another, who can only interpret through their own lens. So first you have to agree on a lens or lenses, before you can discuss what you see, let alone agree. 


For example, when Picasso\index{Picasso, Pablo} attempted to capture the essence of a horse (seen through his lenses) on a single canvas, few people were able to see what he saw as the essence of horseness. You do your best to represent the essence you see through a concrete medium, as with these concrete words on paper, knowing that different people will interpret a different form. 


So there is an inherent challenge in finding lenses good enough for the actuality you are dealing with. 


If you have spent time in physics and art over the past century, you will know that seeing actuality\index{actuality} in the form of complementary pairs is the starting point of modern physics. Quantum physics depends on using a multiplicity of conflicting lenses to get a handle of complementary pairs\index{complementary pairs}. You see some examples in Table~\ref{table:complementary-pairs}. 




\begin{table}[htbp]
\centering
\begin{tabular}{ c  c  }
\toprule
Tangible & Intangible \\
Observed & Observer \\
Particle & Wave \\
Entity & Interactions \\ 
Position & Momentum \\
Duration & Energy \\
Painter & Viewer \\
Perspective & Complementary perspective(s) \\
\bottomrule
\end{tabular}
\caption[Complementarity in quantum mechanics and cubism]{Inherent, irreducible complementarity in quantum mechanics and cubism. Both columns, and the integration of each pair into a new transcending concept are equally and irreducibly real and needed in order to make sense of what is going on.}
\label{table:complementary-pairs}
\end{table}


One consequence of shifting from “either / or” questions to “and” questions is that you shift to a broader view. Instead of looking for which complementary pair is right and which one is wrong, you look at the pairs as an integration of opposites. 


The observer is central to creating the reality\index{reality} that the observer experiences. So if you are the CEO of your company, and you subscribe to the idea that an organisation\index{organisation} is a machine, when you look at your organisation you will see a machine, and build a machine. Equally, if you look at your organisation as a complex adaptive system, you will build a complex adaptive system. All the books by a pioneer in applying these concepts to leadership and business, Margaret Wheatley\index{Wheatley, Margaret}, are well worth reading\cite{wheatley-leadership,wheatley-lost-found}.


However, two more factors are important in determining the results that your organisation actually delivers. First, you are not the only observer creating the reality of your organisation. Every person who touches your organisation is an observer, and shapes its reality across whatever scale they touch. 


Second, you can only shape the reality of your business within the boundaries that your business offers you. If your business is more than just a machine, or more than just a complex adaptive system, you will end up with something that has a hidden nature that you are blind to. Your business will deliver results that you did not expect, do not see where they came from, and cannot reliably reproduce.


As you read in Chapter~\ref{chapter:reasons-for-hope}, when physicists recognised that they were seeing unexpected results, and could neither see where they had come from, nor reliably reproduce them, they realised that the fundamental paradigms of classical mechanics (nature as a machine) were simply not fit for purpose in the realm of the very small and the very large.


And so quantum physics\index{physics!quantum} was born.


Physicists\index{physicists} had to abandon the simplistic notion that particles, waves, and the forces between them were in any sense independent building blocks of reality. They had to become comfortable with the inescapable fact that all particles were waves, and all waves were particles. There was no way of disentangling the two. Everything was inherently both.


This led to their realisation that thinking of particles and their interactions as two distinct things was equally flawed\textemdash instead, particles and their interactions were, in some deeper sense, one and the same thing. This extended concepts prevalent in many philosophies and religions around the world\cite{capra-tao} to fundamental physics, bringing a new depth of useful understanding and precision to what physics can say about actuality.


In classical physics\index{physics!classical}, the particle is primary, and it creates its interactions with its neighbouring particles. Quantum physics\index{physics!quantum} turns that on its head. In many cases it was a more helpful starting point to regard the interactions as primary, where the interactions themselves pulled the particles into existence. 


As you will see in this book, you will get a far better handle on how to deliver the results your organisation\index{organisation} expects from you, and that you are accountable for getting your organisation to deliver, by seeing both the interactions between the people in your organisation and the people in your organisation as equally foundational and independent. The interactions between people create the nature of how each person shows up, at least as much as the individuals and their nature create the interactions between them. Given the nature of the challenges we face today, we find it helpful to regard the interactions as having more power to shape the people than vice versa.


\begin{table}[htbp]
\centering
\begin{tabular}{ c  c  }
\toprule
\textbf{Particles} & \textbf{Interaction} \\
        \midrule
Investor & Votes and dialogue \\
Staff & Dialogue, positional power, strikes \\
External stakeholders & Dialogue, law, money, reputation \\ 
Organisation as a machine & Organisation as a living being \\
Rules and policy & Culture \\
\bottomrule
\end{tabular}
\caption[Complementarity in business]{Inherent, irreducible complementarity in business. }
\label{table:complementary-business-pairs}
\end{table}


What would we get if we applied this thinking to your business? If the interaction has characteristics that are predefined, and that call into existence the nature of the particles that interact, what are these pairs? How does the predefined type of interaction create the very reality that the particles experience?


Investing shareholders\index{shareholders} interact in a very narrow and defined way. They typically interact within the constraints of a general meeting, often through the medium of exercising a yes / no vote. Sometimes it’s done through dialogue in pairs, or small groups. More often it’s done simply by listening to whoever is addressing the shareholders\index{shareholders}, one to many.


Staff interact mainly through dialogue\index{dialogue} and hierarchical layers. The patterns of hierarchical power and many the patterns of dialogue are predefined and exist independently of any individual member of staff. This is similar for all external stakeholders, as summarised in Table~\ref{table:complementary-business-pairs}. 


So if you look at an organisation\index{organisation} as a classical physicist would, as if particles were primary and created the interactions, you will get surprising results, would not see where they came from, and would not know how to reliably reproduce the desired and suppress the undesired results. 


One final thought on complementary pairs\index{complementary pairs}. All three perspectives (the duality of each components' uniqueness and difference; and their non-dual oneness) are an inherent and necessary part of how the world works. If you only had the pair’s non-duality as one, you lose all the power that comes from each in their uniqueness, and if you only have each as an either / or uniqueness, you lose everything that you get with the duality.\index{lens|)}


\section{Maps}
\index{maps|(}
Your maps go together with your lenses. You use a map to simplify actuality so that you can get from A to B without overwhelming yourself with irrelevant detail. Just as lenses hide and distort most of actuality, so too do maps. Choose a map that hides what is irrelevant, so that what is relevant can be seen clearly, and you can navigate well to your desired destination. Choose a map that hides or distorts what is relevant, and you will get lost.


A superb example is the map of the London tube. Early on, it showed each station in a faithful representation of its position on the surface of London\index{London}. After a while, with lots of people struggling to use those maps, it became clear to Harry Beck\index{Beck, Harry} in 1933 that knowing exactly where each tube station was, was irrelevant information. What was relevant was being able to clearly see the connectivity between lines and stations in order to figure out which combination of lines connected where you were to where you needed to be.


This map of connectivity is now standard around the world for many transport maps. Like many disruptive innovations that challenged the lenses\index{lens} and maps of the era, Harry's connectivity map was regarded by the authorities of the time as way too radical. But because normal people found it enormously useful, it took hold and became the global standard.


Maps that work in a predictive context fail miserably in an emergent context, because they hide the very information you need to thrive with emergence. An emergent map is like a probability field in physics, or a painting by Picasso\index{Picasso, Pablo}. It tells you about all the categories of outcomes; it describes different scenarios, without attempting to pin down any details. The map itself changes depending on who is using it; it's not absolute in itself, you shape it because it's you using it, rather than somebody else.


Such emergent maps were used in South Africa\index{South Africa} to navigate the country out of apartheid. They were used at each decision to inform a judgement call. Everyone knew that no choice could ever be right, since the transition was emergent, not predictive\textemdash right could only be known long afterwards. But an emergent map is still useful, because it gives guidance on which choices open up more potential for emergence, and therefore options for thriving; and which choices shut them down.


One good example of how long it might take to know if an outcome was good or bad is illustrated by the quote from the then Chinese president Zhou Enlai\index{Enlai, Zhou}. His words in 1972 \emph{“it is too early to tell”}, during the visit of Nixon to China, are usually believed to refer to the French Revolution that started in 1789, and fit perfectly to it. However, he was referring to the student revolution of 1968, only four years previously.


You use mental maps, constructed in your past, to help you understand what you are seeing now and to make choices. No single map can be absolutely right in all contexts, at all times, everywhere; nor is there a privileged frame of reference. 


Most intractable arguments in families, business, and politics happen when you fail to realise that each of you is using different maps and lenses. For example, those voting in the Brexit referendum for the UK to leave the EU thought that they had made the best decision, based on their lenses and maps; while those voting to remain thought that they were making the best decision. No discussion on whether Brexit is a wise or unwise choice can happen without first comparing maps with each other, and discussing what the maps are saying, what they are highlighting and what they are hiding, and getting to shared clarity on what is actually relevant to the binary choice of in or out. 


That map check made a huge difference in the path South Africa followed into its post-apartheid future.


You can only make progress if you see clearly which maps and lenses each person is using, and then compare them, knowing that there is no privileged perspective or frame of reference. How relevant is each? What is missing? What is useful?


You cannot not use maps and lenses. The best you can do is recognise which ones you are using, and shift the dialogue, or argument, to focus on them all, not just the consequences of a specific map or lens. 


The lens is a tool to generate an inner perspective that is useful. After you've looked at something through a lens, and perhaps compared it to one of your internal maps, maybe by evaluating what you are seeing as good or bad, helpful or harmful, or somewhere in between, you end up having a perspective.


Using multiple lenses\index{lens!multiple} and multiple maps enables you to have multiple perspectives. In today’s nebulous, emergent reality, you can often only grasp what is actually happening and make wise choices if you use multiple perspectives\textemdash even if those perspectives seem to conflict or even exclude each other. 


This was the big lesson that physicists and artists needed to learn a century ago: contradictory perspectives might all be needed to grasp what is actually happening. Which is why mining and refining conflict is such a vital capacity today. I (Graham) experienced this travelling through the Alps to an internal meeting, which gave me more capacity to mine and refine the conflicts we had in the meeting.


\begin{longstoryblock}
I imagine you've been told, as often as I have, to get a perspective. A year ago I was on my way to an internal Evolutesix\index{Evolutesix} off-site strategy meeting, and I had my suspicion that during the two-week period we would experience internal conflict. In Evolutesix we look at conflict through the lens of developmental value. We see conflict\index{conflict} as the most valuable resource that an individual or organisation has to understand where meaning-making stories fit their context and needs.


Using the best dialogue\index{dialogue} tools available to mine and refine this conflict, we can get at the gold nuggets we need to adapt ourselves and our organisations to stay at the maximum fit between who we are and the challenges we are facing. For this to work, though, we need to have the conflict in perspective. This means that we have the most objective yardstick we can find to compare the conflict to, and decide whether the conflict is small, medium, or big.


So I decided to travel down to the meeting over a couple of days by motorbike through the Austrian Alps. I knew that spending a few days in the middle of nature, surrounded by some of Europe’s highest mountains, would make sure that my frame of reference lay in mountains that stayed unchanged across generations.


I got even more perspective than I had bargained for. First, early in the morning on the first day of my journey, the front bearings of my motorbike collapsed. I managed to coax the bike to the next village, where I stopped and waited for rescue. That evening I was back in Brussels, figuring out what to do. I also realised how extremely lucky I was. The bearings could easily have gone a thousand kilometres later in the middle of an Alpine pass, steeply banked over, and under heavy braking because of a rock fall on the exit to a hairpin bend. Instead, it happened whilst I was riding slowly enjoying the scenery approaching the Luxemburg border.


The second perspective was deciding anyway to travel by car, and simply do  in one day what I had planned to do over two days. A very wise decision, because the second unexpected perspective that I got was visiting a glacier in the Alps. Driving up the valley I passed a sign in the middle of verdant vegetation marking where the glacier's snout had been when the valley was first explored in the mid-1800s. I continued to drive up the valley, going up hundreds of metres and about 20km further along before I reached the current edge of the glacier. 


Seeing myself as incredibly lucky, looking at how the Alps in total have barely changed across hundreds of human generations, and yet how rapidly climate change is melting the Alpine ice, meant that I compared whatever conflict I was experiencing during our meeting to the frames of reference of these two perspectives. That helped me see that whatever was happening in the moment in that meeting, it was insignificant compared to our environmental crises. 
\end{longstoryblock}\index{maps|)}


\section{Meaning-making stories} 
\label{section:stories}\index{meaning-making stories|(}
Pablo Picasso\index{Picasso, Pablo} recognised the power that stories have in shaping or even creating the reality that you experience. His whole life revolved around the power of his stories, and harnessing the power of stories to make a difference in the world.


Picasso rejected using his father's name, and instead took his mother's. He realised early on that the name he had been using carried with it the story his father lived by. He rejected his father's story of not ruffling anybody's feathers, as he saw that this story had shaped his father's reality as a mediocre artist. Picasso wanted to stand out, to make a difference, and his story centred on taking risks to do so. 


Throughout this book, you will gain understanding of the central role stories\index{stories} play. 


I don't mean trivial stories, the kind of detective or fantasy story that you read for entertainment. What I mean are the deep stories that you use to work out what everything means. In the example above, Picasso's father shaped his life around a story of harm if he stood out, if he ruffled anybody's feathers. Picasso shaped his reality with a very different story: that if he failed to ruffle feathers he would fail to make a difference in the world, then he would fail to reach the pinnacle of art that was in him, and that meant he would be a failure.


Meaning-making stories, which we introduced in the last chapter, are distinct to stories. Most of your meaning-making stories are so deeply hidden that you are not even aware that they're there. You are still using them, though, to give meaning to yourself, to others, and to what you experience.


\begin{longstoryblock}
I (Graham) was out hiking alone, taking a break from writing this book, when a slippery rock took me down crossing a stream about a kilometre from Granny Dot’s\index{Granny!Dot’s}\index{South Africa!Granny Dot’s} (where we were staying). I crashed onto my right hip, twisting it and bruising my hip muscles. Getting home was painful, and it took well over two weeks before I could even get dressed without sitting down or supporting myself, let alone continuing the long, productive walks with Jack.


One of my meaning-making stories, creating the reality I experienced, was how unlucky I was to have fallen, anger at my carelessness and sad that I could no longer go for a strenuous walk every day to clear my brain and write more creatively and effectively. When that meaning-making story was active and telling me what falling meant, I felt sad, less motivated, and was less able to write well. More than the pain of sitting brought physically.


Another meaning-making story active in me was just how lucky a person I am. Even though I'm in my mid-50s writing this book, I'm still physically fit, and I'm still able to hike long distances, compared to some of my schoolmates who have already passed away or lost some of their fitness. I could so easily have broken a leg, or done myself a permanent damage. Enormously lucky.


The third meaning-making story active in me was that this fall had come about because I had been paying too much attention to meaning-making stories around what other people might think if I didn't get back by the time I had said I would. So I was hurrying, paying less attention than I normally do, in part because I'd already stepped on this rock going the other way 90 minutes previously. This meta-story I was using to chip away at some of these older stories that have been running my life, telling me what I should do, and what it means if I fail to do so.
\end{longstoryblock}


Meaning-making stories are quite different to stories of something. I can tell you a story, and the story I tell is all that it is at the time. Meaning-making stories are the hidden templates you use to create the reality you experience, by giving meaning to the small part of actuality you are aware of. They are deeply hidden; you are unaware of most of them. So you are subject to them, almost a victim of them, blindly accepting them as actuality because you have no idea they are creating your experienced reality.


The purpose of meaning-making stories is to generate your experienced reality out of an actuality that can have inherent meaning, or not.


Your meaning-making stories can be anywhere from a pure fiction, leading to your experienced reality being castles in the sky, or just shapes you see in the clouds, through to 100\% anchored in actuality, leading to your experienced reality being identical to actuality. This book shows you the best approaches we know of to become steadily more aware of where and when the meaning-making stories creating your experienced reality have no grounding in actuality, and how to adjust them so that they become ever more grounded in it. Less fantasy and more real. Actuality\index{actuality} always has far more potential than the reality\index{reality} you experience.


\begin{longstoryblock}
I'm sitting on the edge of the tree-covered valley, with a constant stream of white butterflies fluttering from right to left across the entire 50km depth between me and the distant, hazy blue mountains. 


I think, what if my meaning-making story creating my reality tells me I'm a butterfly? Is that helpful or harmful? 


It is entirely dependent on the actuality I'm embedded in, and what I do with it. If I am on the edge of a mountain, and leap off expecting to be able to flutter across the valley to the other side, the gap between my meaning-making story and actuality will lead to a painful fall, broken bones, and possibly death. 


However, for Muhammed Ali, the meaning-making story of floating like a butterfly and stinging like a bee is grounded in actuality and leads to success.
\end{longstoryblock}


Meaning-making stories are the central theme of this book because they shape all scales, from the global economy to the smallest aspects of your life. And so are the biggest barriers we all face in rising to the adaptive challenges in our lives. They’re the biggest barrier we all face in rising to the adaptive challenges of our climate emergency and the 17 SDGs\index{Sustainable Development Goals, UN 17}. If your stories don't have any place in them for your challenges, you cannot accept them. If your stories lack any place in them for a decision or an action, you will never make that decision, nor take that action. And if that decision or action happens to be essential for you to thrive, your story is a threat to you. 


This book is about recognising your stories, and mastering the skills of enabling your stories to rewrite themselves. You cannot rewrite them directly by using thought alone; they have a life of their own, just as in a book I (Graham) love, the \emph{Never Ending Story}\cite{ende-never} by Michael Ende\index{Ende, Michael}. 


But what you can do is deliberately create experiences for yourself that trigger your stories to rewrite themselves.\index{meaning-making stories|)}


\section{Reality vs. Actuality}
\label{section:reality-meaningness}\index{reality|(}


Quantum physics\index{physics!quantum} is clear: when you look at an electron, what you \emph{see} and can \emph{say} about it is not the same as what it \emph{is}. What you see and can say is reality; what it is, is actuality\index{actuality}.


The reality you experience is the model of actuality you create inside your awareness. You never experience what \emph{is}, only your inner experienced \emph{story} about what is. There are a number of aspects of how you construct your inner model which mean that everything you experience is a distorted part of everything that actually is. The better you know your meaning-making story\index{meaning-making stories} \emph{about} what is, the closer you can bring your internally experienced reality to what actually \emph{is}, by iteratively adjusting your meaning-making. 


Part of the gap between your inner experienced reality, and what actually is, lies in your neurobiology. The physical lenses you use, your eyes, ears, touch, smell, and taste all amplify a small fraction of everything around you, taking in nothing of the rest. If you had the eyes of a cat, you would take in a very different slice of actuality, and even if the rest of you were the same, you would construct a very different inner reality.


For example, you are physically blind 15\% of the day. (This is also a metaphor for the intangible themes of this book.) When you're driving, crossing the road, or doing anything else that may have very fast potential surprises, 15\% of the time your eye is not transmitting anything into your brain. Your eye constantly jumps between focusing on one thing to focusing on another. Between every jump, your eye shuts down so that you are not permanently nauseous the way you are if you watch a flickering movie for too long. But anything that happens during those jumps you physically cannot see.


In addition, the way that your vision works is not at all like a camera. Your eye does not simply take in a constant stream of images and pass them on to the brain for processing. Your eye receives from the brain the hypothesis image that the brain expects to see based on its current model, i.e. your current inner reality. All your brain gets from the eye is the small amount of data telling it where its model differs from the data that the eye is actually receiving. So all you are getting is information that inner reality is different to external actuality.


Then, just to make things a bit worse, you are actually living in the past. Between all your different senses, including ones that may come from your intuition, it takes your brain about half a second to process all the information and update your inner reality model. So by the time that you are fully conscious of your constructed reality, it's half a second out of date.


If something happens that threatens you in less than half a second, say a very loud and unexpected noise, your subconscious brain gets you moving a few tenths of a second before you are aware that you are running. Of course, if you're on the starting blocks of a sprint, your brain has already included a starting pistol and that you will be running in its future projection of your inner reality. So everything gets smoothed out and you experience a continuously unfolding reality.


The reality you experience is an inner model, constantly updated based on the input from your senses, but this input is already distorted and biased by the previous iteration of your inner model. It's never completely unbiased, unfiltered raw data.


Be on your guard against the charismatic or glamorous playing games with your inner reality to benefit themselves. Especially anyone undermining your awareness of the power you have. For yourself, and those you care about, certainly; and because we need all of our power to rebuild a better world. Use this book, especially the chapters of Part~\ref{part:you}, to resiliently anchor your inner reality in what actually is. 


The character Magrat Garlick, in \emph{Lord and Ladies}\cite{pratchett-lords} puts it well when she describes the inner reality people construct when elves (the Lords and Ladies) project their desired reality and glamour into people: \begin{quote} We’ll never be as free as them, as beautiful as them, as clever as them, as light as them; we are animals \ldots What they take is everything \ldots What they give is fear.  \end{quote} 


As Granny Weatherwax\index{Granny!Weatherwax} puts it in the same book, even though we are strong and they are weak, what they project makes us believe the opposite.


Your reality is completely inside you. But, you certainly do not have the freedom to experience any reality. You cannot simply make it what you wanted to be. Not even in your dreams, weird because your brain is constantly constructing an inner reality that it can believe in, but in the absence of any actual data from your senses to ground and calibrated. Even then, you can only dream based on the possibilities that your brain is equipped for, by neurobiology and your past experiences.


Before you are born, your neurophysiology is pretty much unformed potential, out of the differentiation that your genes have brought in. Then your senses and your brain begin to change based on the reality you experience, which depends on the meaning-making stories of your parents and anyone else interacting with you in your early childhood. These stories shape the patterns that you look for when you're looking at something, shaping which sounds you can hear clearly and speak clearly, and much more.


Hopefully from this you can see why this book is so focused on stories. The only reality you can ever experience is your inner story now. Your inner reality is shaped by your entire history of experienced inner reality, not just in terms of the distilled memories that you use to attribute the meaning and shape your inner reality, but also in the very neurobiological hardware that you are using. Sadly, your inner reality cannot be anything you want. You are constrained to experience an inner reality within boundaries.


So if anybody tries to criticise you and your grasp of reality, all they're really saying is that \emph{your} grasp of \emph{their} inner reality is weak. Well, that's cool, there's no way you ever could truly grasp anyone else's inner reality. In fact, it has become quite clear that you experience your own, unique, individual reality because the combined effect of your brain, your senses, your nervous system, your forms of thought, your meaning-making stories, your entire life path to now, expectations of the future, and much more, is as unique to you as your fingerprint is\cite{sci-am-reality}. 


Look at the meaning of what happens in your life as a complementary pair of any inherent meaning and the meaning made by your meaning-making stories. Sometimes it's purely one, at other times purely the other, and then again a mix of both together. This brings together the two mutually exclusive worldviews\cite{chapman-meaningness-eternalism-nihilism}. 


\begin{description}
\item[Eternalism:] everything that is and happens has intrinsic meaning, determined by some divine plan or cosmic inevitability.
\item[Nihilism:] nothing that is or happens has any intrinsic meaning, all meaning is created by ourselves.
\end{description}


Bring them together into one complementary pair\index{complementary pairs} named meaningness, and the mutual validity of both sets of worldviews becomes visible.\index{reality|)}
\section{Your chimp, human and computer systems}
Steve Peters\index{Peters, Steve} in his book \emph{The Chimp Paradox}\cite{peters-chimp} describes your mind as having three systems: chimp, human, and computer. 


\begin{description}
\item[Chimp.] Your inner chimp system always wants more, takes in a limited slice of the world, and interprets it through the reactive meaning-making stories it began learning in the first couple of years of your life. Your inner chimp then tries to take control of you through feeding your emotional stories through to your computer. It's also called your limbic brain.


\item[Human.] Your human system only begins after the age of two, when your hardware has developed enough. Your human system tries to act rationally, look at the facts, and makes meaning of them through stories you have internalised. 


\item[Computer.] The computer system is your library of stories and behaviours. Both the chimp and human systems consult the library, pull out the stories and behaviours they each think is best for you, and then fight with each other to drive your behaviours.
\end{description}


The rest of this book is all about getting more books and behaviours into your library, your organisation's library, and our economy's library.


When we do this well, we have a huge edge over all other species in creating an environment in which each of us can thrive, because it gives us the ability to predict what will happen decades into the future, and prepare for it. It gives us the ability to prepare today for the drought that is likely to come sometime in the next 10 years.


However, prediction alone is not enough. We need to find ways of bringing all our human systems together so that we can act on our predictions. That our human activities were certain to generate climate change, and a climate emergency\index{climate!emergency} if we failed to change our behaviour in time, was already clear by the 1980s.


If you listen carefully to what someone is saying, you can sometimes hear the different systems showing up in and between their words. I had the privilege of sitting next to, and chatting with, a wise South African, Sheila Sisulu\index{Sisulu, Sheila}, former ambassador to the US, in London\index{London} in 2009 at a meeting preparing perspectives for the G20. We started talking about how she was listening to the politician currently speaking, and advised me: \emph{“Listen to the unspoken subtext. Between the words used, he’s saying ‘bla-bla-bla we don’t understand what’s happening, we don’t know what to do.’”}


I listened to the subtexts, heard a very different message to the one in the words, and have never stopped listening for what is really being said. 
\section{Cargo Cults}
\label{section:cargo-cults}\index{cargo cults|(}
During the Second World War, the military built airfields on a number of the South Sea Islands. Streams of aeroplanes flew in with their holds full of useful goodies, flew out again empty, and returned with more cargo. After the war ended, most of these airfields were dismantled completely. The equipment was removed, the personnel left, and the aeroplanes full of useful goodies stopped coming.


The islanders missed their share of the useful goodies. They asked themselves what they needed to do or be for the aeroplanes to come back with their holds full of them. So they repaired what was left of the airfields, or cleaned out long stretches of land to look like them. They lit fires along the airfields, built wooden huts that looked like control towers, and had someone sit in the wooden hut wearing a shaped piece of wood on either side of his head with sticks of bamboo poking out the top like a headset.


It looked to them as if they were doing everything that the military had been doing. Whilst they were doing the visible rituals, they had none of the hidden engines, and none of the underlying worldview and paradigms. They could not know that these hidden engines mattered, not the irrelevant visible rituals.


The physicist Richard Feynman\index{Feynman, Richard} used this \emph{Cargo Cult} as a story~\cite{feynman-cargo-cult,meaningness-cargo-cult} to show the difference between real science and scientism. In scientism people do the visible behaviours of a scientist, but without the underlying engine that makes science work. This engine is: using intuition very strongly to imagine how the world might work, and what you might be able to say about it; and then brutally doing all you can to rip your ideas and everybody else's ideas to pieces.


Ideas that survive long enough become evidence-based theories, or clarity on what we can, with confidence, say about how the world works.


Too much of what you will find in today's self-help, business studies, economics, and much more is scientism. You can read more about this in the excellent book \emph{Overshoot}\cite{catton-overshoot}. Protect yourself against the disasters that follow the fake hope cargo cults lead to. Stay sceptical. Stay empirical.


There is one very good way to tell whether what you are seeing and believing is coming purely out of an internally generated reality anchored in your biases (Section~\ref{section:biases}) and other meaning-making stories, or if it has a sufficient basis in how the world actually works. 


This is the brutal process that scientists have developed to counteract, as best as is humanly possible, situations where our self-deception tendency might lead us to build bridges that collapse in the first strong wind. Like the bridge built across the Tacoma Narrows in Washington, where the engineers failed to take into account that the wind would play it like a guitar string until it vibrated itself to pieces.


Empiricism, or the scientific process, is far more than just collecting evidence, it is the brutal adherence to three key elements.


\begin{enumerate}
\item Letting go of what you believe you know to be true, so that you can follow all kinds of crazy ideas to their conclusion. Like Einstein\index{Einstein, Albert} letting go of what everyone knew to be true: gravity is a force; and then seeing where that took him.
\item Actively disconnecting your self-worth from old meaning-making stories and your biases. From being seen as right, or a need to be seen to follow the norms. Even the most brilliant scientists struggle at times with this one, which also requires rigorous adherence to the third principle.
\item  Always focus on finding evidence to prove something false, never look for evidence to prove something true. Because you never can prove something in actuality to be true. The best you can ever do is fail to find instances where you can prove it false. Everything that has risen in status from model to theory is no more than knowledge under construction, never complete (thought form P5 in Chapter~\ref{chapter:who-am-i-sense}).
\end{enumerate}


If you have ever worked in fundamental science and attended a conference between practising scientists, you will have seen how extraordinarily brutal they are to each other when talking about their research. However, they are also enormously kind to each other as human beings. This is how scientists harness conflict to uncover what they can, and cannot, say about deeply hidden aspects of actuality.


Brutal, because everyone wants to prove the other's statements false. But not the other person. We need far more of the above three principles across all our disciplines, including economics. Our human survival depends on us going beyond scientism.\index{cargo cults|)}
\section{Harnessing, not managing, conflict}
\label{section:harnessing-conflict-intro}
How does your meaning-making, which is created by your meaning-making stories, generate your experience of conflict? Do you then end up feeling anxious, leading you to want to reduce or escape conflict? Or do you feel excited, and want to lean deeper into the conflict?


I (Jack and Graham) developed conflict-avoiding meaning-making stories\index{meaning-making stories} during the early part of my life, and they are still active in me. I have learnt, though, that conflict is very often simply hard data on a mismatch between my inner reality\index{reality} and actuality\index{actuality}, or that someone else has different meaning-making to me, generating a different reality to mine from the same actuality. 


In both cases, harnessing that conflict, mining and refining it for the valuable gold it brings on what actually is, is vital for being more effective. I (Jack and Graham) have found our conflict invaluable in creating this book, even when it has been painful in the moment.


If each of us are to rise to the adaptive challenges we are facing, we must get steadily better every day at harnessing, not managing, conflict. The essence of this book is a how-to manual of the state-of-the-art techniques to mining and refining conflict across all six layers of the hierarchy listed below. If you have room to improve in harnessing conflict for growth, this book is for you.


Each layer requires the layers below it to already be working well, in order to reach full effectiveness in a mature stage; and even more so on the change journey towards mature harnessing of conflict or tension. Here is our proposed current best approach in each of these layers:


\begin{description}                                                                                                                                             
\item[6 Inter-ecosystem] All the layers below, plus the full emergent aspects of open systems in interaction with each other;
\item[5 Inner-ecosystem] e.g., The FairShares Commons or similar applied to the whole ecosystem and all companies within it, (see Parts~\ref{part:economy} and~\ref{part:organisations});
\item[4 Inter-organisational] The FairShares Commons or similar incorporation for almost all kinds of entities, except perhaps a pure trust, or a wholly owned subsidiary of a FairShares Commons;
\item[3 Inner-organisational] Extensions of sociocracy, Holacracy, and other forms of dynamic organisation design or governance (see Part~\ref{part:organisations});
\item[2 Inter-personal] Interactions between people, culture, etc. all belongs here. The Evolutesix Adaptive Way for teams, covered in Part~\ref{part:organisations}, does this well;
\item[1 Inner-personal] Who I am, my inner voices and motivations, meaning-making etc. The Evolutesix Adaptive Way for individuals, in Part~\ref{part:you}, does this well.
\label{list:six-layers}
\end{description}


Each successive layer encompasses the full complexity of the layers below, and adds more of its own. For example, think of yourself. At layer 1, you have all the complexity of who you are, all the different voices telling you who to be and what to do for whom. At layer 2, with just two people, you have all that for two people and your relationship. And so on up to the level 6 level, the interaction between huge ecosystems that contain thousands of people and hundreds of businesses.


But, no level is more important than any other level; and actually level 1, who you are, was shaped by the level 6 you and your ancestors have been in. So really it’s also a circle \ldots 


Effectiveness at each level depends on how well generative conflict is harnessed. The Evolutesix \index{Evolutesix} Adaptive Way\index{Adaptive Way} is designed to harness conflict, to the extent that we even have a pattern called Tiggering\footnote{I (Graham) even have a sticker of Tigger, from Winnie the Pooh, on my computer as a reminder.}, which is about deliberately creating conflict when there is too little. Of course, this is always done with mutual compassion for the other and only by invitation!


Harnessing conflict is critical in business, especially startups. 65\% of startup failures are caused by conflict or tensions between the co-founders\cite{wasserman-founders-dilemmas}. We are in the mess we’re in because we humans are so poor at harnessing conflict generatively. Only managing or avoiding conflict limits our ability to grow and transform the thinking we have today, to get what we need to thrive tomorrow.


If we can mine and refine the critical conflicts today between different disciplines, or paradigms in each discipline, we have a chance of escaping Planck's Principle. Max Planck\index{Planck, Max}, a pioneer in quantum physics\index{physics!quantum}, wrote \emph{A great scientific truth does not triumph by convincing its opponents and making them see the light, but rather because its opponents eventually die, and a new generation grows up familiar with it}.






\section{Boundaries}
\index{boundaries|(}
Boundaries are very relevant to all the sections of this book. We need boundaries to have clarity on what is unique about each individual entity, from the planet’s boundaries making Earth the uniquely life-supporting planet that it is, down to the cell walls defining the unique identity of each of your cells.


If your cells did not have a clear wall defining what is inside and what is outside the cell, you could not be a functional human being; instead, you would still be a large puddle of primordial soup. However, if your cell walls were 100\% solid boundaries, you would be a completely inert dead statue. What makes you a thriving, living organism, is the dynamic interplay between what is inside and outside semipermeable, fuzzy boundaries, at all scales.


We need to construct boundaries in society. Each of us needs to have a self-identity that is uniquely ours. As you will read in Chapter~\ref{chapter:who-am-i-meaning}, one part of us developing our self-identity is the stage where we construct it by internalising the norms and culture of the group we belong to. At the next scale up, we need to have clearly defined social groups. Social groups emerge when a number of people declare some difference to another group to be the defining characteristic separating one group from the other. This is the basis of most of our isms.


This is also the basis of how we incorporate businesses. Keep in mind that incorporation is a social construct that we have invented to define boundaries between stakeholders. The boundaries between stakeholders defined by a limited company, or a trust, or a cooperative, create different kinds of unique business identity.


Similarly, the boundaries between different disciplines like economics, leadership, and engineering, and the boundaries between different factions within disciplines, are vital for each discipline to develop the unique strength that that discipline can bring to us.


However, every boundary reaches a point where it holds us back and where it's time to recognise that the boundary is semipermeable, or even an illusion that we have constructed to simplify our lives.


Physicists\index{physicists} recognised that this is how the world works at a quantum level when they realised that the boundary between the concept of particle and wave was one that nature recognised in some contexts, and was absent in others. That led to seeing particles and waves as complementary pairs which had both an individual uniqueness and a common oneness.


This is also behind the new FairShares Commons\index{FairShares Commons} incorporation, which represents the common oneness across all stakeholders, whilst retaining just enough of the uniqueness that differentiates stakeholders and each company from other companies in the ecosystem.


As you read this book, keep an eye open for where boundaries in your reality are holding you, your business, and our society back.\index{boundaries|)}






\section{How did we get into this mess?}
\label{section:into-mess}


Napoleon\index{Bonaparte, Napoleon} is often reputed to have said that he would rather promote somebody who was naturally lucky; and never to attribute to enemy action anything that he could explain with human incompetence.


Whilst it's only natural to want to blame and punish other people for being evil or criminal, especially the more you feel pain, anxiety, anger, or any other strong emotion, I believe that few of the people who played a role getting us into this mess were inherently bad.


Almost all were not much different to you or me, simply doing what was in front of them to do, without sufficient understanding of the long-term implications of what they were doing, or how related any single action was to everything else that was happening. Simply a collective incompetence that required inhuman superpowers to see clearly in advance.


Add the inevitable random bad luck that always happens, and you have a recipe for the mess we are in.


Getting out of this mess means avoiding repeating the same kind of collective incompetence and fragility in the face of bad luck. One foundation of such fragile sensitivity to bad luck and incompetence lies in focusing on blaming others for what they did in the past, instead of focusing on what we can do now from where we are to build a future we want to have.