%quite a few areas modified here too.
\chapter{Growing regenerative businesses and ecosystems}
\addcontentsline{toc}{chapterdescription}{We can get out of this mess by creating and scaling regenerative businesses and ecosystems based on Level 5 incorporations like the FairShares Commons. Businesses should be incorporated for a finite, rather than an infinite, lifetime, and they need far more power with and power to, rather than today’s ineffectual power over.}
\addcontentsline{toc}{chapterdescription}{\pagebreak}
\label{chapter:growing-regenerative-organisations}




\begin{chapterquotation}
No gluing together of partial studies of a complex nonlinear system can give a good idea of the behavior of the whole. \\
\raggedleft\textemdash Murray Gell-Mann
\end{chapterquotation} 


Each chapter of this book uses one, or at most a few lenses, so each is a partial study of a complex, nonlinear, open system. In this chapter, I zoom out and as best as possible give an idea of the behaviour of the whole, and how to glue some of the lenses together\textemdash as best possible, because of the inherent impossibility that Murray Gell-Mann \index{Gell-Man, Murray}refers to in the quote above. 


Successful businesses in the coming decades will only thrive because they harness the nebulous, continuously changing, non-ergodic\index{ergodic!non}\index{non-ergodic} nature of our world; so they must be designed for nebulous, continuous change. They must be fully Adaptive Organisations, collaborating and competing with other Adaptive Organisations in deep, regenerative ecosystems of businesses.\index{Adaptive Organisation} They will be beyond resilient\textemdash Adaptive Organisations must be antifragile\cite{taleb-antifragile} because they collaborate in our non-ergodic (Section~\ref{section:non-ergodic}) world.\index{antifragile}


They will use human energy and power in a healthy way. Power\index{power} is just energy used to move things. Power applied to Job~1 delivers useful results for the organisation; power into Job~2 does not. In the knowledge economy of today, power \emph{with} is the most valuable for Job~1 and delivering results. Power \emph{over} is usually a cost because far too much of it is harmfully flowing in Job~2. 


To maximise the productive output of a business you need the right balance of power with, to, and over. So you need the Adaptive Organisation structures of this book across all four strata of Section~\ref{section:harnessing-conflict-intro}. Have too much power over, and you have a wasteful, inefficient operation. Which is why the FairShares Commons \index{FairShares Commons} is better than a standard incorporation for any investor wanting to maximise their financial and impact returns today. 


The leaders and all key players will be masters of all 28 thought forms, including the relationship and transformation thought forms needed to optimise for the connections in collaborative business ecosystems, highly competent at maintaining the equilibrium of their emotional state, and most of all adept at continuously adapting their meaning\hyp{}making stories. They will be masters of the best approaches in all six layers listed on Page~\pageref{list:six-layers}, and how to integrate them. 


It is time to shift our economy\cite{dekemmeter-shifting-economy}, because it will crash again\cite{guardian-groundhog-day} as we repeat the cycles we have designed into it. Every crash risks taking us closer to runaway climate change, and all the other SDGs \index{Sustainable Development Goals, UN 17} becoming full emergencies too.


\begin{longstoryblock}
I (Graham) loved the garden route between Cape Town, where I went to university, and East London, where I grew up. \index{South Africa!Cape Town} Some very farsighted individuals many generations ago recognised that the original national forests of South Africa were rapidly disappearing. They conserved sufficiently large pockets for these ecosystems to survive, albeit without most of the larger fauna. Even so, these still are regenerative ecosystems. Leave them alone, and over time they will regenerate.
\end{longstoryblock}


No living organism in nature can survive, let alone thrive, in isolation. All living organisms thrive in a chaotic web of interactivity. Even you are an ecosystem: only 43\% of your cells are human, the rest are symbiotic bacteria and other microorganisms. All of them are multiple related interacting open systems, embedded in larger ones, containing smaller ones, and inter-embedded in peer systems. 


All this is what has made life on earth antifragile from the start.


So how on earth could anyone imagine that we can build a viable economy out of individual cells (businesses) that are artificially constrained to be independent of each other, given the non-ergodic world we live in (Recall Figure~\ref{fig:bet-path-1p-sharing})? 


Imagine how much better we could do to tackle our climate crisis. The difference between the averages in Figures~\ref{fig:bet-path-1c-cutoff} and~\ref{fig:bet-path-1p-sharing} suggests that we could gain thousands to millions better outcomes for the same time, money, and effort. Simply by incorporating at Level~5 for collaborative-competitive deep ecosystems that harness the nebulous, symbiotic, chaotic web of interrelatedness between businesses, society, and the public sector to counter our non-ergodic, luck-dominated world.


So now you know where we need to end up, to get out of this mess that we have got ourselves into ( Section~\ref{section:into-mess}, P~\pageref{section:into-mess}). We need regenerative economic ecosystems, built of inter-related regenerative businesses that are themselves irreducibly nebulous, open, constantly transforming living beings with meaning\hyp{}making capacity.\index{ecosystems!regenerative}


We have not come nearly far enough in creating successful businesses that are part of fully viable, antifragile, circular, or blue economies, as John Elkington (originator of the triple bottom line) himself recognised in his recent HBR article\cite{elkington-triple} proposing a ‘manufacturer's recall’ of the concept to replace the ‘defective parts’. I hope that this book contains at least a few usefully upgraded parts, especially the FairShares Commons!


Elkington is very clear that his triple bottom line (people, profit, and planet as equally important measures of business success, not just profit) was intended to catalyse deep and far-reaching transformations of the role that business plays within the larger context of society and our natural ecosystem. Many companies have people within them that have the triple bottom line deep in their values. Many of these company's staff and customers recognise, even if they may not use those words, that the triple bottom line is vital for their interests, or at least their children’s and grandchildren’s interests. Even many major investors, like the pension funds, recognise at some level that the only way they will be able to provide the pensions\index{pension} we will all depend on is if the businesses they are investing in continue to thrive by adapting to the challenges of the next 50 years.


Asking this question through the lenses\index{lens} of Picasso,\index{Picasso, Pablo} we come to the conclusion it's because we are not seeing what we are looking at \emph{as it actually is}, but distorted because we are only using one perspective. Asking through the lenses of relativity and quantum physics, we begin seeing the fundamental particles and interactions of our economy clearly, and can begin to talk about our economy in a sufficiently representative and useful way.


Most efforts towards a circular economy address the technical challenge of making products circular, not the system-wide adaptive challenge of what it takes for companies to trust each other enough over the decades needed for each iteration of the circle, and across multiple large, interlocking circular ecosystems. This trust is practical, about collaboratively sharing resources and the wealth generated, across all capitals, by each company in the circular ecosystem to all companies over decades, maybe even centuries. Enough sharing for the ecosystem to thrive despite the issues caused by our non-ergodic world (Section~\ref{section:ergodicity}). The typical venture capital strategy of  portfolio, unicorn, and exit in ten years is a recipe for failure. The lack of understanding of non-ergodicity is also behind systemic wealth inequalities in nations; e.g. in the USA~\cite{berman-wealth}.


It is time to integrate all our approaches and efforts to build a better world. \index{Elkington, John}


\section{Immortality, or death and rebirth?}
\label{section:immortality-or-death}
Companies built around shareholder return, with tradable investor shares, have immortality as a core assumption. But unless the company and all of its stakeholders have a sufficiently high adaptive capacity to transform all aspects of themselves as living beings faster than their environment is changing, the company will reach a point where it is ready to die. The problem is, because we design them for infinite life, we end up with zombie businesses that cannot be allowed to die.


This immortality has also created the binary VC model of an exit on the VC fund's timeline: either the business is force-fed into exponential growth, like a goose delivering foie gras to the VC firm, or it is shut down. 


In an evolutionary, driver-based company, it is easy to see when it is time for the company to die. So built into the very heart of an Adaptive Organisation is a frame of reference\index{frames of reference} that constantly evaluates the current driver against the business and its capacity to adapt, to decide if it should die. Just as each cell in your body, and your body as a whole, has a built-in evaluation of when it has reached the end of viability. \index{Adaptive Organisation}


Of course you, as an investor, always need a way to get back at least as much money as you invested, better enough extra to counteract losses and provide a healthy income. (Keep in mind many investors are the pension funds many depend on for their retirement.) You always will want the organisation to operate with maximum efficiency in using your capital wisely to deliver business results 


Equally, if you are a member of staff, you absolutely want to have a healthy pension when you retire. So whatever we do, the key benefits of investor exits and trading investor shares on a stock market must be preserved. We just need to invent a new game that does that without the harm of today's approach.


In an adaptive business the timelines are decoupled. Investors exit on their timeline, staff exit on their timeline, and the business expires or continues according to its nature, needs and timeline.


\section{We need regenerative ecosystems}
\label{section:incubating-regenerative-ecosystems} \index{ecosystem!regenerative|(}


One recent client said to me: 


\begin{quotation}
Graham, there is no such thing as failure in an ecosystem of Adaptive Organisations where all are Level 5 FairShares Commons. No binary classification. Whatever the outcome from innovation or an experiment, the value, the learning stays in the ecosystem and can be used by others. 
\end{quotation}


Each stakeholder getting their unique needs met, on their own unique timeline, with sufficiently low risk and adequate reward, isn't going to work well if we keep investing the way we are today. It's time to mimic nature, the master of evolution and long-term thriving, and primarily invest in the regenerative business ecosystem as a whole, rather than individual companies. 


Then the most advanced companies around, Level 5 in all three axes (Learning as a purpose, Chapter~\ref{chapter:who-is-your-organisation-human}, Autopoietic, Chapter~\ref{chapter:who-is-your-organisation-tasks}, and FairShares Commons, Chapter~\ref{chapter:who-is-your-organisation-incorporation},) can truly deploy their full power for the benefit of all stakeholders.


We have all the elements we need to deal with the global challenges that we face. We do not need to invent some completely new alternative to replace capitalism, we just need to rejig it for all capitals, to end with a regenerative Economy of the Free, built of regenerative businesses, with at least 10\% at Level 5 on all axes, at least 40\% at Level 5 incorporation and Level 4 on the others.


You may well at this point be saying that there is absolutely no way we can design a global regenerative business ecosystem. You're absolutely right, we cannot in any linear predictable way design a regenerative ecosystem. What we can do is take lessons from how quantum physics works, and its central property of emergence (Chapter~\ref{chapter:emergence-einstein-picasso}). 


The first lesson in quantum physics\index{physics!quantum} is that many of our hard distinctions are only conceptual constructs to make it easier for us to grasp how the world might work. Treating stakeholder categories in a business as having hard, distinct boundaries is such a poor reflection of what a business actually is, it can neither see nor make use of the real life of a business as a living being.


You may already be a stakeholder in so many businesses across so many categories that you cannot know of them. You are probably an investor in oil, weapons, pornography, and many other businesses and industries that you do not know you are an investor in, if, for example, you are in a pension fund. 


Because pension funds\index{pension!fund} etc. investments are constrained  through regulation they may have to invest in companies that are harming our future. Even though, paradoxically, that puts your retirement at significant risk. Do you know what your pension fund invests in?


This illustrates how each of us is in multiple stakeholder categories to a far wider range of companies than any of us is aware of. As an employee of one company, you are an investor in all the companies that your pension fund invests in, and so part of creating the employment context for all the employees in those companies. Who themselves have pension funds that may well be investing in your company and creating your employment context.


You are a customer of a very wide range of companies, both directly, and indirectly through the supply chain that supplies them. 


We are already an ecosystem of interrelated stakeholders and businesses. \index{stakeholders} Almost every one of us is a member of multiple stakeholder categories across a range of businesses. So at the end of the day, the common ground that unites all of us is that we are all part of one global economy, even though we pretend it can be separated into disconnected parts.


Businesses are open, living systems in an ecosystem, so we must construct a reality based on the nebulous, ambiguous, volatile, uncertain, and complex nature of such an ecosystem. Then we can reshape the fundamental elements and their interactions to maximise their regenerative capacity.


Another lesson from quantum physics: you cannot predict where a particle will go\textemdash it will go wherever it can. The same is true for the economy as a whole, and any individual business. Like a child ought to grow up to become the best version of itself, growing strong across all its gifts, and so needs parenting, not command and control ruling. \index{physics!quantum}


For humanity to survive and thrive in the coming decades we must have a regenerative global economy, \index{global economy, regenerative} and to have that we must build regenerative ecosystems composed of regenerative businesses. But none of us could ever design one. What we can do is set the starting conditions to develop the natural gifts of each individual and business, and through that maximise the regenerative flow of all capitals\footnote{Indra Nooyi, former CEO of Pepsico, realised early on that sustainability\index{sustainability} was the future, not just a fad; and pointed at the growing imperative to focus on the long term, on society as a whole, and to focus on shapes, not just numbers~\cite{nooyi-shapes}. Given that she was CEO for an amazing 12 years, seven longer than the average CEO tenure, I suggest her words are worth listening to. }. \index{Nooyi, Indra}


We can do this best at a unique individual company level if the companies are part of the oneness of an entire ecosystem of companies. Nature \index{nature} has figured out how to do this extraordinarily well over millions of years. Across nature, almost all life uses a small number of common metabolic pathways and common DNA \index{DNA} or RNA. We should use these same fundamental building blocks in building the elements of our organisations. Elements that you have seen above, like full accountability, circle structures, and inclusion of all stakeholders all the way up to the FairShares Company incorporation.


It also means using the same set of complementary currencies, \index{complementary currencies} one for each unique type of capital. Arthur Brock, \index{Brock, Arthur} with his Holachain and Metacurrency Project, is one person doing superb work in developing a platform optimised for just such a set of complementary currencies. The Telos Foundation \index{Telos Foundation} is also doing superb work on the Telos blockchain.


One core property of ecosystems with multiple capitals \index{capitals!multiple} in flow, as you find across nature, is a floating exchange rate mechanism between each capital and its associated currencies. 


One of the reasons why our dysfunctional global systems are held stable in the dysfunctionality is that we have a fixed exchange rate between many of these capitals and money. For example, you may believe, as many people in business do, that your time is worth a fixed amount of money. When you get hired into a job, and your monthly salary is fixed, what's being done is fixing an exchange rate. \index{exchange rate} The company and you agree, in the complete absence of any live, real-time data on value, that one month of your life and everything that you do for your company through your efforts and energy in that month is worth this fixed amount of money.


We do that because, until recently, it's not been viable to have a time currency, or a reputational currency, or keep track of social capital and all the other kinds of capitals and their flows in native currency units. It’s not been viable to exchange one currency for another with a real-time exchange rate between them, based on real data on, e.g., how much money corresponds to how many hours of your time now, in each context that you use your time in (see Chapter~\ref{chapter:know-our-economy} and the appendix to see how a general theory of economies, \index{general theory of economies} enables this). With projects like Holochain, Telos, and all the work being done in the field of complementary currencies and distributed ledger technologies, we have this capacity.


These are the ecosystems that we need to incubate now, what I call deep business ecosystems. Where all organisations recognise the imperative to use the same DNA, the same metabolic pathways, and work with the nebulous common ground of multiple stakeholder categories spread in an unknowable and undefinable way across all the businesses within the ecosystem.


These are the three foundations of a regenerative economic ecosystem, and are the same as nature's foundations: 


\begin{itemize}
\item all the regenerative businesses use the same DNA, especially for incorporation and value flows; 
\item all the businesses tap into the same metabolic pathways, i.e., value flows: the same set of complementary currencies, at least one for each type of capital; 
\item all are designed to harness collaboration and conflict to get the best possible outcomes for all long term, with all capitals and their stakeholders in an appropriate balance of power. 
\label{list:ecosystem-principles}
\end{itemize}


Then we can adapt as individuals, as companies, and as the entire ecosystem, to address are global challenges.\index{ecosystems!regenerative|)}






\section{Incubating regenerative ecosystems}
\index{ecosystems!incubating regenerative}


This means creating incubators that produce cohorts of companies ready to enter true ecosystems that optimise both their internal processes and the connections between them. Inside each company and between them this means the channels for interaction, for value flow, the sharing of wealth generated. Each is best able to do this when all are at Level~5 in each dimension of Figure~\ref{figure:three-axes}; using a FairShares Commons \index{FairShares Commons} incorporation, having gone beyond conventional Ocracies\index{Ocracy}, and a Level~5 implementation of something like our Adaptive Way . 


This is the minimum that we need to build viable circular economies, blue economies, or triple bottom line economies, because each brings all of the vital elements of a true ecosystem.


These enable, at all six strata introduced on Page~\pageref{list:six-layers}, the optimum level of collaboration (wealth and resource sharing) and competition to have maximum regenerative power in our non-ergodic world. Especially a Level~5 incorporation like the FairShares Commons creates the trust needed to share wealth, because it creates systemic trust and wealth sharing anchored in each company’s articles of association: the players in the ecosystem have appropriate levels of governance power and a share of wealth in each other to maintain ecosystem integrity in the presence of predators and parasites. Without this the level of collaboration needed to counter our non-ergodic world is fragile, and will crumble soon after the first big successes. Then we will all lose.


Each cohort, once sufficiently incubated to have emerged from embryo stage through toddler and into early teenager stage can then stand alone and continue to grow as part of a larger regenerative ecosystem of companies.


This way we will steadily grow the size of our regenerative company ecosystems and, through power-law scaling, can very quickly transform our global economy into a regenerative one. If we really put our backs into this, if enough of you who have passionately protested in the Extinction Rebellion recently, in Occupy Wall Street, start your own regenerative ecosystem incubators or join the ones we believe will soon be in existence, we can scale a regenerative business ecosystem as fast as the school climate strike and Extinction Rebellion\index{Extinction Rebellion} scaled in 2019.


Incubating and investing in regenerative business ecosystems is what I (Graham) am doing now, and intend to focus on for as long as I deem it the best place to put my gifts. Generating startups that have the best possible prospects of growing to Level 5 on all three axes begins long before the source starts talking about their idea to the first follower.


I have begun working with individuals, or at least the founding team before they incorporate, and before they have sown the seeds of irreconcilable conflict\index{conflict} in their interpersonal interactions. These individuals come together into an initial startup university, master sufficiently well state-of-the-art approaches in working within themselves, between each other, on the tasks and roles of the work, and creating and maintaining FairShares Commons incorporations. In parallel, they begin using design thinking, lean start-up thinking,  business model canvases, and everything else that is standard, like marketing, sales, and developing minimum viable products, to get to their first pre-seed minimum viable company. At this point, it may take off as an incorporated company, or it may continue as a virtual company within the incubator (e.g. as a blockchain based DAO).




\section{Financing regenerative ecosystems}
\index{ecosystems!financing regenerative|(}
To have regenerative ecosystems we need to invest in the ecosystems as a whole, not single businesses. This requires a paradigm shift in each, and for some, also a shift in legislation. 


Large institutional investors invest in huge diversified portfolios (based on modern portfolio theory) spread across multiple arenas of the economy. Such a portfolio is a shallow ecosystem, so they lose money compared to the deep ecosystems of this book, because they cannot satisfy the three principles of an ecosystem on Page~\pageref{list:ecosystem-principles}; the companies do not have an incorporation that can fully harness the nebulous interconnectedness within each company and across the ecosystem to counter the loss\hyp{}causing consequences of the unpredictable random bad luck inherent to our non-ergodic\index{ergodic!non}\index{non-ergodic} world (recall, the difference between Figures~\ref{fig:bet-path-1c-cutoff} and~\ref{fig:bet-path-1p-sharing}.) 


Large institutional investors, though, such as BlackRock,\index{BlackRock} are beginning to recognise this. Across the world, large asset managers have 62 trillion dollars in assets under management, with nowhere safe enough to invest, confident that they can pay out your pension when you need it in 10, 20, or 50 years. 


Mobilise a good fraction of these \$62tn assets under management into an adaptive business ecosystem, and we have all we need to address our crises.


Investing in a deep ecosystem composed of fully Adaptive Organisations has a number of advantages, both for investors primarily interested in maximising their returns over the long haul, and those investors only interested in maximising their reward/risk ratio over just a few years. \index{Adaptive Organisation} If you care about ergodicity \index{ergodicity} (Section~\ref{section:ergodicity}), you will love it.


Investing in deep ecosystems is the best way of investing in an economy that is  path-dependent, rather than ergodic,. Nature\index{nature} has proven this over millions of years and has consistently benefitted from an all-capitals positive cumulative annual growth rate. From the beginning of life on earth, life has steadily built all the capitals  it needs to thrive\textemdash until two centuries ago when we exploded as apex predator. 


First and foremost, investing in a regenerative ecosystem of Adaptive Organisations will return a better long-term return on investment than any other investment decision you could take today. That is the central function of investing versus charity: to generate a return sufficiently in excess of the losses that the underlying capital continues to grow.


Some of you reading this book may believe that all profit is fundamentally wrong. But nature tells us that high levels of profit are fundamentally necessary, provided they are part of a sufficiently large open system. One helpful illustration I like is an oak tree. Nature\index{nature} plants one acorn, which grows into one tree. Over a number of centuries the oak tree will generate a return to nature of thousands of acorns. Nature thrives because each oak has an annual rate of return of multiples of thousands. Of course, because nature is a living system, that return is to the capital base of the ecosystem as a whole, not just the original oak tree that produced the acorn this oak tree grew from. 


A big positive return on investment is absolutely vital for any functioning natural or business ecosystem that is going to regenerate rather than degrade, big enough to absorb any unpredictable losses from system shocks.


Hopefully this addresses any concern you may have had that we are describing a utopian fantasy; now let's look at why investing in a regenerative ecosystem is a no-brainer.


The one obvious one, that I won't spend more than this paragraph on, is that many of today's investments are stranded assets for anyone who hopes to have a viable life themselves, and for their children, in a few decades. All current investments in companies that are based on polluting our atmosphere with carbon dioxide and other greenhouse gases, and cannot viably transform themselves by 2030 within their shareholding structure, are stranded assets. They will have zero value before many of you reading this need the pension that depends on these companies.


It's time to invest in regenerative ecosystems, because of their advantages versus the isolated limited companies of today.


\begin{enumerate}
\item The biggest boost to your return on investment is that all stakeholders are aligned in their exposure to risk and reward. All the staff in an Adaptive Organisation share in the long-term wealth generated, and have balanced voting rights with you as an investor, so when times are hard they will naturally invest money in the company by taking a cut in salary. Their long-term interests are maximally aligned with your long-term interests.\index{Adaptive Organisation}


\item Staff will be even more supported by you as an investor, because the FairShares Commons \index{FairShares Commons} rewards you for more than your financial investment, it rewards you financially also for investing your relationships, your connections, your word-of-mouth marketing, and perhaps even your effort in meetings in the company delivering tangible results. So the value that is sucked out of some companies by the mistrust and antagonism between founders and staff on the one hand, and investors on the other, is transformed into collaboration.


\item Customers, suppliers and those who are currently held outside the hard company boundaries will be contributing to the company’s success because they too are rewarded for investing themselves in its long term success and taking risks. Because of the FairShares Commons\index{FairShares Commons} structure, and because of the patterns of collaboration in getting tasks done and between human beings, it's possible and appropriately rewarding for anyone in any stakeholder category to bring their creativity, effort, word-of-mouth advertising etc. into play to benefit the company as a whole.


\item No longer do investors need to be the only ones making sure executive remuneration packages stay within reasonable limits, because the staff, customers, and suppliers do that as well. 


\item When a startup in a deep ecosystem proves sufficiently flawed that it gets shut down, the value that your investment has generated stays in the ecosystem. The people who you have invested in will rapidly be snapped up by the other companies that are scaling in the ecosystem. They will not need to spend anything training new hires from outside the ecosystem in how the Adaptive Organisation ways of working and being. Low risk, higher average return than today. The companies are fully Level 5 Autopoietic \index{autopoiesis} from the start.\index{Adaptive Organisation}


\item Core IP remains part of the ecosystem as a whole, rather than becoming unproductive, or delivering value somewhere else. Key here is making it natural and easy for lots of inventions to come together into one big innovation that transforms,  Like the inventions of TCP/IP (the internet protocol) and HTTP coming together at CERN, the particle physics research centre, to create the business ecosystem based on the world wide web. 


\item Innovation, invention becomes systemic. We combine the speed and agility of modern startups with the deep innovative capacity of places like the old AT\&T’s labs because the ecosystem enables inventors to focus only on inventing; and startup founders in the startup programme to turn the inventions into profitable, impactful businesses. \index{AT\&T} The ecosystem is a production line for innovation and invention.


\item Money\index{money} is not the sole metric in such an ecosystem. All relevant capitals have direct currencies that are metrics. Currencies \index{currency} of attribution for trust, for community, for creative capacity, for simply showing up each day with energy and a smile that increases the whole team’s output are all equally valid metrics. Even better, what truly counts is valued, not just what can be counted, and certainly not just what can be counted in money.
\end{enumerate}


\index{ecosystems!financing regenerative|)}


\section{Examples}
\label{section:financing-examples}


A wide range of companies have demonstrated how consistently applying only one or a few of the elements in this book consistently reduces risk and increases reliable long-term performance; and many prove the power of integrating impact and profit into one. Below are a few examples. There are many more examples than these.


\paragraph{Visa} The Visa \index{Visa} corporation had elements of the Free company incorporation until 2007. The credit card issuing banks were all members of a non-stock membership corporation which could not be sold. Membership gave non-transferable, non-sellable, irrevocable participation rights, and the founder, Dee Hock, \index{Hock, Dee} refers to it as a reverse holding company, as the parts hold the core. Dee sees this as the central reason why Visa transformed the emerging credit card industry from one rapidly killing itself as it destroyed its capital foundation into one that worked. All we need to do now is reapply his lessons to all our capitals. 


\paragraph{Mondragon} Mondragon\index{Mondragon} is a deep ecosystem of cooperatives. It is currently the 10th largest company in Spain, had €24.725 billion in total assets, and 74,117 people spread across 257 companies in 2014. Mondragon is active in finance, industry, retail, and knowledge. The success is directly attributable to Mondragon being an ecosystem of interrelated cooperatives, with an investment bank in the middle (somewhat like Berkshire Hathaway). They have a cap on managerial pay of between three and nine times the lowest-paid full-time equivalent. The level of engagement and collaboration (reciprocal helping) led to Mondragon going through the 2008 crash without adding to the 3.5 million new recipients of unemployment benefits in the rest of Spain. Only one cooperative, employing 30 people, closed during that period, and the staff soon moved to other coops.


\paragraph{Bridgewater} Bridgewater \index{Bridgewater} is a \$160bn (2019) asset management firm, founded by Ray Dalio\index{Dalio, Ray} in 1975. It runs on some of the core principles of this book around intra-and inter-personal developmental dialogue patterns\cite{dalio-principles}. This is widely seen as the reason why they have succeeded consistently better than 90\% of comparable companies since they were founded. Life in such a company is not a life of ease, free of challenge; far more, it is a life of continuous adaptive challenge. The only difference with other companies, and why Bridgwater thrives where others have failed, is that the adaptive challenges are accepted, not swept under the carpet; and the entire organisation is geared to provide everyone willing to take on the challenge of changing who they are with the scaffolding they need to stay in or below their Zone~3 assisted growth zone.


\paragraph{Wikipedia.} In a handful of years Wikipedia\index{Wikipedia} has become the world's pre-eminent source of reliable information. One of the biggest things that makes Wikipedia trustworthy versus almost any other source of information is that it is immediately apparent if there is a disagreement between factions on what the current best answer is\textemdash unlike the average newspaper, or book, which will support the author’s perspective without pointing out right at the very beginning how solid that perspective is. Wikipedia's core strategy is that people are basically good and will collaborate, given the chance, a fair context, and enabling structures. 


\paragraph{Doughnut Economics} \index{doughnut economics} A number of excellent steps have been triggered by Kate Raworth’s \index{Raworth, Kate} brilliant book \emph{Doughnut Economics}\cite{raworth-doughnut}. For example, in New Zealand M\=aori traditions have been integrated with doughnut economics into the Te Reo M\=aori doughnut\footnote{\url{https://www.projectmoonshot.city/post/an-indigenous-view-on-doughnut\%2Deconomics-from-new-zealand}}\index{doughnut economics!Te Reo M\=aori}\index{Te Reo M\=aori  |see {doughnut economics}} created by Teina Boasa-Dean, \index{Boasa-Dean, Tenia} Juhi Shareef,  \index{Shareef, Juhi} Jennifer McIver, \index{McIver, Jennifer} and Tineke Tatt.\index{Tatt, Tineke}


\paragraph{And many more} Many more companies are incorporated at Level~3, 4, even with elements of Level~5. You can find details in the excellent book from the Purpose Foundation\cite{purpose-foundation}, such as Carl Zeiss, Robert Bosch, Sharetribe, Ecosia, Ziel, Organically Grown Company, Waschb{\"a}r, Elobau, and Ghost.org; as well as companies high on the roles and tasks, or human axes, like Decurion,\index{Decurion} described in detail by Laloux\cite{laloux-RO}. Also look at the Deliberately Developmental Practitioners Network\index{Deliberately Developmental Practitioners Network}\footnote{\url{http://ddp-network.org}} for many other success stories of life-giving organisations.


\section{Your FairShares Commons ecosystem}
\label{section:create-join-ecosystem}


It's time to consistently think \begin{quote} if it is to be, it's up to me. \end{quote}


None of this will happen if any of us continues to think that we are too small to make any difference. 


None of this will happen if we think that they must do it, they must fix what is broken and give us what we need. There is no “they”; they are us.


Punitive justice, blame, attempting to divide ourselves into distinct groups of victims and perpetrators will not get us to a regenerative global economy.


If you want to have a world that is healthy for yourself and all of life, I believe that the best thing to do is put into practice any part of this book that you can put into practice today. Steadily expand how much you put into practice until you are joining or creating a regenerative FairShares Commons startup and regenerative FairShares Commons ecosystem.\index{FairShares Commons!ecosystem}


I believe that now is the best time we will ever have to build regenerative businesses and economies that work for you, for our planet, and for profit. And with the huge difference in average outcomes you saw in Figures~\ref{fig:bet-path-1c-cutoff} and~\ref{fig:bet-path-1p-sharing} we can still turn things around, if we move fast and big.