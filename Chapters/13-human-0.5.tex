% Added Ongoing Regard, Public Agreements, Deconstructive Dialogue to index on 210601
% Changes to yellow para on deconstructive dialogue
\chapter{Your organisation is its people}
\addcontentsline{toc}{chapterdescription}{Starting and then growing an organisation begins and ends with the strength and purity of the people acting as source. Individual and cultural meaning\hyp{}making creates the reality that everyone in the organisation experiences. Psychological safety, leading to deliberate personal development and high levels of interactivity is the foundation for a long-term thriving business. Learn dialogue patterns to harness the value of conflict in yourself and teams, to have an antifragile and successful business.}
%\addcontentsline{toc}{chapterdescription}{\pagebreak}
\label{chapter:who-is-your-organisation-human}


\begin{chapterquotation}
Every child is an artist. The problem is how to remain an artist once [you] grow up.\\
\raggedleft\textemdash Pablo Picasso\index{Picasso, Pablo} 
\end{chapterquotation}




\section{Where your organisation is: human}
\label{section:where-is-your-organisation-human}


The six stages of organisational development given here were developed by one of us (Graham) and Bernhard Possert. \index{Possert, Bernhard} You will be able to read more about these in a forthcoming book, and can download the latest white paper on these from us, or access the online diagnostic via our websites\footnote{This chapter was sponsored by the Consorticon Group}.


The list below gives you an idea of where your organisation is now.


\begin{description}
\item[5: Learning as a purpose.] 
Development in its own right is a purpose of both the organisation and the individual, with developmental activities extending as far beyond formal organisation boundaries as is helpful. The organisation is designed in full recognition that only an open system in continuous nebulous, unknowable transformation can provide the high level of learning needed to fully fulfil the organisation's and individual's learning purposes. 

All employees participate systematically in improving the processes to promote development. People invite others to trigger them, in order to support development by creating developmental tension and conflict situations. Challenging assignments and colleagues are actively sought in order to promote individual development. People stay within their roles only as long as they are developing, subject to business needs.


\item[4: A learning system.]  
The organisation works on all the collective meaning\hyp{}making stories, continuously refining them according to context, strengthening them where they are a helpful representation of actuality, and chipping away at them where they are a poor, unhelpful representation of actuality. The organisation uses a common language and has clearly articulated principles for development. Everyone has the duty to surface tension, conflict, potentially dysfunctional or misaligned behaviours, using developmental and neutral languages like NVC and deconstructive dialogue, and then process them to facilitate development. 


The organisation’s values and principles are laid down clearly, and misaligned behaviour is treated seriously using structured dialogue processes to facilitate development. Hierarchy of power, age, expertise, background, etc. does not automatically give legitimacy to opinions, rather they are evaluated according to an appropriate frame of reference and the content. Conflicts are seen as a source of development, and there is a clear process to learn from them.


\item[3: Common learning practices]  
People are matched to roles according to the fit: Size of Role vs. Size of Person,\index{Size of Person}  the person’s natural energy versus the role’s requirements, the person’s skills versus the role’s requirements, and any other consideration relevant to the individual and the organisation maximising their capacity to thrive as independent living beings. \index{Size of Role} 

Leadership shares vulnerability, actively role models and strengthens the entire organisation as a psychologically safe space, where showing emotions is safe and supported. Exposing your limitations is encouraged and safe. Errors are actively harnessed, in the cultural norms, and in the structures, processes, and policies, for their developmental potential. Everyone, and the system, supports each other in achieving their personal goals. Structured practices and dialogue patterns are used for giving and receiving feedback.


\item[2: Psychologically safe for personal development.]  
How meaning\hyp{}making stories shape the experienced realities, and how they can hold someone back, is understood and applied in personal developmental processes. People have an understanding of their personal developmental goals, and implications of these, in all three axes and actively work on developing themselves. 


People know how to get support for their personal development from their peers, mentors, and coaches. People are able to recognise and evaluate when they are acting defensively, the vulnerability that their meaning\hyp{}making says needs protection, and can foster a positive climate to engage in this. Social skills facilitating interaction, cognitive depth, and other areas of personal development are visibly and clearly recognised as necessary to progress in the organisation.


\item[1: Individual skills, strengths]  
There are various organisation level programmes and support to foster individuals developing their personal strengths. The diverse workforce’s uniquely different talents, and the consequences that these differences have, is valued and actively turned into productive strengths using development processes. People have competency or skill developmental goals, and the organisation knows which skills they expect their employees to be able to use at which level of competency.


\item[0: Short-term gain.]  
Differences are not accepted let alone valued. Development is not part of work, it's a waste of time and money. People are either good for their job or not, they are an anonymous and easy to replace resource without any loss to the organisation from firing and hiring. The organisation is psychologically dangerous, with significant amounts of individual's energy going into Job~2, protecting themselves, including the need for significant amounts of recovery time outside work.
\end{description}


\section{Source}
\label{section:source}


At the beginning of anything new is a source\cite{koenig-source}. Every startup needs one person, sometimes two, very rarely more, to be the source. The source plays a critical role at every potential transformation point in the company's journey. For Apple, Steve Jobs was the source. When Steve Jobs was fired from Apple, \index{Apple} there was nobody who was capable of picking up the source role that Apple needed\textemdash someone with the exceptional farsighted view of long-term trends that Steve had. Apple shrank steadily until the company rehired him. With the source back in the company, the stream of innovation began flowing again.


The role of source is partly inside and partly outside the company. Their primary accountability is to anchor themselves far enough into the cloudy, nebulous emergent future. They must anchor themselves outside, not inside the company. And this must be something that is so natural to them as a living being that they cannot do otherwise.


In that sense, the source may well be the same person for multiple companies, or business units in a conglomerate. In P\&G, I (Graham) \index{Procter and Gamble} met a few people who were extraordinarily powerful sources and worked across multiple business units. One who I knew and admired was the source for products that turned into billion-dollar brands across many of P\&G's business units, from dental care to nappies.


In the traditional world of startups, most sources run at a fraction of their capacity, because they are the founders of one startup and spend huge amounts of time doing everything necessary to start up the business operations. This means we can be very hopeful that we will find ways of dealing with our impending crises as soon as we focus, as I propose in this book and am doing via Evolutesix, \index{Evolutesix} on investing in and incubating regenerative ecosystems of adaptive living businesses.


Because all the businesses are using the same FairShares Commons\index{FairShares Commons}  approach, along with multiple elements we will cover in this chapter, a single source can do that role for many companies in the ecosystem. Once the source has sufficient clarity on what it is they are seeing in the nebulous future as a potential business concept, and can tell the story of the emerging future sufficiently concretely and visibly for the first follower to engage, people start coming together. 


Every business begins with a meaning\hyp{}making story flowing from a source person. Every successful business change begins by transforming the old story into a new one.


\section{Coming together}
\label{section:coming-together}
This is where foundations are laid for a successful Adaptive Organisation\index{Adaptive Organisation}  to form and deliver the business concept with excellence through multiple pivots and transformations\textemdash or not. 


People come to an organisation because its central story connects with enough of their individual stories and inner motivators. So long as enough of their personal meaning\hyp{}making story fits within the organisation's meaning\hyp{}making story and its tangible structures and processes, they are highly likely to stay and contribute value. So at the beginning of any organisation, the more clearly and unambiguously the source is able to craft and communicate the organisation’s central stories (in other words its deep evolutionary purpose), the more likely the source is to have just the right people coming together to bring the organisation to life.


I (Graham) have seen the source disappearing like a stream vanishing into a hot desert, in my own startups, and in companies that I have consulted for, because the founding team coming together resulted in individual and collective meaning\hyp{}making stories that could not fully embrace the message the source was bringing.


Sadly the source themself often sows the seeds for this, because they do not always have a large enough Size of Person \index{Size of Person} to be able to do the job that a source is there to do. 


Mining and refining inner and interpersonal conflict from before a founding team even begins to talk about working together, through until long after the founding team ceases to work together, is the best way of creating thriving startups. Do this with the on-the-job, just-in-time, accelerated development of team and individual performance (at no extra cost and no extra time) described in Section~\ref{section:team-coaching}.


If you are a source, a founder, or anyone else forming a team, there are two questions you need to ask. Get good answers to these questions, and the probability that you will fly is high. Get poor answers to these questions, and the probability that you will tank is equally high. 


\begin{description}
\item[Individuals.] What natural talents, spread of personal natural energies, and skills do I need on the team? What minimum and maximum Size of Person (SoP) is appropriate for the adaptive challenges we will be facing?\index{Size of Person}
\item[Teams.] What is the right balance between team and individual perspectives, and what do we need to do to turn the individuals into a team?
\end{description}


Comparisons of teams and organisations that consistently deliver top-level results show that most businesses get the answers to these questions wrong. The role of individual talent is seriously overrated.


A team of competent but average individuals will typically outperform the same number of top talented individuals who haven’t really bonded into a team. So if you are a source, or a team leader of any nature, emphasise bringing together people who can function quickly and effectively as a single team because their meaning\hyp{}making stories and Sizes of Person match, rather than a collection of prima donnas who struggle to bond into a single team.


Make sure that you have one or two people on the team who cannot not open their mouths, even if it lands them in trouble. During my (Graham) time transforming Edgetalents (a strategy and organisation design consultancy specialising in impact and social enterprises in India, Kenya, Brazil and Colombia), \index{Edgetalents} we had a couple of people who were really good at bonding and bringing harmony to the team, and a few who did not hesitate to open their mouths. This combination brought just the right mix of conflict to make sure that we missed nothing important, and the team dynamic to then mine and refine the value that conflict. 


In that team, we initially had very clear boundaries, with everyone knowing who was on and who was off the team. As we grew, that clarity weakened, and it was a reason why performance began dropping. As you form your team, be ruthlessly clear about who is on the core team and who is not. Be equally clear in the startup’s early phase that you have the best chance of success only if the team’s nucleus is committed. Not just committed in their alignment with what you're trying to do, but most importantly in all their working time and much of their private time. Be very wary of trying to succeed with a large team of people each working less than half of their time.


Paradoxically, just as you would expect from a book using relativity,\index{relativity}  quantum physics,\index{physics!quantum}  and Cubism\index{Cubism}  as a guideline to how we should think about business, the best teams need a few people who are absolutely not team players.


You can read more on why focusing on hiring the right individual is far less likely to give you the team you need to deliver excellent results in the HBR article of Diane Coutu\cite{coutu-why-teams-dont-work} \index{Coutu, Diane} on why teams don't work, and by Bill Taylor \index{Taylor, Bill} on why great people are overrated\cite{taylor-great-people-overrated-1, taylor-great-people-overrated-2}.


A wide range of different studies have demonstrated clearly that the star’s performance, whether it's a top-performing CEO, a Wall Street analyst, or a startup founder, has far more to do with the team around them, and the company’s culture, structures, and processes than with their own individual talent. For example, Groysberg\cite{groysberg-stars} looked at 1,000 star analysts at Wall Street investment banks. He saw without any doubt that whenever any of them changed firms their performance dropped immediately and stayed lower than before; except if they changed to a firm that had overall better teams, culture, structures and processes; or if they took their entire team with them. 


The other significant exception was when a woman changed companies. Perhaps more women these days are inherently better at landing in a new team and facilitating the team bonding better, thereby increasing everyone's performance. So if you are an investor, you may well do better investing in female rather than in male founders, at least until everything in this book becomes standard practice amongst everyone.


And if you are a founder or source or any other kind of team or circle leader, take hope from the evidence that the war for talent is a myth\cite{gladwell-talent-myth}. There are more than enough highly talented individuals around. Don't waste your time or money trying to find the one person that you imagine will be so talented that they will fix everything that is a mess in your organisation. Attract people who can bond quickly to become one high-performance team.


And then focus your effort on growing a living organisation.


\section{Growing a living organisation}
\label{section:growing-living-organisation}


Sadly most startups, and businesses attempting to transform themselves towards becoming an Adaptive Organisation, \index{Adaptive Organisation} focus on the activities variable of Equation~\ref{eq:living-organisation}. Traditional companies train people in yet another flavour-of-the-month skill or organisational redesign, or perhaps change from one software to another, typically hiring a consultant specialised in a single aspect of the Activity variable. Sad, because this is the smallest lever you have to do your job of delivering superb business results.


The biggest lever that you have lies in transforming each individual’s meaning\hyp{}making stories and those of the whole team: the Context, or C variable.


Linked to this, your second biggest lever is in developing high-performance collaborative relationships between people as living human beings in your organisation. Relationships that are highly collaborative, in the sense of reciprocal helping (Section~\ref{section:reciprocal-helping}), and that have high interactivity (Section~\ref{section:interactivity}).


This means focusing your efforts first on the human dimension, and putting in place the minimum viable structures and processes to support your activities. Get your effort into the culture, how people relate to themselves and each other; take into account the problems of relational loss and the rapid scaling of relational complexity as organisations grow, described in Section~\ref{relationship-scaling}.


\section{Personal arena}
\label{section:personal}
We've already covered most of what you need at work to understand yourself, and to grow yourself, by harnessing the internal conflict you experience. So in this section, we will assume everything that has already been covered in Chapters~\ref{chapter:who-am-i-base} and~\ref{chapter:who-am-i-meaning}, and only cover the essential additional factors that the organisation needs to provide you with.


\subsection{Psychological safety}
The most important factor that the organisation needs to provide you with is psychological safety. In your private life, psychological safety \index{psychological safety}  is part of what you need to construct yourself. There are many ways that you can do this in your private life, both by rewriting your own meaning\hyp{}making stories and by making decisive choices about who you spend time with. Sometimes you will be in situations that are beyond your capacity to create psychological safety alone, and will need to turn to others for support, maybe even lawyers. 


Amy Edmondson \index{Edmondson, Amy} in her excellent book \emph{The Fearless Organization}\cite{edmondson-fearless} writes about the competitive edge organisations have when they take care of the psychological safety of everyone working in the organisation. This applies the words of Peter Drucker, \index{Drucker, Peter} that the purpose of an organisation is to make each individual strength productive and their weaknesses irrelevant, to the organisation’s job of to minimising the energy invested in Job~2, and maximising the energy invested in Job~1.


Management by fear is an absolutely certain recipe for the death of any organisation facing adaptive challenges. You as an individual living being, and your organisation as a collective living being, will never have the adaptive capacity you need unless you have psychological safety.


Creating psychological safety \index{psychological safety}  requires action on all three axes of Figure~\ref{figure:three-axes}. It is simply not enough to work on the culture, and other aspects of the human dimension; better, but also not enough to bring in self-management on the roles and tasks dimension, if you still have a limited company or similar. You must incorporate at least at Level~2 to have some level of robust psychological safety, better at Level~3. 


Only at a Level~5 fully free company do you have fully antifragile psychological safety. I strongly advise everyone starting up now to go straight to Level~5. If you cannot, look carefully at what risks that brings. Risks that may prevent you from building the better world you intend to; because you cannot build a better world on the foundations of any incorporation that treats companies as if they were property. 


Also choose wisely on the Roles and Tasks axis. You have psychological safety if you have a say over who can exert power over you, whether through their role or their person. If managers / lead links are simply appointed you have at best fragile safety, because you cannot easily protect yourself if, say, a narcissistic person is allocated power over you. This is why some implementations of Ocracies fail. 
\subsection{Peak adaptive performance}
\label{section:vygotsky}
\index{Vygotsky zones|(}
\index{zones of proximal development}
Otto Laske\index{Laske, Otto}  and I described what happens at the individual level during the change journey towards Holacracy or sociocracy, drawing on my experiences in P\&G China, where I used what I know would now call self-organising team dynamics, in my own businesses and as a consultant.


Every challenge that you face in an organisation will put you into one of four Zones versus your Size of Person (the combination of your dominant meaning\hyp{}making stories and your fluidity in using all 28 transformational thought forms).


\begin{description}
\item[Zone 1, Comfort zone.] In Zone~1, you are able to deal with the challenge completely within your Size of Person. \index{Size of Person} In a sense, the challenge is smaller than you are. You have more capacity for transformation thinking than is necessary to fully overcome the challenge, and the challenge fits completely within your dominant meaning\hyp{}making stories. 


So the challenge will feel easy and natural to you, without any stress.


\item[Zone 2, independent growth zone.] If you are in Zone~2, the challenge is marginally bigger than you are. To rise perfectly well to the challenge, you need to exert yourself to use forms of thought that you have begun using, but are not yet fully fluid in using. The reality that you need to shape and experience is just within your grasp, because you have already begun developing the appropriate meaning\hyp{}making stories. 


So you can support yourself in fully rising to the challenge. You will feel some level of stress, but on the safe side. So long as less than 40\% of your work places you in Zone~2, you will feel stretched while you are rising to the challenge, proud of how you have grown, and the results you've delivered at the end.


\item[Zone 3, assisted growth zone.] If you are in Zone~3, the challenge is bigger than you are. To fully rise to the challenge and deliver the results that business success needs you to deliver, you will need to use thought forms that you are only able to use if somebody else supports you. You will also need to use meaning\hyp{}making stories that are just off the edge of your current set of stories, but so long as somebody else is able to support you, you will be able to build a bridge into the void. 


With support from others, and of course from your organisation's structures and processes, you will just be able to deliver adequate results. You will feel stressed, perhaps anxious, during the process. But, at the end you will likely feel exhilarated and proud that you have accomplished something that at first seemed beyond your capacity. Even more so because you can see how overcoming this adaptive challenge has led you to change yourself into someone that you are proud to now be.


\item[Zone 4, panic zone.] If you are confronted with an adaptive challenge that places you in Zone~4, find a way to delegate the challenge up, down, or sideways within your organisation, or to somebody outside your organisation. If a challenge is so much bigger than you that it places you in your Zone~4, it requires transformational thinking fluidity and meaning\hyp{}making stories that are beyond your grasp, even with the best possible support that your colleagues and an organisation can give you.


So regardless of what you or anyone else does to support you, you will fail, unless you hit a seam rich in the ore of underserved luck. Someone else must pick this up. If you are the CEO, that means finding consultants you can really trust, or perhaps relying on your board.
\end{description}


You will be at peak adaptive performance if your challenges place you in exactly the right mix of Zone~1, 2 and 3 for you. If you are leading a team, or are the lead link for a circle in one of the Ocracies, keep a very careful eye on the challenges that each individual and the team as a whole is facing, compared to each individual Size of Person. Do your best to keep yourself and everyone else out of Zone~4, and well supported in Zone~3. This will maximise the energy in Job~1, and minimise the energy in Job~2. Also compare this with Section~\ref{section:divided-teams} on divided teams.


Think about yourself and your colleagues in your organisation. Your feelings\index{feelings}  are your most reliable guide to which zone you are currently in. If you are feeling completely overwhelmed, if you think you are so far in over your head\cite{kegan-in-over} that you know that you are going to underperform because you lack something essential, you are most likely in Zone~4.


If everything feels easy, if you feel no stress at all, then you are most likely in Zone~1. Usually a technical challenge places you in Zone~1 or~2.


You may also be in Zone~4 and feel no stress at all because you have failed to recognise the challenge’s inherent nature, and so see yourself as fully capable of rising to the challenge. This is most likely an adaptive challenge so big that you cannot even recognise how much bigger the challenge is than your Size of Person. \index{Size of Person} This is one of the most common causes of business leaders making atrocious decisions. They are facing an adaptive challenge that requires a Size of Person so much bigger than they are that they do not even understand that they haven't understood the challenge. 


There are only three possibilities here. 


\begin{enumerate}
\item You and your company may get lucky. 
\item You find out that the challenge was so much bigger than you are when you find out far too late that you've made decisions that have led your company into collapse.
\item Or, ideally, you have a colleague who does have sufficient Size of Person to grasp the true nature of the challenge, and you trust them enough to listen to what your colleague says. Then, together, you may be able to find a way forward that will enable your company to rise to the adaptive challenge and thrive.
\end{enumerate}


Zones~3 and~4 are not always easy to distinguish alone. In both cases, if you have recognised that the challenge is significantly bigger than you are, you will recognise that you need the support of somebody else and your organisation's structures and processes. To truly tell the difference, you will need to get the input of somebody who has these challenges within their Zone~1, and who knows you well enough to be able to evaluate how big the adaptive challenge is versus your current capacity for transformational thinking and meaning making.


If your work requires you to be in Zone~3 for more than around 20\% of the time, or it requires you to be in your Zone~4 at all, it's vital for your health, and for the short and medium-term success of your organisation, that you find your way to handing over all Zone~4 and enough of your Zone~3 challenges to somebody else.


Peak performance needs the right balance of zone~1, 2, and 3. Just what the balance is depends on your unique individual mix of cognitive biases, psychological makeup, transformational thinking capacity, and meaning\hyp{}making capacity. It also depends on how strong the scaffolding is that your colleagues and your organisation provide to support you as you grow yourself into the void. 


The more that everyone in your organisation \index{organisation} sees it as a living being, where its meaning\hyp{}making stories are the biggest lever to lift results, the easier it will be for you to get the scaffolding you need to spend more of your time in Zone~3, 2 and~1.


However, the more that your organisation's meaning\hyp{}making sees it as a complex system, perhaps as a highly complex machine or software system, the less capacity you will have to support yourself in Zone~2, let alone get the scaffolding\index{scaffolding}  you need to support you in Zone~3.


This is especially important if your organisation is changing from a traditional vertical management accountability hierarchy to any of the newer, agile approaches\footnote{Some of these are becoming known as Teal. \index{Teal} I’ve chosen not to use the name much here, as there is a gap between how the word is often interpreted and what is needed to be truly at teal.}. Such a change journey can very easily put everyone so deep into their respective Zones~3 and~4 that the entire organisation’s productivity drops well below the minimum survival threshold.


If you want to read up more on this, these four zones are derived from the original Vygotsky zones of proximal development\cite{vygotsky-thought, vygotsky-mind}. Originally applied to the development that takes place during childhood, I find the concept equally applicable to adult development.
\index{Vygotsky Zones|)}
\subsection{The workplace is not for changing others, nor for therapy}
I cannot stress this enough. Your boss, your colleagues, and your organisation have absolutely no right to demand that you change or grow in a way that they desire. None of them is you, none of them will live with the consequences, none of them is a god, even if they believe they are!


You are the only person who has decision authority over your developmental path. You are the person who has to live with the consequences of all of your decisions about growing yourself, and you are the person who has the best possible insight into all the hidden meaning\hyp{}making stories\index{meaning-making stories}  that shape who you are today. You are the only person with the right to look at who you \emph{might} become, and choose to start the journey towards becoming that person.


If organisations, or your boss, or colleagues, demand that you become someone, they are overstepping the mark. What is perfectly acceptable is for them to request unambiguously that whoever is in a specific niche of the organisation is a sufficiently good fit to perform well. If who you are is unable to perform sufficiently well, then the organisation must give you that feedback unambiguously, and the organisation has the responsibility to work in partnership with you to figure out whether or not this is an adaptive challenge that is part of your life path. If it is, then the organisation must offer you whatever support it is resourced to offer, to ease your growth, if you choose to.


If this is not part of your adaptive path\textemdash which is your decision, not the organisation's\textemdash or the organisation does not have sufficient resources to support you adapting yourself, it's important for both you and the organisation that you either move into a niche that does fit you, and that you can perform well in; or that you move to another organisation with a niche that you fit well into. Anything else is disrespecting the organisation’s right as a living being to perform at its best, and grow into its unique potential, or disrespecting your right as an equally living being to thrive and grow into your unique potential.


Finally, the personal arena of an Adaptive Organisation \index{Adaptive Organisation}  is also not there to give you whatever therapy you may need to deal with large trauma or mental health issues. Of course, the better an organisation is at everything in this book, the easier it will be for you to perform, regardless of your unique nature and life history.


A good example of this is the UniOne Foundation. \index{UniOne}  Working there I (Graham) saw how high some people were on the autism spectrum. The skills that they brought were highly valued, and the way that UniOne ran and was organised enabled them to be fully themselves and maximally productive. There's no reason why you can't do the same thing for yourself and your organisation.




\subsection{Life-threatening organisation designs}
Many approaches to organisation design\index{organisation!design}  are at worst toxic to the living organisation, or at least weaken it considerably. Even the modern approaches, such as Sociocracy, Holacracy, Agile, etc. can be perfect enablers of an organisation thriving as a living being; or can be toxic; depending on the meaning\hyp{}making stories behind how they are deployed. Some approaches more clearly recognise that the essence of an organisation is alive, similar to a beehive, with an intelligence and meaning\hyp{}making story way bigger than that of any individual; and others less clearly. 


The more the story creating an organisation design paradigm exclusively focuses on the visible structures and processes (activity variable of Equation~\ref{eq:living-organisation}, or lower right visible collective in Section~\ref{section:integral-organisation}), the more toxic it is likely to be to an organisation’s living, meaning\hyp{}making aspects (context variable, or lower left hidden collective in section~\ref{section:integral-organisation}). 


And if the powerful in an organisation impose a story on a quadrant that fails to add value to any of the other quadrants, let alone suppresses it, the organisation as a whole will suffer. 


The meaning\hyp{}making story dominates, as you saw in Equation~\ref{eq:living-organisation}.


Some Ocracies  recognise this more clearly. The human being is always seen to be speaking from a number of roles, including the permanent role that all humans have in the living organisation, as its senses and voice. And especially that, in a living being, very often what is happening is so nebulous that there is no rational way of putting it into words; it can only be felt. You feel when something is emerging or is wrong; and then you must be able to object to a proposal, not because you have a rational reason why not, but simply because it feels wrong to you in your core role as the living organisation’s representative. 


This is the same as recognising that you are not primarily your bones, muscles, and energy supply. You are primarily the meaning\hyp{}making stories\index{meaning-making stories}  that you use to shape the unique reality you experience.
\section{Interpersonal arena}
\label{section:interpersonal}
Delivering results together is the difference between a high-performing team, and the same individuals who happen to be in the team. Doubtless you can think of sports teams who didn't have any specific star player, yet consistently outperform teams that have a couple of expensive Prima Donnas yet were unable to play as one team.


To understand where your organisation is on a scale from zero adaptive capacity to full-spectrum adaptive capacity, look at how effectively all types of interpersonal conflict can be immediately and rapidly surfaced without any blame or shame. Just as in nature's evolution, the task of everyone in the organisation is to mine the interpersonal conflict for the valuable information it's giving you on how to maintain or even increase your fit to the external drivers.


In Chapters~\ref{chapter:who-am-i-base} to~\ref{chapter:who-am-i-one} you dived deeply into who you are and how you are anchored in the meaning\hyp{}making stories you have internalised through your life. You learnt how to iteratively transform your identity to get ever better alignment with actuality. This chapter will hopefully whet your appetite to master those chapters even better, because they are the prerequisite for the interpersonal layer of an adaptive living organisation: superb capacity for working effectively with the current self-identities of all stakeholders, and using work for further self-development.


For example, take the work of Brene Brown \index{Brown, Brene} (e.g. \emph{Rising Strong} or her TED talks) on blame and shame. Any level of blame or shame directed towards you for being who you are, or doing what you have done, reduces your adaptive capacity, and so reduces the results you can deliver for the business. If you are leading an organisation, especially if you are leading a change towards any or all of an Adaptive Organisation’s \index{Adaptive Organisation} elements, every single time you blame or shame someone pushes that person deeper into their Job~2, leaving less energy available for a success.


Everything in this section revolves around understanding, and then increasing, your individual capacity working with your colleagues, and the capacity of your organisation as a whole, to mine and refine tension and conflict. Doing so will extract the nuggets of gold telling you about who your organisation currently is, and who it needs to become, to perform in your current context.


\subsection{The physics and art of work teams}
Creating a high-performance team and organisation is as much physics as art. The physics has much more in common with relativity and quantum physics than it does with Newton's\index{Newton, Isaac}  classical physics. The art has much more in common with Picasso \index{Picasso, Pablo} than it does with a traditional landscape painter.


There will always be aspects that are nebulous, that you cannot know about. Aspects where, as soon as you do something to know, your doing will change the very aspect you need to learn about. So you'll still not know!


If you are a business leader, there is no way back to a safe and predictable life. If your business does not deliver output, and deliver it efficiently, it will soon die. Businesses exist to deliver what is needed by society, within the context of their niche in society. Everything else is there to support delivering. This output can span a very wide range, not just the number of widgets you make or your total delivered cost of making them. Also, always remember that profit and shareholder returns are not primary business results. Profit and TSR is a consequence of delivering excellent business output with high efficiency.


Because the first seven years of my (Graham’s) working career was in theoretical high-temperature particle physics, I naturally have looked at everything I've touched since then through the lenses of a particle physicist. I am always asking myself how the final performance of an entire business can be improved by working on the smallest particles that make up that business, and their interactions or forces between them.


I've long seen that external regulations to change the behaviour and output of business sectors are much the same as changing the properties of, for example, a bucket of water by heating it, cooling it or putting it under pressure. Of course, that can change the bucket of water into a block of ice or a nebulous cloud drifting away in the wind, but you're not going to create a diamond by doing that.


If you need a diamond, you need to change the individual particles from water molecules to carbon atoms, and you need to change their interactions, from weak Van der Waal's to strong covalent\footnote{Van der Waal’s forces are very weak forces between molecules together; they are easy to break, which is why water melts at 0\textdegree{}C. Covalent forces are way stronger, and hold atoms together to form molecules, which is why water only breaks up into hydrogen and oxygen with both a catalyst and 500-2000\textdegree{}C temperature. }. The same is true in your organisation. If you need your organisation to deliver better results, you are far more likely to get there with less effort by working on the individual elements, how they interact, and the overall context they are in, and trying to work via pressure and temperature.


Research in high-performing teams supports this.


Take the difference between a diamond, and the graphite in a pencil. Although both are the same pure carbon atoms, the relationships between the atoms, how they are arranged, and the kind of interaction that results from that, differs. The diamond is extremely hard; the graphite is soft and slippery. In graphite the individual carbon atoms only interact strongly with neighbouring carbon atoms in their 2D plane, and outside the plane through weak interactions. In diamond, they interact strongly in all three directions.


This is a lot like many organisations. Team members\index{teams}  only interact strongly with those team members on the same team or silo, and weakly with anyone else. I know this only too well from working within P\&G; although I believe that P\&G \index{Procter and Gamble} was one of the best companies around at recognising this problem and actively enabling individuals interaction across the entire organisation.


A diamond is quite different. The entire diamond, no matter how big it gets, is one single crystal. The carbon atoms interact strongly with all the carbon atoms they can. This is what makes a diamond so strong.


Develop high-performing teams as a physicist would. Focus on the interaction between people, and support each person developing their skills, their fluidity for transformational thinking, and their meaning\hyp{}making stories.


\begin{longstoryblock}
During the early start-up days at Evolutesix\index{Evolutesix}, our focus was on developing the Adaptive Way component of the Adaptive Organisation.\index{Adaptive Organisation}  Two colleagues, Marko Wolf and Adrian Meyer joined me in London for two weeks of intense work trying out the Adaptive Way prototype on ourselves, and improving it to bring to others.


After ten days of hard work, with just a few days left to finish everything, we decided that working in a different environment would do us good. So we all headed off to a nearby National Trust stately home and continued working while walking through the grounds, and enjoying tea and delicious scones in the café.


Approaching midday, the discussion became both heated and drained of energy. In particular, between Adrian and myself. Marko put his hand up and interrupted the conversation, asking \begin{quote} Which meaning\hyp{}making stories are active in each of us right now? \end{quote}


I said \begin{quote}you guys have both given a significant amount of your precious time and money in coming here, I must honour your gift by giving back far more value through what we develop in the Adaptive Way. \index{Adaptive Way} And, my meaning\hyp{}making story is that what I already know and can do is not very valuable, so the only way I can honour your gift is by giving you in these few days \emph{everything} that I have mastered over the past years.\end{quote}


Adrian replied, with a sense of immediate ease and a grin of recognition, \begin{quote}The story alive in me right now is a typical story for many consultants\textemdash we insecure over\hyp{}achievers. I'm looking at our relationship as an archetype client-consultant relationship, so I need to take whatever you give me, digest it, and give back to you something that is more powerfully insightful and helpful to you. So every hour that we spent talking, stays with me as an additional hour of digesting and processing, and perhaps an hour discussing that with you. In addition, I also judge that whatever I am able to do is never quite enough, and prize achievement very highly. This combination of insecurity and over achievement means that I only begin to think that I've done enough long after I have actually done too much. The more you give me, the worse my life becomes.\end{quote}


Marko was grinning broadly by now. \begin{quote}I probably don't need to tell you what's happening inside me; you know it already. My background is philosophy and other aspects of the human dimension, I don't have a business background. This conversation is triggering in me stories of not having enough value to contribute, and questioning my place. 


So the more that I see the two of you ratchet everything up, the more I judge myself as having no place here, but very much want to have a place here, and so look for anything and everything I might possibly contribute to add to what you are doing.\end{quote}  


All three of us grinned, realised how each of our vulnerabilities, and our strategies to protect our vulnerabilities, were colluding with each other to make things worse for us all. Despite us all thinking that what we were doing was our very best to achieve our common goal of creating an Adaptive Way program sufficiently polished to bring to clients, what each of us was actually doing was far more about job~2, defending our self-esteem. (Notice how the stories pick up on a number of the biases of section~\ref{section:biases}, such as reciprocity bias, social proof, and commitment.)


At that point, we realised three very important things. 


Firstly, that we had just experienced the power of the Adaptive Way in action. Even though we were still relatively clumsy at using the Adaptive Way, within ten minutes of Job~2 kicking in, we had registered that our emotions were telling us that something else was happening. Recognising the tension that each of us was experiencing, we shifted gear into understanding what was actually happening, and using the Adaptive Way\index{Adaptive Way} to use that data to get us to focus back on the result we were trying to deliver together.


Secondly, each of us realised that we had just done an impromptu experiment to challenge our meaning\hyp{}making stories. We had learnt just how much our meaning\hyp{}making stories were giving us a distorted, dysfunctional reality. We had just had an experience showing that a more functional story was that each of us was bringing in more than enough value to settle everything.


And finally, Marko reminded us of the objective we had set ourselves on the first day. 


It was immediately apparent to each of us that we had actually passed that goal two days ago. What we were doing now was a combination of perfectionism and developing the next module. In the absence of any consumer data to tell us what was actually better or worse, and what was needed next! 


We immediately decided to stop, celebrate what we had achieved, and go out and test with real consumers the current stage of our Adaptive Way prototype.
\end{longstoryblock}




\subsection{Relationship scaling}
\label{relationship-scaling}
\index{relationship scaling}
When I was at school, maths exams regularly had questions along the lines of \begin{quote} if a team of 10 people take 10 days to dig a trench to lay an electricity cable, how long will it take a team of 20 people to do the same job? 
\end{quote} 
At school, you get full marks by saying it would take 20 people five days.


In business, though, it is quite clear that the answer is often 20 days. 


Jeff Bezos and Steve Jobs are both known to favour small teams. Amazon\index{Amazon} has a metric for team size based on how many pizzas it takes to feed the team for a meal. More than two, and the team should be split. Why is it that individuals in larger teams deliver weaker results than the same people do in smaller teams?


One reason is relational loss\cite{mueller-why-larger-teams-perform-worse}. In a larger team, each individual's uniqueness and unique contribution becomes less visible to all others, and each has less support and interaction than they need with key people. This is also linked to social proof bias,\index{biases} Page~\pageref{section:social-proof-bias}. 


The more people there are in the team, the more likely somebody is to keep quiet when they see a risk, or not take action when they see an opportunity, especially if that means acting alone or convincing everyone. The bigger the team, the more courage it takes to go against the group opinion, especially for the many at S3 or S2.


There is also information overhead; the effort needed to get all necessary information to and from everybody, coordinating everybody, and maintaining psychological safety. This scales nearly exponentially because the number and complexity of relationships in the team scales nearly exponentially. You can see how this goes as follows. The left column is the number of people in the team, and the right column is the number of relationships of each size in the team.


\begin{table}[h]
        \centering
\begin{tabular}{c|cccc|c}
\toprule
\textbf{No. people} & \textbf{Self} & \textbf{Dyads} & \textbf{Triads} & \textbf{Quads} & \textbf{Total} \\
\midrule
1  & 1  &   &   &   & 1 \\
2  &  2 & 1  &   &   & 3 \\
3  &  3 &  3 & 1  &   & 7 \\
4  &  4 &  6 & 3  & 1 &  14 \\
5  &  5 & 10  & 10  & 1  & 26 \\
7  &  7 & 21  & 35  & 15  & 78 \\
10  &  10 & 45  & 120  & 126 & 301  \\
\bottomrule
\end{tabular} 
\caption[Relationship scaling]{Table showing how the number of each type of relationship and the size of the largest relationship scales with team size. You have the relationship of each person with himself, each group of two, three, and four people, and so on.}
\label{table:relationship-scaling}
\end{table}


You already get the picture looking just at dyads (pairs), triads (threes) and tetrads (fours). 
As you can see from the Table~\ref{table:relationship-scaling}, as your team size grows, your relational overhead to maintain high performance grows nearly exponentially\footnote{If you want to extend this table for your team, the formula for the number of different combinations of $k$ people out of a team of $n$ people, the Binomial Coefficient, or $n$ choose $k$, is $\frac{n!}{k!(n-k)!}$.}. 


The perceived loss of support in larger teams coming from relational loss means that larger teams are even more stressed in rising to adaptive challenges than they are in rising to technical challenges. 


Simply adding three more people to a team of seven increases the number of relationships nearly four times. So if you want to take a team on a journey towards becoming an Adaptive Organisation, \index{Adaptive Organisation} you are far better off doing it with small teams of fewer than ten people.


Relational overhead depends on the absolute number of people in the team, not on how many days each person is working in the team. This is why startups are far better off, once they go beyond four people, to have a smaller number of people working full time, rather than a large number of people working part time.




\subsection{Interactivity and reciprocal helping}
\label{section:reciprocal-helping}
\label{section:interactivity}
\index{interactivity}\index{helping, reciprocal}


\begin{longstoryblock}
I (Graham) co-led, during my time with P\&G China, \index{Procter and Gamble} one of the best teams that I have ever worked in. It delivered excellent business results and work was fun. It was easy to feel that this was a high-performing team because there was a constant crackle of excitement and energy whoever I was interacting with. And I was interacting with everybody.


We were tasked with fundamentally reinventing the laundry process for our low-income markets. Hari Nair, my counterpart in product research, and I were motivated by the impossible deadlines to get the first minimum viable prototypes done in time for the next scheduled senior management meeting and launch window.


We brought in the global design company IDEO, and the Indian innovation and marketing consultancy, Ray+Ke\-she\-van. The combination of these external partners and our internal cross-functional team members lead us to repeatedly do the impossible. Where typical project teams took four or more weeks, we were taking two or fewer.


We knew it was a high-performing team because you could feel its energy in every interaction, whether we were together in a meeting in one of our offices in Beijing or Singapore, or with the consumers in the Philippines who were testing the minimum viable prototypes, or sitting down together in the evening over a meal and a few drinks. 


Two other significant characteristics were that we were in a small core team of seven, working either full time or more than 50\% on the project; and everybody was talking to everybody and helping each other constantly, at work and outside work. 
\end{longstoryblock}


Collaboration is reciprocal helping, as Edgar Schein \index{Schein, Edgar}describes in his book \emph{Helping}\cite{schein-helping}. Reciprocity bias \index{biases} has been core to humans thriving, enabling us to bond together into collaborative groups (Page~\pageref{section:reciprocity-bias}). The more performance you expect from your team, the more reciprocity in helping each other you need. And all reciprocal helping depends on how well everyone in the team is able to interact.


Research on which characteristics correlate with high-performing organisations show that the more staff interactivity there is, the higher the performance\cite{pentland-new-science-great-teams, pentland-hard-science-teamwork}. This is one of the few clear correlations with performance across many different measures. It’s only natural when you look at an organisation as a living being, where the reality experienced is shaped by the whole organisation’s collective meaning\hyp{}making stories. The more everybody in the organisation interacts, the stronger the stories become, and the easier it then becomes for people to collaborate.


So if you want to evaluate your organisation, or one that you are about to be hired by or consult for, have a look at the interactivity. The less there is, the worse the business results will be compared to the potential, because the reciprocal helping will be lower. If you are leading an organisation, and you want better results, one area you have to take loving care of is the amount of interaction between members of your organisation and the quality of that interaction.


Surprisingly the research showed that it didn't matter much whether the interaction was while working together in project teams on specific tasks, or conversations over lunch, or tea breaks with colleagues you don't normally collaborate with. More important was how much interactivity there was in the team in total, that every person was interacting with every other person. How people were interacting was what mattered, not the content of their interaction. Again, looking at an organisation as a living being composed of lots of people like you, bonded into a community, it's no surprise that our strengths as social human beings are still the dominant strengths in our companies. 


What makes it so difficult for people to interact effectively, harnessing the power of reciprocal helping to deliver the business strategy with excellence? If I think back to my time with P\&G, I mostly found it very easy to collaborate with colleagues across the organisation. It's very clear to me that all my successes lay in people who I had helped then helping me in return, or vice versa. Reciprocity bias\index{biases} bonded us together into a very powerful, informal, flat organisation that tunnelled through all the barriers between silos and hierarchical position. Much like quantum physics, the real power of P\&G \index{Procter and Gamble}to create disruptive innovation lay more in this emergent, hidden, nebulous human capability hierarchy, than in the visible management accountability hierarchy.


One of my first actions when I took over responsibility for a section in the Chinese organisation was to invite one of the best-connected women in P\&G, Kathy Felber, to rapidly ramp up the amount of interactivity between my fledgling Chinese R\&D section and the rest of the world. The team was so hungry for this that she returned home sucked completely dry. Kathy was also leading the annual week-long training programme, and because of her visit to China and getting to know just what the team there were capable of, she then started bringing people from the Chinese organisation to deliver training in the US, and ironed out some barriers for the new Chinese recruits attending training in the US.


Interactivity, and specifically reciprocal helping, only exists when there is sufficient psychological safety. Collaboration within an organisational hierarchy is one of the most psychologically contorted arenas there is, because, if I help you, then psychologically we're stepping into a parent-child transactional relationship, with me is the parent and you as the child. If ten minutes later you are then helping me, then our relationship inverts to one where you have the role of parent and I have the role of child.


It works when the formal boss is helping the formal subordinate, as then the relationship is aligned formally and psychologically. But when the formal subordinate is helping (=leading) the formal boss, the two relationships oppose each other.


All of us function best when we are in a constant relationship with another person. We have difficulty functioning when the very relationship itself is emergent, flipping from one type to another according to the context that we are in, independent of both people in the relationship. Given our very normal human needs for stability and consistency, this kind of nebulous relationship triggers our psychological defence mechanisms, our Job~2.


This is also what makes it so hard for some bosses to collaborate effectively with their subordinates, especially those yet to reach S4.


How much psychological safety do you experience in your organisation? If you are in a leadership position, and someone very junior in the hierarchy offers to help you do something that they can see you need help in, are you able to accept their initiating help? 


If you are in an Ocracy, how easy is it to truly interact with your colleagues and help each other reciprocally? Whilst that is at the heart of how these organisation designs are meant to work, in some organisations the individual stories\index{stories} of each person prevent them from doing this kind of relationship flipping.


There is no better piece of evidence for the way that any kind of interactivity increases reciprocal helping the research of Pentland\cite{pentland-new-science-great-teams, pentland-hard-science-teamwork}.


\begin{quote}
\ldots we advised the center’s manager to revise the employees’ coffee break schedule so that everyone on a team took a break at the same time. That would allow people more time to socialize with their teammates, away from their workstations. Though the suggestion flew in the face of standard efficiency practices, the manager was baffled and desperate, so he tried it. And it worked: [average handling time] fell by more than 20\% among lower-performing teams and decreased by 8\% overall at the call center. Now the manager is changing the break schedule at all 10 of the bank’s call centers (which employ a total of 25,000 people) and is forecasting \$15 million a year in productivity increases. He has also seen employee satisfaction at call centers rise, sometimes by more than 10\%. Any company, no matter how large, has the potential to achieve this same kind of transformation.
\end{quote}


Pentland identified three aspects of the interaction between team members. These will be no surprise to you, now that you look at your business first and foremost as a living being.


\begin{description}
\item[Energy] The most important is energy. \index{energy} No surprise there! The more exchanges there are, and the more that these exchanges raise the productive energy of individuals and their relationships, the better the productivity. The more that dialogue patterns, such as deconstructive dialogue on Page~\pageref{section:deconstructive-criticism}, are used to give each other the scaffolding you need to stretch deep into your respective Zones~2 and~3, the more productive you will be. The more that you can use the ground pattern yourself to transform your meaning\hyp{}making stories and increase your adaptive capacity, the more productive you will be. 


And the more that all of these combine to reduce the amount of energy you put into unproductive Job~2, the better the business results you and your company will produce. This is where bringing in the radical transparency of ongoing regard and deconstructive dialogue can really drive your productivity\cite{mueller-why-larger-teams-perform-worse}, provided of course it emerges naturally because you have a psychologically safe space. Not because you mandate with power that everybody uses these patterns.


\item[Engagement] The more evenly distributed the activity and energy is across all the team members, and all their relationship combinations, the better the performance.


\item[Exploration] The more that there are team members who go off on random explorations outside it, the better the overall performance. So make sure that your team has one or two people with naturally high explorer energy, and recognise that the issues caused by the times they are not there, or their struggle finishing tasks, is a small price to pay for the overall boost in performance that they bring.
\end{description} 


Engagement and exploration are somewhat like complementary pairs in quantum mechanics. The more you are one, the less you are the other. But, truly high-performance teams find a way to switch themselves from the paradigms of either or into the paradigms of both. 


\begin{longstoryblock} 
This is one of my (Graham) personal challenges. I have a somewhat introverted nature and a high explorer energy. So it's very easy to go off exploring, not so easy for me to finish things nor to regularly and consistently interact with everybody. So I block off chunks of time for structured interactivity and plan in limits to how much time I spend on that, limits to how much time I spend on exploring, and layer them like a slice of mille-feuille cake.
\end{longstoryblock}


Seeing an organisation as a living, meaning\hyp{}making being composed of people like you, human meaning\hyp{}making beings, the sequence of energy, engagement, and exploration probably seems quite obvious. The research of Pentland and his team now shows just how much of a performance boost comes from this\cite{pentland-new-science-great-teams, pentland-hard-science-teamwork}, with the following hard data coming from their research. 


\begin{itemize}
\item The number of face-to-face interactions accounts for a full 35\% difference between the high and low performing teams. 


\item Up to one interaction within the team every five minutes increases performance; beyond that, it starts dropping. 


\item In a typical high-performance team, whole team interaction uses around half the time spent interacting, the rest is in pairs and a little in triads. 


\item When speaking in the group, members use time efficiently and fairly across the group, primarily listening and not speaking. 


\item Social time is a core driver of team performance, often accounting for more than 50\% of positive changes in communication patterns.
\end{itemize}


The interactivity research has also confirmed from a different perspective that talent is overrated. Getting smart, talented people onto your team is far less a driver of productivity than getting your team to increase their interactivity. 


The problem with stars is that they tend to talk more than others on the team, and tend to listen less, both of which reduce interactivity and performance. Even more so, highly charismatic people are only helpful to performance if they use their charisma to increase the quantity and energy of interactivity between all team members by facilitating it, rather than driving it themselves.


The research shows that interactivity is at least as important as all the other factors together that are typically believed to drive team performance. So if you are starting up a company, and raising the level of interactivity is not one of your topmost activities, you better have somebody really good supporting you in this area. The big reason to be hopeful, in your capacity to build a highly productive Adaptive Organisation\index{Adaptive Organisation}, is that interactivity is a learnable skill, as well as an adaptive challenge.


This is also a very important lens\index{lens} to look through as you choose which approach to dynamic organisation design to deploy, or to create your own unique flavour. Anything that reduces the human energy in the interactions between people as human beings will push towards lower business results. Do as nature does, and find highly creative ways of integrating all the different elements and layers of the living system, in ways that maximise the human energy that is transformed into business results.


\subsection{Stories and culture}
Since the essence of a living being is the meaning\hyp{}making stories that shape the reality experienced, an Adaptive Organisation\index{Adaptive Organisation} needs to go beyond the level of psychological safety and interactivity needed for reciprocal helping in an organisation that is designed to deliver excellent business results while only facing technical challenges. An Adaptive Organisation needs the step change in psychological safety and interactivity to make the hidden stories and culture visible, to become a deliberately developmental organisation that deliberately develops its collective meaning\hyp{}making stories.


In your organisation, do you feel that it's safe to talk openly about your feelings? If not, your organisation has its hands tied behind its back in trying to rise to adaptive challenges. As you read in Chapter~\ref{chapter:who-am-i-nature} your true feelings are hard data that will tell you if something is not right. (Caution if you or other people are using judgements dressed up as feelings; these are distorted data. They come from your unique inner reality, generated by your meaning\hyp{}making stories attributing meaning to a small biased segment of actuality.)


The more playful your organisation is, the more likely it is to grow and adapt. If you are a playful entrepreneur, evidence is that you are more likely to thrive\cite{dodgson-playful-entrepreneur}. 


A playful South African company, Blueprints, \index{Blueprints} has pioneered a way of making the desired stories visible and memorable to all in a playful, visual and intuitive way. These are captured as a measurable equation that can be monitored monthly to track how well the business is living its stories. 


The BBC\cite{bbc-humour-nazi}  successfully used humour during WW2 to counteract the internal Nazi propaganda, to keep those Germans who were against the Nazi regime connected with themselves and the common ground all humans share. 


\subsection{Thinking and deciding together}
A few decades ago the airline industry realised that the interpersonal arena was where a number of accidents and near misses were taking place. The worst happened on 27 March 1977 on the Spanish island of Tenerife. 583 passengers died when a KLM 747 collided with a Pan Am 747. 


This crash might have been prevented if the KLM pilot and co-pilot had had the right interaction in the interpersonal arena. Instead of using all the information they each had to think and decide together, the captain overruled the co-pilot and flight engineer. Most likely authority bias played a significant role, because the captain was the head of the KLM pilot training, with the authority to pass or fail the co-pilot in the next training session.


The consequence was big changes in standardising the interactivity between flight crew members and the control tower. In particular, authority bias is explicitly counteracted by the elimination of all authority-reinforcing patterns of interaction. For example, in some languages the word \emph{you} comes in two or more variations, depending on relative rank and authority. Flight crews today are not allowed to use anything except the rank neutral peer-to-peer version of \emph{you} if they are not speaking English to each other. 


The more the arena of interactivity is psychologically safe, \index{psychological safety} the more that everyone can bring to bear their most powerful level of transformational thinking fluidity, and the better the decisions.




\section{Personal and interpersonal arenas}
We put these two arenas together in this chapter because you use the same dialogue-based “mining technology” you have covered in the ground pattern of Section~\ref{section:ground-pattern} to extract the valuable information buried in human conflict, whether the conflict is within you, or between two or more of you.




\subsection{Deconstructive dialogue}
\index{deconstructive dialogue|(}
\label{section:deconstructive-criticism} 


For the ground pattern to work at its most powerful, you need to introduce deconstructive dialogue and ongoing regard. These are the two most powerful tools to mine and refine the valuable gold from the raw conflict you have with your colleagues. Deconstructive dialogue can also be called deconstructive criticism, feedback, or exploration, depending on the specific situation.


\begin{longstoryblock}
A colleague once gave me some constructive criticism, saying \begin{quote} Graham, at times you can be such an intellectual terrorist. \end{quote} I immediately felt both anger and fear, triggering thoughts of fighting back, themselves a consequence of the worry that I might lose my belonging to the group. Since I was co-founder and MD that was not likely, but even the slightest hint of a threat and the guardian angels put their ballistic missiles onto red alert. 


He then proceeded to give me some well-intentioned, but actually harmful, feedback. Harmful for me personally and for the company, because it was just not actionable by me, and significantly reduced my capacity to lead. All that his feedback did was reinforce the very meaning\hyp{}making stories that were leading me to behave that way.


Now, a decade later, I know how to translate that feedback into developmentally useful language using the ground pattern, especially by embedding clearly my needs, true feelings, and judgements into the translation. Back then I was still learning how to do this, and whilst I did OK, it was still a catch-22. 
\end{longstoryblock}


Take care when you are giving someone feedback: you are not a god, you lack the ability to define who someone else is. Rather, when you give someone feedback you are partly\textemdash maybe even only\textemdash talking about yourself and your own meaning\hyp{}making stories. What you are actually saying when you give feedback is that when someone else does and says certain things your internal meaning\hyp{}making stories attribute, or create, meaning (e.g. “intellectual terrorist”). Rephrase what you normally say using the clean language of NVC\cite{rosenberg-nonviolent}: what was actually said or done in pure observable facts; how you feel; what you think; your judgements cleanly distinguished from your feelings. 


Use the opportunity to both express your clean feedback\index{feedback} and to get information on how you generate reality through your own stories. Then both of you can explore your respective meaning\hyp{}making stories (and your natures and forms of thought) as part of the discussion. You can work together to uncover how your stories and natures are banging into each other, like two icebergs, hidden from your sight. Through that you can both figure out how at least one, and most likely both, can grow, and perhaps you can figure out a joint castle move that will keep both of your hidden commitments satisfied.


A deconstructive approach is hugely powerful and minimises the risk of making a grave error when you are giving feedback in a power hierarchy.


This approach is also extraordinarily effective when you are in a power game. By actively exploring, or deconstructing, how you and the other person are making meaning, you will take distance from yourself and the other person, helping you remain centred. You will continue to see them from a compassionate stance, as full human beings, with their own unique set of big assumptions driving their behaviour. You will remain clear on what is yours and what is theirs.


We often look at criticism and encouragement as a polarity of being fed carrots or hit with sticks. Deconstructive criticism opens up ways of being hit with a carrot and then eating it, of seeing complementary pairs when before you only saw exclusive opposites.


\begin{longstoryblock}
One of our Adaptive Way \index{Adaptive Way} participants, Sharon, is in the leadership team of a 50-person company. She recounted a situation she had had on the very day we covered deconstructive criticism for the first time.


\begin{quote}Three people in my team had simply not done what they should have done during the week. Nor did they communicate in advance that there was an issue. But in our meeting last week we all agreed that we would do it, that we would bet on the 50\% probability of getting a ‘go’ for the product launch. Not on the 50\% risk of a delay. I criticised their lack of ownership, their laziness, and their general gap as middle managers in the company.\end{quote}


As we explored what might be happening, in particular what the three people’s stories might be, and how they might be banging up against Sharon's stories, she had a big insight: \begin{quote}I know they are uncomfortable making assumptions. I suspect that no answer is better for them than a wrong answer based on false assumptions. Also, up until now I have stories around avoiding conflict, so I have never openly challenged their fears. More, I may have reinforced the danger of giving a wrong answer that triggers conflict with the founder.\end{quote}


Sharon, a few days after the dojo, invited the three for her first attempt and led them  through deconstructive dialogue. In the following dojo she reported back saying \begin{quote}It was really clumsy at first, but now we have bonded together far more strongly as a team, I understand far better how to help them deliver, and they understand far better how I see the world. We are wasting less effort, have less stress, and are delivering better than ever before.\end{quote}
\end{longstoryblock}


Deconstructive dialogue is not a simple language pattern. Rather, it is a lens of exploration, a lens to look at yourself and the other person through, where you use everything in the dojo to get the best possible grasp on what is actually happening, at all levels, clearly distinguishing between yourself and them. Deconstructive dialogue is very much applying the core lenses of quantum physics\index{physics!quantum} and Cubist art \index{Cubism} to yourself and your interactions with your colleagues at work.   


Looking through the deconstructive lens at someone you are about to criticise, you see the following clearly:


\begin{enumerate}
\item Your view likely has some value. You may even be completely right that the other person is wrong. 
\item Your view may not be complete, and / or may not be accurate.
\item The other person's view may have some value. At least it makes sense to them within their set of meaning\hyp{}making stories and way of constructing their reality, and so through using deconstructive dialogue each of you can learn something about their meaning\hyp{}making stories.
\item Your criticism may come from a difference in constructing reality, and so cannot be effectively addressed until both of you understand each other. This can cover all their commitments, their hidden commitments, and their meaning\hyp{}making story.
\item So there may be multiple different and valid realities at play here, which means that their view is vital for you to evaluate how 'right' your criticism actually is.
\item Since your internal tensions are the driver for your growth, now you have the chance to use tensions inside and between each other to open up even more avenues for growing your capacity, both individually and as collaborators.
\item The whole focus is on exploring each other, or sparring, not on fixing the other person. This generates huge psychological safety, because both sides know, so long as they stay within the pattern of deconstructive dialogue, that the dangerous forces of blame and shame\cite{brown-rising} are excluded.
\end{enumerate}


Comparing deconstructive criticism to constructive criticism, you can see a number of differences. 


\begin{enumerate}
\item The biggest, and most important difference, is that in deconstructive criticism you both have the chance of learning something that leads to growth. Rather than only the other person learning what your reality imagines they are, or have done wrong. Which may have no relevance to their reality, or even actuality. It may be purely in your mind!
\item What you learn in deconstructive criticism comes from a more accurate view of all that actually is, rather than your own limited reality emerging from your own meaning\hyp{}making stories.
\item You see the other person as a whole person, you are outside your box\cite{arbinger-leadership}. You see them and their unique way of constructing reality, with their own meaning\hyp{}making stories; and that their way may be just as valid (or invalid) for them as yours is for you.
\item Instead of seeing the truth as something you alone possess (especially an issue if you are their manager, the lead link, or a subject matter expert, and identify with your role or expertise), you see the truth as something that spans all of: you, them, your relationship, and the external context.
\item In deconstructive criticism you see that quite possibly neither of you gets it, so you are a curious explorer; versus constructive criticism attempting to educate or correct someone who fails to get it, with you as a flawless teacher, or judge and jury.
\item We normally view tensions and conflict as an issue to manage away, whereas a deconstructive stance says \begin{quote}Yay, we have a tension! Vital raw material for our self-reconstruction activities. Now we both have even better chances of transforming ourselves.\end{quote}
\item Deconstructive applies all three Evolutesix dojo rules: care for yourself, the other person, and the whole. 
\item Deconstructive criticism is active, not passive. You go in with a spirit of curiosity; you are anchored in not knowing.
\end{enumerate}


Of course, the outcome of deconstructive criticism may still confirm that your point of view is completely correct. It may well still be, if you are accountable for who is on the team or in the company, that you still follow through your decision to, for example, fire the other person. Deconstructive dialogue is absolutely not about ducking your accountability. It is about living up to your accountability in the biggest possible way, maximising the number of options available to you to deliver better results by, at the same time, with no extra effort or cost, developing yourself, others, and your entire organisation from an adaptive perspective.


A final tip: I have found it very helpful to look at what anybody else does through the lens of them doing it for themselves, not against me. Even if what they have done works against me, they are doing it first and foremost to meet their needs, not to obstruct me in meeting my needs. 
\index{deconstructive dialogue|)}
\subsection{Ongoing regard}
\label{section:ongoing-regard} 
\index{ongoing regard|(}
One Adaptive Way \index{Adaptive Way} participant, Tania, had a central meaning\hyp{}making story: \begin{quote}If I make mistakes, I will feel vulnerable and insecure because it means I can't do a good job of making the best of my life, and then my life is not worth living.\end{quote} This big assumption drove Tania's behaviour strongly when deciding where to go on holiday but was absent when she took big life-changing decisions.


\begin{longstoryblock}
Tania's development took another step a few weeks later, sparring with William. He said \begin{quote}Tania, you are so strong, you don't need to worry about this.\end{quote} 


I interrupted \begin{quote}William, that comment sounded to me like it came from a need \emph{you} have to be \emph{seen} as helpful. It risked reinforcing Tania's big assumptions and potentially creating new big assumptions.\end{quote}


Neither had seen ongoing regard, so we went into a quick explanation of why it is actually harmful to praise someone with a “you are” statement. Such “you are” compliments are the root causes of many of our big assumptions later in life, because by praising somebody with the sentence stem \emph{you are}, you’re acting as if you have godlike powers to define who they are, when in reality all you can do is project your own making meaning\hyp{}making stories onto them.


William thought for a while, responding with \begin{quote}This has really made so much clear to me. About what has created my big assumptions. I will never say that again to anyone, and certainly no-one reporting to me at work or in my family. \end{quote}
\end{longstoryblock}


\emph{“You are so strong / clever / beautiful \ldots”} is an example of how so many of us typically try to help someone who doubts themselves, is in pain, etc. It is also how we commonly support someone developing a new skill, or a child growing up. 


Saying something like \emph{“I am worthy”} to ourselves, perhaps in front of the mirror each morning, is just the same as someone else saying it to you. It seldom improves your sense of self-worth for longer than a few minutes. 


Such statements are the origin of many of your big assumptions, \index{assumptions} one cause of imposter syndrome, that most common of challenges which many of us face. Most of the work you now need to do to deconstruct your big assumptions is a consequence of the well-meaning praise of people who care about you, and your own attempts to assert yourself to already be someone else, by saying \emph{You are \ldots .} 


\begin{longstoryblock}
When I pointed this out, saying \begin{quote}Tania, this kind of helping is actually harmful because it can reinforce the very big assumption you are trying to break down,\end{quote} she replied \begin{quote}but it feels so good, I got a definite lift from what William has just said to me.\end{quote} 


\begin{quote} Yes, \emph{I replied,} and next time you try to achieve your commitment of speaking from your intuition, your big assumption of avoiding mistakes will be even stronger. Because you now have strengthened your big assumption that you need to avoid anyone ever seeing anything in you that is weak or makes mistakes. You have become just a little more dependent on others for your self-identity. You are giving other people the power to define who you are, and your worth as a person. But no-one has the power of god to be able to do that sufficiently well.\end{quote} 


\begin{quote}This makes a lot of sense to me, Graham. I can see why big decisions have become easy, but small decisions are still quite hard. It's the small decisions that still trigger the big assumptions coming from other people's ideas of who I should be.\end{quote}
\end{longstoryblock}


This illustrates why we never ever do anything with the intent to help or fix someone in the Adaptive Way. Ongoing regard is the only language for feedback in an Adaptive Organisation. 


It also illustrates why it is so important to give people feedback, \index{feedback} positive and negative, in ways that clearly distinguish between the praise / critical feedback giver's own meaning\hyp{}making stories (which are creating what they value, and hence what they believe needs to be said in praise) and the praise / critical feedback receiver's nature, meaning\hyp{}making, and qualities. 


Ongoing regard is a form of nonviolent communication, whereas conventional praise is a form of violent communication. Violent, because it imposes your way of making meaning onto the other person as if your personal meaning\hyp{}making was objective actuality, rather than just your own personal internal reality. 


The pattern for ongoing regard has two forms: one when you are recognising the value you got from someone else's actions or words; the second for when you want to express your admiration of a quality you see in another. 




\begin{enumerate}
\item Recognising the value you perceive in another's words or actions. \emph{“John, thank you for spending an hour with me going over the proposal for the client, because it helped me see the potential pitfalls for the client.”}
\item Acknowledging the inspiration you get from another. \emph{“John, so often after talking to you I understand the issue far more deeply and see the complexity more clearly and simply. I admire the way you think and talk to me.”}
\end{enumerate}


Notice how much more effective this is than simply praising John by saying \emph{“John, you are so clever.”}


Ongoing regard is far harder than simple praise because you need to think about yourself and the other person. Which means it is invariably far more valuable to the other person. How often have you discounted praise because you know the other person has not really grasped you, but is just throwing out a standard phrase?


Most importantly, ongoing regard invites both parties into transformation. 


Each can use the ongoing regard as input to their own experiments to transform their big assumptions. Neither is at risk of the words reinforcing old big assumptions, nor at risk of new ones being created. 


After you have read all of this, you may be wondering whether coaching has a role to play in an Adaptive Organisation. \index{Adaptive Organisation} It has a role to play, but far far smaller than most would think; and keep in mind the warnings of Section\ref{section:empiricism}. The most powerful and efficient way of spending your money is to train everybody in the organisation to use peer-to-peer developmental practices like the Adaptive Way.\index{Adaptive Way}
\index{ongoing regard|)}
\subsection{Public agreements} \index{public agreements|(}
\label{section:public-agreements}


A public agreement is \emph{very} different to the rules and policies that are used in and outside work to force people to behave inside a narrow range, regardless of their meaning\hyp{}making stories. This is easiest to see in an example.


\begin{longstoryblock}
In one small Adaptive Way \index{Adaptive Way} dojo we had Rob, Tania and Cathy as participants, and myself as sensei. We brought the following hidden commitments into our sparring:


\begin{description}
\item[Graham: ]\emph{ I need to avoid any situation where I am not seen as the smartest person in the room.}
\item[Rob: ]\emph{ I need to avoid conflict in myself or with others. If I don't, we will all get emotional, and then I risk being deserted. So whenever conflict begins I feel discomfort and start smoothing it over.}
\item[Cathy: ]\emph{I must avoid being in the centre of attention, I must avoid showing off. If I don't then I'll feel lonely because they will all think I am arrogant.}
\item[Tania: ]\emph{ I must avoid making mistakes, or any situation where I might fail, not use my time wisely, or not take full advantage of all the opportunities in front of me. Otherwise I am a waste of space on the earth. }
\end{description}


It is clear that, if we had been colleagues, each of our hidden commitments would trigger the defensive blocking behaviour of each of our hidden commitments! Each of us would invest energy into Job~2, thereby taking energy and time away from delivering results.


Company rules, policies, etc., will not help. If anything they will drive the issues underground, making them show up to protect our vulnerabilities in ways that are even more damaging to our capacity for reciprocal helping, and with each of us suffering in silence. 


The following public agreement complements rules and policies. It gives a far better alternative for any behaviours anchored in nature or meaning\hyp{}making, which are vital for a developmental organisation of Level~2 upwards:


\begin{quote} 
When we have a work meeting we will give each other space to speak. Each of us will have a specific meeting nudger, who will ‘activate’ what the other might need to say, but be hesitating because of their hidden commitment. Once activated, the right to talk or not goes back to the person with the hidden commitment. 
\end{quote}


We also discussed how, over time, we might add more to it, like building in pauses so that each could listen to any quiet internal voice prompting some different action.


Most importantly, we would consciously support each other in running experiments to transform each of our big assumptions until they no longer had power over us. 
\end{longstoryblock}


Whenever there is something we see as ‘wrong behaviour’ there are two responses to maintain group stability and cohesion. Either we create laws, policies or rules to regulate behaviour, with the intent of punishing those who break the rules and rewarding those who follow them. Or we create public agreements, with the understanding and intention that they will be broken. When, not if, they are broken, we react by exploring with curiosity each other's meaning\hyp{}making so that we can each grow. The best way to do this is with the pattern of deconstructive exploration in Section~\ref{section:deconstructive-criticism}.


One reason why some organisations fail in their attempt to implement Holacracy \index{Holacracy} or sociocracy \index{Sociocracy} effectively is a lack of true public agreements for human interactions. Our society, especially in the Anglo-Saxon worldviews, is designed around policies, rules and laws, not public agreements.


By bringing in public agreements to an organisation or community, you provide part of the scaffolding needed for those who naturally gravitate to rules to better use deconstructive exploration instead of enforcing rules, and to extend their Zone~3 deeper into their developmental boundaries. 


Be careful to never use public agreements as if they were rules or policies. If a rule is needed, use a rule. 


The difference can easily be felt, but it’s hard to describe. If you feel pressure, shame, or a sense of blame, then quite likely public agreements are being abused as covert rules. 


Public agreements create a sense of curiosity and exploration, with more psychological safety to be who you are now, with all of your behaviours, so that you can safely explore who you can become. But they are only agreements so long as each person voluntarily agrees that they are useful to their own developmental path.


Any public agreement imposed by someone with more power, or perceived to have more power (whatever the source and legitimacy of that power)\textemdash rather than co-created by both parties based on their individual development paths\textemdash is a rule hiding behind the mask of a public agreement. 


Any sense of external force, reward, or punishment points at this actually being a rule, not a public agreement. Then declare it to be a rule. 


For example, 


\begin{quote} 
we will always arrive on time for meetings, or put a Euro into the biscuit tin 
\end{quote} 
is a rule. Even if you call it a public agreement. However, 


\begin{quote} 
we will strive to arrive on time for meetings, and when we do not we agree to explore, with a sparring partner who we feel safe with, what in each of us (nature and meaning\hyp{}making) and our organisation’s structures, processes, and culture led to late arrival; and even why arriving on time was important. 
\end{quote} 
is a public agreement.


What’s most important is that the public agreement has absolutely no promise to change behaviour, no punishment for not changing; rather, it is an invitation to explore what could grow. Maybe you change your behaviour now; maybe it takes five years to change a deep big assumption; or maybe the organisation needs to change. Or maybe we just need to accept that arriving late is an inherent downside of a highly valued strength\textemdash for example, the strength of optimism.
\index{public agreements|)}






\subsection{Group coaching, not individual}
\label{section:team-coaching}
\index{coaching!team|(}
In their book \emph{The Art of Possibility}\cite{zander-art} the Zanders described the difference between superb musicians who never quite manage to create music that moves people, and musicians who do. The musicians who do move people pay attention to the music’s long line, not each individual note. They recognise that it is sometimes necessary to play individual notes imperfectly in order to get the long line of music right, and through the imperfection of those notes communicate the more important underlying message and energy of the long line of music. Their music is alive. Those who never quite make it focus so much on playing each note with perfection that the long line of music lacks life.


Over the past two decades the individual coaching industry has exploded, both in the form of one-on-one executive and life coaching, as well as in the form of workshops run by highly charismatic individuals. 


Delivering business results has common ground with moving an audience with music. Just as that works better if the musician plays the music’s long line, not individual notes, so too is money usually better spent on improving the long line of the team as a whole, not each individual. Spending company money on wholesale individual coaching is at best an inefficient use of resources; and at worst harmful. 


If you are starting up a company, taking an existing company on a change journey, or just aiming to increase its performance by 10\%, you are usually better off spending the amount of money on everyone than on hiring an expensive coach just for the CEO. Get everybody better at using the ground pattern for themselves and peer-to-peer sparring with each other as part of their work: using deconstructive dialogue; and ongoing regard; to mine and refine the conflict between each other.


The best way of doing what good coaching is there to do\textemdash namely surface and grow meaning\hyp{}making stories\textemdash is as an integral part of work. As part of how colleagues collaborate with each other, i.e., reciprocal helping, where everybody recognises that an essential element of work is using what is happening today to grow capacity to do better tomorrow.


Where coaching individuals is most likely to pay off long term is strictly developmental coaching of key change agents, tasked with leading by example. Support them to increase their fluidity of transformational thinking. This will help them to shift their entire category of stories to the next stage of meaning\hyp{}making, and increase their capacity to manage their emotional state and subtlety working with their hardwired nature, so that they can embody the new way of being and interacting flawlessly. Ideally do this before they even begin their work.


As we’ve seen, research into high-performing teams and organisations is now showing that the level of interactivity between all team members is an excellent predictor of team performance. It is far more strongly correlated with high-performing teams than the presence of excellent talent in a few individuals. See for example this summary in Harvard Business Review\cite{coutu-why-teams-dont-work} from which I have taken the following:


\begin{quote}
… the HR department will set up training to (hire and) develop the “right” people in the “right” way. The problem is, this is all about the individual. This single-minded focus on the individual employee is one of the main reasons that teams don’t do as well as they might in organizations with strong HR departments. Just look at our research on senior executive teams. We found that coaching individual team members did not do all that much to help executive teams perform better. For the team to reap the benefits of coaching, it must focus on group processes. 
\end{quote}