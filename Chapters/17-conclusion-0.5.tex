% Changed Your / You to our / we in 18.3 yellow highlights; and Proctor to Procter; added index.
\chapter{Reasons for hope}
\addcontentsline{toc}{chapterdescription}{The climate emergency and other global crises are rightly cause for concern, and we are entering the last decade where humanity can do something. This is our best and last chance ever to create a regenerative economy that works for you, for our planet, and for profit. All we need to do is integrate what is currently separated.}




\begin{chapterquotation}
We ask ourselves, ‘Who am I to be brilliant, gorgeous, talented, fabulous?’ Actually, who are you not to be? \ldots Your playing small does not serve the world.\\
\raggedleft \textemdash Marianne Williamson   
\end{chapterquotation}  \index{Williamson, Marianne}


\section{Stories of hope}
\index{hope!stories of}
You've read in this book how the meaning-making stories of separation created the identities of each distinct grouping in South Africa. These meaning-making stories were distilled and internalised over four centuries of events experienced in different ways by different groups; answered for each two central themes: \emph{who am I?} and \emph{what am I for?} and defined everyone’s actions and their outcomes.


From the first Europeans\footnote{The story is far, far older. It goes back to when the first Homo Sapiens began moving across Africa, into Europe and the rest of the world, into territory previously held by others; and then migrating back again generations later.} pushing those living in what is now Cape Town ever deeper into the desert; the wars with and suppression of the local population, as the Europeans moved deeper inland; the many experiences of loss of freedom and suffering, including the concentration camps, invented by the British during the South African wars to keep the Boer families under control; and the many experiences on all sides of trust offered and broken. Countless experiences were distilled into the interlocking meaning-making stories that created the reality of apartheid experienced by all South Africans. The reality of separation. The loss of what unites all into the oneness of being human, being alive. Of just being.


All these were distilled and internalised into the meaning-making stories that told South Africans of all groupings what to do, and how to behave.


The transformation of South Africa to the post-apartheid Rainbow Nation was a reality created by new meaning-making stories of hope. Stories that brought onto an equal footing the oneness that united all South Africans, and the uniqueness of each, to create an awesome new South Africa.\index{South Africa}


These new meaning-making stories were clear in three examples:


\begin{enumerate}
\item Then vice-president, later President, Thabo Mbeki \index{Mbeki, Thabo} addressed parliament on 8 May 1996 during the passing of the new constitution\cite{mbeki-I-am-an-african}. 
\begin{quote} 
I am an African. I owe my being to the hills and the valleys, the mountains and the glades, the rivers, the deserts, the trees, the flowers, the seas and the ever-changing seasons that define the face of our native land. \ldots
I am formed of the migrants who left Europe to find a new home on our native land. Whatever their own actions, they remain still part of me.
In my veins courses the blood of the Malay slaves who came from the East. Their proud dignity informs my bearing, their culture a part of my essence. The stripes they bore on their bodies from the lash of the slave master are a reminder embossed on my consciousness of what should not be done. \ldots 
\end{quote} 


\item Then deputy president and former president F.W. de Klerk \index{de Klerk, F.W.} replied, beginning with the words: \begin{quote} I am also an African. Although my people came from Europe more than 300 years ago, I became an African\dots\cite{deklerk-I-am-also-an-african}\end{quote}


\item Then-president Nelson Mandela \index{Mandela, Nelson} at the rugby world cup in South Africa wore the South African rugby jersey. A jersey which had, until then, symbolised rugby as the sport of white South Africans, began being reclaimed as a symbol of all South Africans.
\end{enumerate}


These three illustrate how rewriting the meaning-making stories of identity began shaping the peoples of South Africa into one story: \emph{We are Africans because we identify as Africans. And not for any other reason.}


The central theme of this book is that we can, and we must, take ownership of the stories that have shaped, and are shaping still, the reality we are experiencing. Then, owning our stories, rewrite them into stories that will shape a viable future realities. Realities where each of us thrives individually in our uniqueness, all of us thrive together, and where the entire ecosystem of life on earth thrives. Either all life in this interlocking ecosystem thrives, or none of us humans thrives.


You can take ownership of your stories\index{stories}, and how they shape your reality. You can take ownership of which forms of thought you are able to use, and you can take ownership of working subtly and compassionately with unchangeable aspects of your nature and history. Regardless of how small or big the arena of your stories and actions is, begin today re-crafting those stories to ones of hope,\index{hope} to ones of your uniqueness and our common oneness, and to ones that rewrite the stories of business, capitalism, and our economy.


\section{Memories, connections, and emotions}
We have been here before. Over the past four centuries since the Enlightenment we have been through a series of disruptions filled with a mix of extreme emotions, from irrationally exuberant hope to anxiety. When cheap coal and steam engines were developed in the mid-19th century, there was a wave of collective optimism that swept through many people: coal and engines would finally free us from all of our problems. Oil and gas wells flowed, lighting our cities just in time to save the whales (hunted for oil to light homes) from extinction.


For a century, many had their hope validated. Coal, and later oil-fuelled engines, generated disruptive innovations, and eliminated that era’s pending environmental catastrophes. Now they are just the opposite, with horrific unintended, unknowable consequences.


But there are reasons to stay optimistic, or at least to act with optimism.


We have an amazing superpower in our capacity to feel higher-order emotions, to tap into our own memories and those of previous generations that have been captured in oral traditions, books, artefacts, even Wikipedia. Add these to our extraordinary capacity to bond together at all scales, from the smallest pair of friends through to our largest groupings with over a billion people in them, and we clearly have the latent adaptive capacity we need.


We just need to master the development laid out in this book so that we all use our emotions, rather than being used by them. Grow our interconnectivity and our uniqueness. Tap into our memories of times when crises have led to the good things we have today. Use these to generate our new collective cultural meaning-making stories, and your individual new meaning-making stories; ones that are right to shape the reality we need to create the thriving version of the world that will come after we have multi-solved\index{multisolving} our climate emergency,\index{climate!emergency} and all others.
\section{Integrating what is separated}
One reason why our economy is failing to do the multisolving\index{multisolving} across all emergencies needed, so that we can thrive today and throughout your life, and our children throughout theirs, is because neoclassical economics\index{economics!neoclassical} is too wedded to classical physics instead of relativity and quantum physics. Even more, there is too much scientism\index{scientism} in economics, because of the paucity of rock-solid data gathered using double-blind studies with reliable control groups and robustly evaluating the validity of all assumptions, such as the assumption of ergodicity\cite{buchanan-ergodicity, peters-ergodicity-economics}.\index{ergodicity}


It's time for economics \index{economics} to become much more like relativity and quantum physics. Recognising that the vacuum, the interactivity between all parts, the effect of the observer, and the inherently unknowable nebulous nature of much of the economy is what matters. The consequence of recognising this is giving up all hope\index{hope} of predicting many specifics. All that you can do is work with probabilities, complementary pairs and paradoxical contradictions.


The two of us have been able to develop what is in this book and write about it because each of us has expertise in multiple disciplines, and each of us has learnt to integrate our expertise and that of others, quite different to us. 


Over the past couple of centuries the world has moved from polymaths who had expertise in very many fields (even though none had anywhere near the depth that our average expert has today) to our current focus on being an expert in single clearly defined disciplines.


Today we need both. Polymaths \index{polymaths} to do the job of integrating, and those with single-minded expertise that goes all the way down into the depth of single disciplines, held by people who have the inner capacity to hold expertise lightly, not strongly as an identity. A lightness which allows them to collaborate seamlessly with experts in complementary and conflicting disciplines that we can then integrate into one body of knowledge.


It's no longer about being either an expert or a generalist; it's about all of us together collaborating as one humanity with the full depth of expertise and a seamless integration of that into one whole, just as the universe is one whole universe.


This integration will of course be filled with internal contradictions, paradoxes, and tensions. But it's precisely those tensions that make it a faithful model of actuality. Do this well, in the context of a FairShares Commons, \index{FairShares Commons} and with a few multisolving\index{multisolving} interventions\cite{sawin-multisolving} we can address a wide range of our challenges in one.


Strengths come from interconnectedness,\index{interconnectedness} just as most of nature’s properties come from quantum interconnectedness, and not from each separate particle. We need to master our interconnectedness as distinct unique individuals, and each of you needs to increase the inner interconnectedness between each different part of you. The scientific, the artistic, the rational, the emotional, the individual, the member of a community, the self-interest, and the altruistic.


This integration is what the two of us took from our chance meeting in London, Graham’s life experiences growing up in South Africa, in physics, in Procter and Gamble\index{Procter and Gamble}, and as an open entrepreneur; Jack's experiences, growing up in the US, in physics, economics, founding and editing a disruptive journal, as an economics professor, and novel writer. Each of us also has other areas, like ballroom dancing, American football, rowing, free climbing, and enjoying good food.   


We took inspiration to create what's in this book from everything we have experienced in the reality of our lives. Wherever we've looked, we've seen people who have gone through the full range of human emotions. Whatever emotion you can imagine, people have been through that extreme before. The rollercoaster ride that was the transformation of South Africa\index{South Africa} captured all emotions from extreme hope and elation through to deep despair and depression in a very compact period of time and a very large number of people. 


Both of us have concluded that it's more than okay to feel the emotions, it's absolutely necessary. This is what makes us human, this is what enables us to bond together and work as a critical mass of people towards achieving impossible outcomes. Whatever you are feeling right now, whatever emotion is in the reality you are experiencing, play with it, experiment with it, and use it. Play with it as you used to when you were a young child, and use it the way that the wisest role model you can think of uses emotions to grow their interconnectedness with other unique individuals as part of our common oneness with all other human beings and life itself. Your life is real-time improv.


Ends and beginnings are just meaning-making stories. Nothing is actually an end, nor is it actually a beginning, as  \emph{The Never Ending Story}\cite{ende-never} by Michael Ende\index{Ende, Michael} perfectly captures. So as you reach the end of this book, it simultaneously is also the beginning of the next. Whatever your next is, all of you are interconnected with each other, and Jack's next, and Graham's next. What can we integrate across all of us, that is now separate? What uniqueness in each of us can we make even stronger and more powerful, together?






\section{First they ignore you \ldots}
The quote from Gandhi (\emph{First they ignore you, then they laugh at you, then they fight you, then you win.}) may be relevant to what I (remember that means, both Graham and Jack) have written in this book.\index{Gandhi, Mahatma} Or, maybe too much in this book is nonsense and you are right to ignore us. Maybe what is in this book will never get to the stage of polarisation where people start fighting against what I am writing here, let alone to the stage where this becomes the new story creating the new reality.


But I am convinced that enough of this story of the future is already here and proven, just unevenly distributed, to paraphrase Otto Sharmer. \index{Sharmer, Otto} All we need to do is take action now, in whatever way we can together, with as many other people as we can. What will you do today to change your story and so to change your reality?




\section{You're alive and breathing}
I wrote this section (Graham) early in the morning, just after sunrise on our last morning on our writing retreat in South Africa at Granny Dot’s\index{Granny!Dot’s}\index{South Africa!Granny Dot’s}. I'm feeling right now an exquisite sense of joy, simply still being alive with almost all of my faculties intact, looking at a magnificent sunrise. Clouds hug the valley, skirting the slopes rising to the mountains. The tops of the distant mountains are covered in pale pink clouds, topped by a blue sky deepening by the instant. The clouds themselves are constantly changing, and I'm imagining how the trees that have been hot and dry over the past weeks are relishing the cool, life-giving water. Heavy rain fell during the night.


I look at all of this, and I feel joyful to be alive, cannot imagine now ever having experienced depression, nor ever doubting that life will find a way into the future. At the top of my meaning-making stories right now is the power of one of the things that I use as a meaning-making story when I'm in deep doubt: I'm still breathing, and that's enough success for now.


Because I've kept taking one breath at a time and have not stopped yet, I'm here to feel the exquisite joy looking across the valley at this ever-changing cloudscape. To feel hope and optimism because I can see the huge adaptive capacity throughout nature. 


\smallskip


You are part of nature. 


You have the adaptive capacity.


To become who you can grow into.


To be for what the world needs you for.






\chapter{Ten actions for you}
\index{actions, ten (for you)|(}
\addcontentsline{toc}{chapterdescription}{If it is to be, it's up to me. No matter how small and powerless you are, you can act. Here are ten ideas, some of them within everyone's scope, around the central themes of this book, that will contribute.}


\begin{chapterquotation}
In my view, all that is necessary for faith is the belief that by doing our best we shall succeed in our aims: the improvement of mankind.\\
\raggedleft\textemdash Rosalind Franklin
\end{chapterquotation}


These ten actions range from easy ones that you can do in half an hour before you go to sleep tonight, to challenging ones that may well occupy your activities and all of your time for the rest of your life. Jack and I hope that these ten examples will trigger you to come up with many more that are uniquely yours.


If you have found this book useful, 


\begin{enumerate}
\item join our online and in-person gatherings, via the mailing list on \url{https://graham-boyd.biz/rebuild-the-economy-leadership-and-you/} 
\item tell your friends and colleagues about it, give it them as a present, 
\item and review it online, tweet, etc. Follow @GrahamBoydphd \newline @ProfJackReardon and @RebuildEcoLeadU on Twitter, post on LinkedIn (@GrahamBoydphd) or your favourite social media.
\item Become a patron, or donate once what this book was worth to you, so that we can accelerate creating videos, trainings, writing the next books, and continue developing the General Theory of Economies, via \url{graham-boyd.biz/rebuild-the-economy-leadership-and-you/} 
\item Take one of Evolutesix’s (\url{http://evolutesix.com}) programmes.
\item And if you are interested in founding, working, or investing in FairShares Commons startups; or using this in your company; keep an eye on \newline \url{http://graham-boyd.biz} and \url{http://evolutesix.com}.
\end{enumerate}


Do not hesitate to do what you can now. Every small step that you take now will make it easier for you or someone else to take another step later. As Granny Weatherwax\index{Granny!Weatherwax} says, in \emph{I Shall Wear Midnight}\cite{pratchett-wear-midnight}, 
\begin{quote}
‘You've taken the first step.’ \newline
‘There's a second step?’ \emph{said Tiffany.} \newline
‘No; there's another first step. Every step is a first step.’
\end{quote}
Each first step becomes successively easier because of the step you took yesterday, and the step that your friend took this morning. 


Once we have our first fully functioning ecosystem of FairShares Commons companies,\index{FairShares Commons!company} it will be so much easier to take the first step of creating the second ecosystem. That will make it easier to take the first step of connecting the two ecosystems to form an ecosystem of ecosystems.\index{ecosystem} Each step will make it successively easier. And once we have all the companies in the world running as FairShares Commons (or whatever improvement we find by taking those first steps) then it will become very easy to steadily improve our new regenerative economy.


\begin{enumerate}
\item Keep an eye open for your stories,\index{stories} work hard on lifting the hidden ones that you are subject to into visible ones. Notice the reality that your stories are shaping. Where your reality is a poor match to actuality, use the approaches in this book to experiment, improvise, and create the experiences of new stories that will rewrite your old stories.     


\item Take a sabbatical for a year, as an accelerated boot camp to give you an experimental laboratory and experiences to rewrite your story to work on your own stories. Read books that give you grounded reasons to hope. I’m eagerly awaiting Katherine Hayhoe’s,\index{Hayhoe, Katherine} \emph{The Answer to Climate Change: And Why We Can Have Hope}\cite{hayhoe-hope}, due out in September 2021. \index{The Answer to Climate Change}


\item Make your money matter. Read the excellent book by impact investor Ben Bingham, \emph{Making Money Matter}\cite{bingham-money}\index{Bingham, Benjamin}. Write to your pension fund (if you are lucky enough to have a pension), ask them what industries they are investing your pension in, and demand they invest it in businesses that are creating the kind of future you want to retire into rather than harming it. Do you really want your pension destroying the capacity of your planet to give you a comfortable retirement? Invest enough of the money that you are putting aside for your future in companies that are creating a viable economic future for all of us\textemdash FairShares Commons companies, or any other organisation that is free, purpose-driven, and focused on systemic regeneration.\index{FairShares Commons!company} 


\item Start up your own Adaptive Organisation, \index{Adaptive Organisation} incorporated at Level 5 (FairShares Commons company), using Level 4 or 5 processes for roles and tasks, and at least Level 4 or 5 developmental practices, like the Adaptive Way.\index{Adaptive Way}


\item If you are an economist,\index{economists} get to work gathering data from real businesses to support or disprove everything we describe in this book; contact Jack about the research journal we intend starting. 


\item Speak to a neighbour, colleague, or friend about what you have learned. Write to your local and national politicians to support creating local regenerative circular ecosystems addressing city level challenges. 


\item Identify allies, both individuals who resonate with this, and organisations who are likely to resonate with this. Identify the common ground that unites you and the uniqueness that each of you brings that adds value to that common ground. Get into dialogue to stand together in that common ground and bring all of your uniqueness to bear. 
 
\item Start a local business ecosystem using one or more complementary currencies,\index{complementary currencies} so that the ecosystem continues to thrive, regardless of fluctuations in the global financial economy. 


\item Sit down, and imagine you are sitting one year in the future. Write a letter to someone important to you about what you have done in the past 12 months that has taken you a step closer to being part of an Economy of the Free. \index{Economy of the Free} (It's vital to write this in the past tense\cite{zander-art}, sitting a year in the future and reflecting back over the past year, not in the form of a resolution of what you will do in the next 12 months. Remember, as described in Chapter~\ref{chapter:who-am-i-meaning}, resolutions often trigger your defensive stories, whereas a new story of what has happened that you aspire to live into has more power.)


\item Apply all three rules in the best balance you can: 1) Care for yourself, 2) Care for the other, and 3) Care for the whole, for whatever definition of other and whole you find fitting.
\end{enumerate}