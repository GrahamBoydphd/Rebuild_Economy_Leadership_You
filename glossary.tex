%Ergodicity new.
\newglossaryentry{Ergodicity}
{
  name=Ergodicity,
  description={is what you have if random events happening in a sequence have the same expected value as the same random events happening independently. Technically, the ensemble average is the same as the time / path average. Ergodicity is not true for most business activities, and our lives: the time average is very different to the ensemble average we were taught to use.}
}
\newglossaryentry{Reality}
{
  name=Reality,
  description={is, in our narrow usage here, your inner experienced reality. Your reality comes from the limited elements of actuality absorbed, filtered, and modified by your senses. It is then shaped or distorted by your nature, sense-making, and meaning-making. So each of us experiences a unique reality. None of us can ever directly experience all actuality, nor another’s reality.}
}


\newglossaryentry{Actuality}
{
  name=Actuality,
  description={is what actually is. None of us can ever experience actuality directly and totally, rather each experiences our own unique reality.}
}




\newglossaryentry{Scientism / cargo cultism}
{
  name=Scientism,
  description={we use narrowly for the superficial application of science, or dressing opinions up in the language of science, without scientific rigour. This is at best naively harmful, at worst malicious manipulation. Scientism in our usage is synonymous with cargo cultism.}
}


\newglossaryentry{General-Relativity}
{
  name=General Relativity,
  description={is the current best description of what we can say about the physics of gravity. In general relativity all of the effects are a consequence of the geometry of spacetime, not of any gravitational force.  (Often represented with the image of a cannonball on a rubber sheet, stretching it down.)}
}  


\newglossaryentry{Theory}
{
  name=Theory,
  description={A theory, in strict usage in science, unlike in daily language, is the current best falsifiable description of what we can say about how the world works; that has survived multiple attempts using rigorous processes to prove false.  Conventional daily usage is synonymous with an hypothesis or phenomenological model in science.}
}




\newglossaryentry{general-theory-of-economies}
{
  name=General theory of economies,
  description={is based on the uniqueness of each individual’s experienced reality. It recognises that every decision is rational within the frames of reference, meaning\hyp{}making, and sense\hyp{}making of each individual. Instead of seeing decisions as inherently rational or irrational, in a general theory the focus is on how one person’s experienced reality and frames of reference  can be transported to another person’s, so as to make the relativity visible.}
}




\newglossaryentry{ocracy}
{
  name=Ocracy,
  description={is our collective noun for all of the different implementations of the original philosophy of August Comte, including Holacracy, sociocracy, and Sociocracy3.0. The philosophy describes how peers rule, rather than a distinct class of rulers. Sociocracy is from socius (companions, colleagues) and cratia (the ruling class), i.e., governance by colleagues not bosses.}
}






\newglossaryentry{meaning-making}
{
  name=Meaning-making,
  description={ is the final step, after sense\hyp{}making, in constructing your internally experienced reality. You use your meaning\hyp{}making templates, or stories, to give meaning to what you have taken in of actuality.}
}




\newglossaryentry{sense-making}
{
  name=Sense-making,
  description={ is the second step in constructing your experienced reality. Sense-making is  using your capacity for logical and post\hyp{}logical thought forms to assemble the puzzle\hyp{}pieces you have taken in, prior to your attributing meaning using your meaning\hyp{}making stories. The smaller your capacity, the less actuality you take in, and the more you distort it to fit into your capacity. The limits to your capacity for sense\hyp{}making limit the meaning you can make, and hence the reality you experience.}
}




\newglossaryentry{adaptive}
{
  name=Adaptive,
  description={points at the need to change who we are, to change our Size of Person by growing both our capacity for sense\hyp{}making and our meaning\hyp{}making stories, used with Capacity, Way, Organisation, and Challenge.}
}




\newglossaryentry{lens}
{
  name=Lens,
  description={ refers to everything and anything you use in constructing your experienced reality out of actuality. The lens(es) you use pre-determine the reality you can construct, by hiding some parts of actuality, magnifying others, and distorting all. Choose your lenses wisely (though you may not be free to choose in the context you are in), to bring what is important into sharp focus, and hide what is unimportant noise. Never imagine that you experience what actually is; your lenses are always present.}
}




\newglossaryentry{cognitive-developmental-framework}
{
  name=Cognitive-Developmental Framework,
  description={(CDF) is the framework developed by Otto Laske describing how we construct the reality we experience in a sequence of taking in elements of actuality, then assembling them using our sense\hyp{}making capacity, then attributing meaning using our meaning\hyp{}making stories.}
}




\newglossaryentry{economy}
{
  name=Economy (An),
  description={is a tool used to do the job of provisioning; i.e., transporting a capital from where it is abundant to where it is needed.}
}




\newglossaryentry{capital}
{
  name=Capital,
  description={is something that has actual or perceived value. Examples include energy in nature, time and creativity in humans, relationships in society, manufactured capital, and of course financial capital. Every capital has an associated currency with it, e.g. money is the dominant currency of financial capital.}
}




\newglossaryentry{currency}
{
  name=Currency,
  description={is a tool used to attribute, store, or trade in the value of an associated capital. Each capital has one or more intrinsic currencies that fully reflects the nature and value of that capital. So time is best reflected in a time currency, not in money; energy in an energy currency, not in money; and only positive interest debt is correctly represented with money. Money is often confused with currency; and in some usages our definitions of money and currency are swapped.}
}




\newglossaryentry{Complementary-pairs}
{
  name=Complementary pairs,
  description={also called conjugate pairs are central to quantum physics. These are two apparently different entities that have a deeper relationship. They may even be perceived as mutually exclusive. Position and momentum, particles and waves are common in physics. Actuality is filled with complementary pairs. }
}




\newglossaryentry{FairSharesCommons}
{
  name=FairShares Commons,
  description={is one way of constructing a free company, suited to building a regenerative Economy of the Free. The FairShares Commons includes all relevant stakeholders in governance and wealth sharing, and is inherently a protected commons for the benefit of current and future generations of stakeholders. Stakeholders can include abstract institutions like cities, nations, the environment, etc. }
}




\newglossaryentry{ground-pattern}
{
  name=Ground Pattern,
  description={is the foundation of the Evolutesix Adaptive Way. It describes how we construct our experienced reality and how that then determines our behaviours. It is especially useful when used to explore our meaning-making and sense-making by working back from behaviours we judge as limiting, bad habits, or that block us from achieving our goals.}
}




\newglossaryentry{Thought-forms}
{
  name=Thought forms\textemdash post logical / post rational,
  description={are the 28 different thought forms that we begin developing after we have mastered sufficient logical thinking. These forms of thought are based on opposites, and are vital to grasp aspects of actuality that run counter to binary logic.}
}




\newglossaryentry{adaptive-way}
{
  name=Adaptive Way from Evolutesix,
  description={is a set of best practices designed to enable companies to reach Level 5 on the human axis, and to enable Level 5 on both the incorporation and roles and tasks axes. It includes best practice from a wide variety of fields to enable any individual or team to use the cognitive developmental framework in daily practice, transforming tensions into growth, along the three domains of growing your sense\hyp{}making capacity, your stage of meaning\hyp{}making development, and your subtlety with your nature.}
}




\newglossaryentry{adaptive-organisation}
{
  name=Adaptive organisation,
  description={refers to any organisation that is at least at Level 4 on all three axes, and striving towards Level 5 on each.}
}




\newglossaryentry{requisite-organisation}
{
  name=Requisite Organisation,
  description={refers to the organisational design framework developed by Elliott Jaques. He identified that organisations had an intrinsic hierarchy given by the irreducible complexity of different strata of work. In each stratum a role has a certain size (Size of Role) and requires someone with a matching Size of Person to execute the role effectively, without risking burnout or poor performance, neither through being bored nor overstretched. A common cause of business failure is when a manager is smaller than the role they fill.  }
}




\newglossaryentry{regenerative}
{
  name=Regenerative,
  description={is growing at least one of the non-financial capitals, such as natural or human capitals. A regenerative business is designed to intrinsically multiply all capitals it touches, not just financial.}
}




\newglossaryentry{neoclassical economics}
{
  name=Neoclassical economics,
  description={a major school, or ideology, within economics characterized by focus on the rational individual (with one rational applied to everyone) seeking to optimise their goals within an overall context of equilibrium. It was developed in the late 19th century.}
}




\newglossaryentry{rational}
{
  name=Rational,
  description={from the Latin word for reason, it means the capacity logically think through a situation to connect facts with premises. So an economic agent is rational if their behavior is congruent with their overall objectives. Neoclassical economics constricts rationality to maximising self-interest.}
}




\newglossaryentry{quantum-physics}
{
  name=Quantum physics,
  description={quantum comes from the Latin quantus meaning how much. Physics comes from the Greek word nature. Quantum physics investigates energy and matter at the atomic and the sub-atomic levels, where matter and energy comes in discrete indivisible chunks.}
}